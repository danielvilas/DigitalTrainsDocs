% !TeX encoding = UTF-8
% !TeX spellcheck = es_ES
% !TeX root = ../../main.tex

\epigraph{Todo viaje, por largo que sea, empieza por un solo paso}{Lao Tse}

\begin{abstract}
¿Porque este documento? ¿Que es lo que veremos en él? Empezamos un camino, este pdf, post o una maqueta, y por algo hay que empezar. Así que este capitulo trataremos de presentar como orientamos el resto de los capítulos, tanto en forma como en contenidos.
\end{abstract}

\section{Introducción}
La motivación que hay detrás de este trabajo no es otro que ir documentando un viaje con la esperanza de que sirva a otras personas. Este es viaje el del autor mientras crea la maqueta de trenes, a la vez, servirá para ir llenando un sentimiento de necesidad de pulir sus habilidades comunicativas en el formato académico.

En este aspecto se ira escribiendo el texto siguiendo un poco la normativa y estilismos  recomendados para los textos académicos. Potenciando el uso de la pasiva y una estructura que contemple al menos los siguientes apartados, aunque no sean con esos nombres
\begin{itemize}
	\item \textit{Resumen}, una breve explicación del capitulo, lo que se espera de el.
	\item \textit{Introducción}, donde se exponga el problema y situe al lector en contexto.
	\item \textit{Estado del arte}, 
	\item \textit{Experimento} o \textit{Texto principal}
	\item \textit{Resultados}(Opcional) 
	\item \textit{Discusión}
	\item \textit{Conclusiones}
	\item \textit{Próximos pasos}
	\item \textit{Bibliografia} y \textit{Referencias}
\end{itemize}
A lo largo de los   
\cite{acemoglu2000} Es una prueba

\section{Estado del arte}
``No conviene confundir maqueta compacta con óvalo.

Y de ahí que sus diferentes elementos, entre otros:
\begin{itemize}
	\item Tema principal
	\item Forma de la maqueta
	\item Flujo del tráfico
	\item Plano de vías
	\item Orografía
	\item Instalaciones ferroviarias
	\item Circulaciones
\end{itemize}


Es en la meditada ELECCIÓN de todos esos elementos interrelacionados a introducir en la maqueta (y que nunca es sólo EL PLANO DE VÍAS) lo que le dará juego sin fin a la vez que funcionalidad, más allá del grado de destreza y oficio modelísticos. Cuando también estos entran en escena, es cuando se habla de realismo, pero englobando las anteriores, que muchas veces se olvidan.

El foro está llenito de buenas maquetas modulares, módulos, dioramas, maquetas compactas y grandes maquetas.
Y todas las que son admiradas y reciben mayores elogios interrelacionan muy hábilmente los ingredientes anteriormente mencionados.

Y, por la misma razón, la vegetación convincente no se obtiene tanto por su altura a escala, que también, sino quizás más por cómo ese árbol en concreto se relaciona con todo su entorno y su disposición concreta en la maqueta en relación a lo demás.

MODELAR ES ELEGIR CON BISTURÍ: componentes, ángulo de visión, perspectiva y fondo. Y eso es lo que tienen las buenas maquetas, sean grandes, pequeñas, con forma de roscón de reyes o de txapela vasca''` \cite{carrington2021}

\section{Bibliografía}
\printbibliography[heading=subbibliography]
