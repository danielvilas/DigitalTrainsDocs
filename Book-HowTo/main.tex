% !TeX encoding = UTF-8
% !TeX spellcheck = es_ES
% !TeX root = main.tex

%%%%%%%%%%%%%%%%%%%%%%%%%%%%%%%%%%%%%%%%%
% The Legrand Orange Book
% LaTeX Template
% Version 2.4 (26/09/2018)
%
% This template was downloaded from:
% http://www.LaTeXTemplates.com
%
% Original author:
% Mathias Legrand (legrand.mathias@gmail.com) with modifications by:
% Vel (vel@latextemplates.com)
%
% License:
% CC BY-NC-SA 3.0 (http://creativecommons.org/licenses/by-nc-sa/3.0/)
%
% Compiling this template:
% This template uses biber for its bibliography and makeindex for its index.
% When you first open the template, compile it from the command line with the 
% commands below to make sure your LaTeX distribution is configured correctly:
%
% 1) pdflatex main
% 2) makeindex main.idx -s StyleInd.ist
% 3) biber main
% 4) pdflatex main x 2
%
% After this, when you wish to update the bibliography/index use the appropriate
% command above and make sure to compile with pdflatex several times 
% afterwards to propagate your changes to the document.
%
% This template also uses a number of packages which may need to be
% updated to the newest versions for the template to compile. It is strongly
% recommended you update your LaTeX distribution if you have any
% compilation errors.
%
% Important note:
% Chapter heading images should have a 2:1 width:height ratio,
% e.g. 920px width and 460px height.
%
%%%%%%%%%%%%%%%%%%%%%%%%%%%%%%%%%%%%%%%%%

%----------------------------------------------------------------------------------------
%	PACKAGES AND OTHER DOCUMENT CONFIGURATIONS
%----------------------------------------------------------------------------------------

\documentclass[11pt,fleqn]{book} % Default font size and left-justified equations

\input{structure.tex} % Insert the commands.tex file which contains the majority of the structure behind the template
\usepackage[spanish]{babel}
\usepackage{epigraph}
\setlength \epigraphwidth {0.65\textwidth}
\renewcommand {\epigraphflush}{center}

\let\originalepigraph\epigraph 
\renewcommand\epigraph[2]{\originalepigraph{``#1'' }{\textsc{#2}}}
%   Reduce the margin of the summary:
\def\changemargin#1#2{\list{}{\rightmargin#2\leftmargin#1}\item[]}
\let\endchangemargin=\endlist 

\newenvironment{abstract}{\begin{center}\bfseries{Resumen}\end{center}\centering\itshape\begin{changemargin}{2cm}{2cm}}{\end{changemargin}}

%\hypersetup{pdftitle={Title},pdfauthor={Author}} % Uncomment and fill out to include PDF metadata for the author and title of the book

%----------------------------------------------------------------------------------------

\addbibresource[label=intro]{chapters/00_Intro/intro.bib}
\addbibresource[label=jugar]{chapters/01_jugar/jugar.bib}
\addbibresource[label=jugar]{chapters/01_Ing_02_requisitos/requisitos.bib}

\begin{document}

%----------------------------------------------------------------------------------------
%	TITLE PAGE
%----------------------------------------------------------------------------------------

\begingroup
\thispagestyle{empty} % Suppress headers and footers on the title page
\begin{tikzpicture}[remember picture,overlay]
\node[inner sep=0pt] (background) at (current page.center) {\includegraphics[width=\paperwidth]{background.pdf}};
\draw (current page.center) node [fill=ocre!30!white,fill opacity=0.6,text opacity=1,inner sep=1cm]{\Huge\centering\bfseries\sffamily\parbox[c][][t]{\paperwidth}{\centering Como hacer una maqueta de trenes ...\\[15pt] % Book title
{\Large ... y no perderse en el intento}\\[20pt] % Subtitle
{\huge Daniel Vilas}}}; % Author name
\end{tikzpicture}
\vfill
\endgroup

%----------------------------------------------------------------------------------------
%	COPYRIGHT PAGE
%----------------------------------------------------------------------------------------

\newpage
~\vfill
\thispagestyle{empty}

\noindent Copyright \copyright\ 2021 Daniel Vilas\\ % Copyright notice

\noindent \textsc{Publicado por Self-Pubished}\\ % Publisher

\noindent \textsc{https://mimaquetaarduino.wordpress.com}\\ % URL

\noindent Este trabajo de Daniel Vilas esta licenciado bajo Attribution-NonCommercial 4.0 International. Para ver una copia de la licencia acceda a \url{https://creativecommons.org/licenses/by-nc/4.0/deed.es}. A menos que lo exija la ley aplicable o se acuerde por escrito, el software distribuido bajo la Licencia se distribuye en un \textsc{``tal cual'', sin garantías ni condiciones de ningún tipo}, ya sea expreso o implícito. Consulte la Licencia para conocer el idioma específico que rige los permisos y las limitaciones de la Licencia.\\ % License information, replace this with your own license (if any)

\noindent Editado con \LaTeX{}

\noindent \textit{Primera Impresión, ? 2021} % Printing/edition date

%----------------------------------------------------------------------------------------
%	TABLE OF CONTENTS
%----------------------------------------------------------------------------------------

%\usechapterimagefalse % If you don't want to include a chapter image, use this to toggle images off - it can be enabled later with \usechapterimagetrue

\chapterimage{chapter_head_1.pdf} % Table of contents heading image

\pagestyle{empty} % Disable headers and footers for the following pages
%\renewcommand{\contentsname}{Contenidos}}
\tableofcontents % Print the table of contents itself

\cleardoublepage % Forces the first chapter to start on an odd page so it's on the right side of the book

\pagestyle{fancy} % Enable headers and footers again

%----------------------------------------------------------------------------------------
%	PART
%----------------------------------------------------------------------------------------

\part{Motivación}

\chapterimage{chapter_head_2.pdf} % Chapter heading image

\chapter{Introducción}
%!TeX encoding = UTF-8
%!TeX spellcheck = es_ES
%!TEX root=../../main.tex

\epigraph{Las normas están para cumplirlas, pero cuando se hacen por el beneficio o mejora de todos, si no, es el capricho de alguien que no está trabajando por el colectivo}{Ulises Barrera}

\begin{abstract}
Ponemos unas reglas y disponemos de ellas como una herramienta para facilitarnos el desarrollo de nuestra maqueta. ¿Que nos motiva a tener unas reglas?, ¿Son necesarias?
\end{abstract}

\section{Introducción}
Si bien la cita de Ulises Barrera se refiere a un evento deportivo, ante decisiones arbitrarias de sobre que coches pueden o no disputar una carrera, es un buena explicación de por que existen la reglas, para el beneficio del colectivo y no propio. No sin razón podemos preguntar ¿Que colectivo? si total, la maqueta es para mi mismo y para nadie más.

En el futuro, tendremos que modificar la maqueta, ya sea por mantenimiento o por que queramos ampliarla. El colectivo seremos nuestra versión futura y seguramente no nos acordemos de porque hicimos tal cosa o que cable es el que lleva la alimentación a la vía. Ya que como dice un gran filosofo:

\epigraph{Cuando hice este código solo yo y Dios sabíamos lo que hacia, ahora solo Dios lo sabe}{Comentario anonimo en internet}

Sabias palabras que medio en broma, medio en serio nos muestra la debilidad de nuestra memoria.

\section{Estado del arte}
Hoy por hoy existen muchas normas a la hora de hacer una maqueta. Prácticamente cualquier persona con un blog, canal de youtube o en un foro, expone sus normas, algunos humildemente pero otros de manera tajante. En este apartado trataremos de categorizar y recopilar las normas mas importantes que hemos encontrado.

Las categorías las organizamos según su rango de aplicación, de mas global a más especifica. Dando se la casualidad, que serán de las que menos apliquemos a menos y en caso de seguirlas mal de las que tienen más efecto a menos. Siendo estas:
\begin{itemize}
	\item \textbf{Legislativas}: Las que pone un gobierno o autoridad, que puedan afectar a nuestra maqueta. Suelen ser de seguridad y de sentido común.
	\item \textbf{Para fabricantes}: Son las normas que las asociaciones de fabricantes han puesto para que sus productos sean compatibles, alguna de ellas nos impactara en el diseño.
	\item \textbf{Para Módulos}: Son normas para hacer una maqueta de módulos intercambiables. debemos seguirlas si queremos ir a encuentros y que se pueda unir al resto.
	\item \textbf{Especificas o locales}: Estas las estableceremos para una maqueta en concreto, o si estamos en alguna asociación, las que ponga para poder hacer la maqueta entre varios socios.
\end{itemize}

\subsection{Normativas Legislativas}
El marco legal vigente nos establece una serie de normas en cuanto las actividades que se pueden realizar en según que sitios. No en todos los sitios, aun siendo de nuestra propiedad, podremos construir una maqueta de tren. En general estas normas son de seguridad, y como suele pasar con las leyes sobre seguridad, se ponen tras accidentes donde la gente ha resultado herida. 

Para una maqueta personal y pequeña (de una habitación normal) casi seguro que no haya muchas leyes que nos impacten, mas allá de las normas de convivencia. Aun conviene conocer ciertas normas que nos puedan impactar. 

Conviene conocer las normas que las autoridades locales tengan, podemos ver las siguientes:

\begin{itemize}
	\item \textbf{Reglas de Convivencia}: Básicamente, son el ruido máximo que podemos hacer y en que horas. Pero depende de como queramos "explotar" la maqueta algunas afecten más o menos.
	\item \textbf{Reglas de Construcción}: Aqui tendremos que mirar, si existe alguna ley o normativa que nos indique como debemos construir la maqueta, cuanto puede pesar. En que zonas de una casa. También en este apartado vemos los materiales que se pueden usar o no, por si resultan ser tóxicos en caso de incendio.
	\item \textbf{Reglas Eléctricas}: Puesto que vamos hacer una instalación eléctrica debemos conocer la normativa, para no sobrecargar los conductores. Seguramente simplemente con usar varios enchufes de la habitación sea suficiente.
	\item \textbf{Reglas Sanitarias}: Desde la ventilación que deba tener nuestra habitación, hasta los sanitarios que deba tener.
	\item  \textbf{Normativa de Actividades Económicas}: Si se va realizar una actividad económica en torno a la maqueta es necesario conocerlas. No es el objetivo de estos artículos desarrollar un plan de negocio, y si el lector esta planteándose montar un negocio, seguramente la parte de construcción ya la tenga más que superada.
	
\end{itemize}
Es cierto que esta lista se desarrolla no descartando ninguna para abarcar desde maquetas pequeñas a grandes como "Miniature Wünderland" pasando por el profesional que se dedica a construir maquetas o módulos para otros. Y que por ello muchas normas de esta categoría no se aplicaran, o podemos simplificarlas. También es cierto que ignorarlas (hasta el punto de hacer lo contrario) puede ser fatal.

En general, un maquetista que usa una habitación de su casa, o como mucho un anexo de la casa del pueblo. Solo tiene que preocuparse por no poner materiales peligrosos, no pasarse de peso (para que el suelo no se caiga), de que los cables de luz sean lo suficientemente grandes y de no hacer mucho(pero mucho) ruido por las noches.

Si somos un grupo con un local, deberemos tener en cuenta alguna norma más, como la sanitaria, pero en general, con un conocimiento básico y de sentido común sera suficiente. 

Estas normas alimentaran nuestra normativa de construcción y de explotación.

\subsection{Normativas Para Fabricantes}
En el mundo del modelismo ferroviario hay dos asociaciones de fabricantes, la NMRA de Estados Unidos y MOROP de Europa y que a su vez se han coordinado para que sean compatibles y referenciándose entre si.

Estas normas básicamente se establecen para poder correr material de cualquier fabricante sobre maquetas hechas con piezas de diferentes fabricantes. De tal forma podemos usar maquinas de Piko sobre vías Rocco y mezclar coches de varios fabricantes.

Estas normas se dirigen a lo siguiente:
\begin{itemize}
	\item Enganches
	\item Protocolos DCC y LCC
	\item Características eléctricas
	\item Propiedades de las escalas (dimensiones del material)
	\item Distancias de vías (galibo, curvas,...)
\end{itemize}
Para nuestras normativas, realmente solo necesitamos seguir las distancias de vías como una recomendación para ajustarnos a radios para que puedan pasar nuestro material.
\subsection{Normativas Para Módulos}
Otra forma de hacer maquetas es por modulos normalizados, esto son por partes y luego unir cada modulo para formar una maqueta más grande usando piezas de varias personas, pudiendo organizarlas de formas diferentes cada vez.

Los modulos normalizados estan pensados para realizar encuentros de maquetistas llegados de varias ciudades y montar una maqueta nueva cada vez.

\begin{figure}[h]
	\centering\includegraphics[scale=0.10]{chapters/0X_Normativas_01_Intro/IMG_0017.JPG}
	\caption{Modulo FreeMo TT}
	\label{fig:modulott}
\end{figure}


Es necesario tener definido una serie de cosas para que sea posible conectarlos entre si en cada encuentro y a su vez no haya modulos depentientes entre si. Estos puntos se recogen en normativas y los modulos se conocen por los nombres de dichas normativas, Modulos Maquetren, Free-Mo, T-Track,... . 

La organizacion de cada encuentro decide que normativa usar y si hay alguna varicion sobre las normas oficiales. Dicha organizacion tambien suele ser la responsable de tener modulos especiales, que  se salen de la normativa pero son necesarios, como las curvas, bucles o similares.

Como minimo las normativas modular debe definir:
\begin{itemize}
	\item \textbf{Perfil de conexion}: Esto es el perfil que debe mostar un modulo para que al menos las vias coincidan al juntar. Y asi los trenes pasar de modulo a modulo.
	\item \textbf{Conexion Mecanica}: O la forma de unir y anclar dos modulos entre si. De esta manera no se podra desplazar un modulo sin mover el otro.
	\item \textbf{Conexion Electrica}: Es decir los conectores para pasar la corriente a las vias.
\end{itemize}

\begin{figure}[h]
	\centering\includegraphics[scale=0.5]{chapters/0X_Normativas_01_Intro/PERFILMQ40.jpg}
	\caption{Perfil Maquetren MQ-40}
	\label{fig:perfilmq40}
\end{figure}

Pero normalmente suelen definir tambien:

\begin{itemize}
	\item \textbf{Perfil escenico}: Ya no solo el perfil sirve para que las vias coincidan, sino tambien el paisaje, de tal forma que se vea una continuidad escenica\footnote{Al menos sin saltos bruscos, pues cada modulo tendra colores diferentes}.
	\item \textbf{Altura de los modulos}: Desde el suelo.
	\item \textbf{Dimensiones}: Esto es cuanto debe medir un modulo, una dimension ya la tenemos fijada por el perfil, pero la otra puede ser más o menos libre. Fijarla a un valor permite, al encuentro, facilitar la organizacion de la maqueta y cambiar modulos de sitio a voluntad, puesto que todos miden lo mismo. Dejarla libre, dota de mayor creatividad al creador del modulo, pero la maqueta montada requerira de un esfuerzo mayor de montaje.
	\item \textbf{Forma de construir}: Hay normativas que indican como exactamente hay que crear el modulo. Suelen ser más una recomendacion que una obligacion, pero en caso de concurso puede ser razon de descalificacion. 
	\item \textbf{Conexion Electrica}: Ademas de la conexion de la via, la conexion de energia a los accesorios. Tambien en esta punto pude ser el tipo señal (DCC, Analogica,...) a las vias
	\item \textbf{Otros}: Una normativa puede ademas definir otras cosas que considere importante como puede ser el frontal, inclusion de logotipos, cartel identificador,... .

\end{itemize}

Por ultimo algunos encuentros permiten libertad en los modulos mientras se provea de algun lado normalizado para que se puede conectar a otros modulos.
Por ejemplo permiten tener un conjunto largo de modulos propios, pudiendo conectarse entre si como el maquetista quiera siempre y cuando haya al menos un lado normalizado. Como por ejemplo una estacion larga.
  
\subsection{Normativas Especificas o locales}
Por ultimo, a manera muy local, se pueden poner otras normativas que haya que cumplir en la maqueta. En esta categoria podrian entrar tanto las que se pongan para una maqueta concreta o una asocicion ponga en su maqueta de tal forma que los socios puedan hacer partes por su cuenta y luego juntarlas en el local social.

En este ultimo caso difiere de los modulos en que el objetivo es poder partir una maqueta concreta en segmentos "fijos" y hacerlo diferentes personas. Una vez montada va a ser permanente y un segmento conectara siempre con los mismos compañeros, no tienen que ser intercambiables.

\section{Motivacion de las Reglas}
Como podemos suponer las reglas se han ido creando y modificando por una u otra razon. A veces estas razones se olvidan o desaparece la necesidad, pero la regla sigue. Lo que nos lleva al refranero popular y sus maravillosas contradicciones:

\epigraph{Las reglas estan por una razon.}{Refranero popular español}

\epigraph{Las reglas estan para romperlas.}{Refranero popular español}

La primera cita nos dice que sigamos las reglas por que tienen una razon, y la seguna nos dice que nos las saltemos, una contradiccion en toda regla. El significado completo es que todas las reglas estan por una razon, si no sabes cual es siguela por si acaso, pero si la sabes y no se aplica su razon saltatela.

Es decir debemos conocer siempre las normas que se nos aplican y su motivacion, y seguirlas siempre al pie de la letra a menos que no se apliquen a nuestro caso.

En general la motivacion para cada tipo de normativa es:

\begin{itemize}
	\item \textbf{Legislativas}: La mayor parte de estas reglas estan relacionadas con la seguridad. Segun lo que queramos hacer tendremos unas u otras.  

Dentro de la creacion de maquetas lo más probable es que nos afecten legislacion para instalaciones electricas domesticas (a menos que sea muy muy grande) y referidas al peso/montaje. Seguramente podremos consultar a un electricista o un carpintero \footnote{Un amigo te cobrara en cafes, pero siempre se puede contratar a un profesional para hacer las adaptaciones pertinentes}. En caso de querer hacer algo más complicado recomendamos contratar una asesoria legal o un gabinete tecnico.  
	\item \textbf{Para fabricantes}: De estas normas nos fijaremos sobre todo en las dimensiones minimas y recomendadas que debemos seguir, como por ejemplo el radio minimo de las curvas. Si las hemos cumplido y luego no nos va bien algun tren, podemos asegurar que el problema viene de fabrica y no por nuestra maqueta.
	\item \textbf{Para Módulos}: Si queremos hacer un modulo para encuentros debemos seguirlas al pie de la letra, pero en caso de duda consultar con la organizacion. Incluso si no vamos a hacer un modulo como tal, nos conviene conocerlas puesto que son una fuente de ideas provadas para conectar varias secciones. 
	\item \textbf{Especificas o locales}: En este punto es donde debemos hacer más esfuerzo puesto que son las que nos ahorran muchos problemas en el futuro. Nos tocara hacer nuestras propias reglas y normas, sera un esfuerzo importante, pero cuando tengamos que hacer cualquier cosa, podremos ir más rapido y perder tiempo intentando averiguar como van las cosas.
\end{itemize}

\section{Resultados y Discusion} 
No hemos querido entrar en detalles de normativas especificas, para poder abarcar muchos más lectores. Puesto que en cada lugar existiran unas normas u otras. Ya bien sea por diferente legalidad o por prefencias de la zona (T-Track vs Free-Mo).
En este apartado presentamos las recomendaciones de los autores, no tanto su clasificacion o su existencia.

\subsection{Legislativas}
En el termino Legistalivo podemos ver las diferentes normas que existen para el cableado electrico de una casa (o de uso industrial), pero estas varian de pais a pais. Estas variaciones podemos pensar que nos afectan a lo que se ve tipo de enchufe o voltaje\footnote{Si viajamos al extranjero tenemos que llevar adaptatores y asegurarnos que nuestros dispositvos soportan 110V y 220V}
pero en la practica hay muchas normativas que cumplir, tamaño del conductor, materiales validos, distanciuas entre los elementos,... Y por suerte o por desgracia varian de pais a pais o incluso de ciudad a ciudad\footnote{Realmente de provicias, condados o como sea la organizacion territorial}.

En este aspecto de normativas legales, proponemos desde este capitulo que pensemos en tres posibles situaciones y actuemos en consecuencia.
\begin{itemize}
	\item \textbf{Maqueta Grande o Negocio}: Si vamos a montar una maqueta tamaño club, para lo cual hemos adquirdo un local o hemos decidido montar un negocio entorno a la maqueta/s. Debemos aseguranos con profesionales de que el local cumple las normativas correspondientes.

Como toda adquisicion de locales requiere una adecuacion al uso que se le va dar a dicho local, recomendamos encarecidamente aprovechar este momento para contratar a los profesionales que correspondand

Al menos se tendra que revisar que el local:
	\begin{itemize}
		\item Puede soportar el peso de la maqueta junto con otros muebles\footnote{Armarios, sillas, mesas, neveras, televisiones, ...} y de todos los visitantes
		\item Cumple con la normativa electrica para la maqueta, iluminacion,...
		\item Existen los elementos sanitarios y de seguridad correspondientes segun normativa
	\end{itemize}
	\item \textbf{Maqueta Mediana Casera o en habitacion Nueva}: En el caso que no entremos en el caso anterior, pero creamos que nuestra maqueta va a ser más pesada que un armario lleno, vamos a necestir más enchufes/circuitos de los que ya tenga la habitacion o tengamos dudas sobre la construccion de la misma. Recomendamos igualmente contratar a algun profesional. 
	\item \textbf{Maqueta Pequeña Casera en habitacion Existente}: Si vamos a montar una maqueta pequeña, de poco peso
\end{itemize}
En resumen si vamos a usar un negocio, o realizar una adaptacion grande, recomendamos encaridamente contratar a profesionales que se encargen. Pero si vamos a utilizar un lugar conocido y seguro es una sugerencia, segun la confianza que tengamos en la construcción.
Ya que en general, si una vivienda es valida para habitarla, podemos creer que ya se cumplen estas normativa y por lo tanto no deberemos preocuparnos de mas.
\subsection{Para fabricantes}
Estas normas son las que desde el punto de vista de un maquetista son las que podemos decir que menos importancia tienen. Salvo las distancias minimas de margen que esas si son un poco más importantes.

Recordemos que estas normas, indican el tamaño de las ruedas, distancia entre vias, Altura de los enganches, su forma,... . Muchas de estas normas, por no decir todas, son para que el material y/o componentes de diversos fabricantes funcionen en una sola maqueta sin fallos. En esta idea, nos convertimos en "fabricantes" en el sentido de que ese material va a circular por nuestra maqueta, y ahi es donde tenemos que ver cual es el radio minimo de curvas.
Alturas entre plantas, distancias entre carriles, galibos,... .
\subsection{Para Módulos}
Los módulos han definido una serie de normas para poder crear un modúlo y llevarlo a los encuentros. Pero existen varios standares, incompatibles entre si y realmente ninguno mejor que otro. Porque cada uno se enfoca en problemas especificos de cada uno y en las preferencias de cada zona. Si hemos de elegir alguno deberia ser aquel que se ajuste a dos Características:
\begin{itemize}
	\item Nos permita acudir al mayor número encuentros.
	\item Nos acerque al mayor número de personas.
\end{itemize} 
Y como se puede ver, ninguna de ambas es de razones tecnicas, sino sociales. Los modulos son el objetivo y la consecuencia de los encuentros. Para una maqueta personal, o incluso para una grupo reducido, no tiene sentido utilizar una normativa modular estricta, pero si diseñar la maqueta pensando en juntar y separar rapidamente. Para estos casos es mucho mejor un sistema segmentado. Donde cada "Modulo" o segmento puede modificar cosas para que se ajueste mejor al tema de la Maqueta, como puede ser el perfil, dimensiones,\dots Lo que tampoco quita que los segmentos se inspiren en normativas modulares.
\subsection{Especificas o locales}
Queremos volver a recalcar lo imporante de tener una serie de normas para hacer una maqueta, y crearnoslas para nosotros mismos. Una norma puede ser el tipo de cables y sus colores para el bus DCC, esta norma es sencilla. Y, si la respetamos a lo largo de toda la maqueta, cuando tengamos un problema identificaremos rapidamente cuales son los cables del bus DCC. de los que iluminan la maqueta.
\section{Conclusiones}\section{Próximos pasos}

\section{Bibliografía}
\printbibliography[heading=subbibliography]


\chapter{Como Jugamos}
% !TeX encoding = UTF-8
% !TeX spellcheck = es_ES
% !TeX root = ../../main.tex


\epigraph{Los hombres no crecen, solo cambian el precio y tamaño de sus juguetes}{Cita anonima en Internet}

\begin{abstract}
Hay varias formas de jugar con una maqueta de tren, en este capitulo se revisaran algunas de las más comunes
\end{abstract}

\section{Introducción}
La mayoría de aficionados, como el autor, empiezan con una caja de iniciación. Montando un óvalo y dando vueltas, lo que sinceramente tras unas cuantas, es un poco más divertido que ver secarse la pintura, aunque no mucho más.

En este momento, el aficionado común, es cuando decide montarse su propia maqueta. Busca el espacio más grande que dispone y en definitiva, siendo su primera maqueta, hace una revisión del óvalo. Más grande, con más vías y desvíos, pero al final se tratará de lo mismo. Dar vueltas.

Lo cierto es que con esta maqueta habrá puesto alguna estación con apartadero, o una playa de vías, o algo con lo que maniobrar. Con esta experiencia acabará haciendo una maqueta nueva que se vaya ajustando a una forma de jugar.

\begin{figure}[h]
	\centering\includegraphics[]{chapters/01_jugar/maqueta.png}
	\caption{Maqueta un poco más compleja (pero un ovalo al fin y acabo)}
	\label{fig:maqueta}
\end{figure}

A este aficionado quizás le guste más simular las operaciones de una línea, o resolver puzzles “time-saver ” o … Al final hay tantas formas de jugar como aficionados. Y la maqueta personal se deberá hacer con forme se piense que se va a jugar con ella.

Este es un proceso que se debe pasar y existen errores que se deben cometer si se quiere disfrutar al máximo. Aunque es posible tomar algún atajo, siempre y cuando al final sepa como va a jugar. Si algún amigo tiene ya una maqueta o siendo socio de un club, tendrá a su disposición unas primeras maquetas con las que aprender cómo jugar.

Otro atajo es leer foros y artículos de revistas. Entorno al 2019/06/03 en forotrenes publicaron unos pdfs hablando sobre una serie de artículos explicando cómo planificaron una maqueta según una explotación realista. Y la conclusión que podemos obtener es la misma. Pensar antes como jugar y el contexto (de la línea imaginada) y luego diseñar la maqueta.

Bueno, teniendo claro que antes de diseñar una maqueta (e incluso antes de buscar el espacio) tenemos que saber como jugar, queremos saber que formas de jugar hay.

\section{Estado del arte}
Unos grupos iniciales de como jugar serán y de los que es fácil encontrar información:

\begin{itemize}
	\item Time-Saver o Puzzles
	\item Cartas o Americano
	\item Explotación o Europeo
	\item Exhibición
\end{itemize}
Así mismo hay otras formas que no se dicen expresamente, pero que se nombran o se intuyen.
\begin{itemize}
	\item Libre
	\item Otras formas Regladas
\end{itemize}
\subsection{Time Saver o Puzzles}
Son maquetas pequeñas y, por norma general, lineales abiertas. Una vía recta recorre toda la longitud y representa la línea principal, de la cual sale una zona de maniobras.
\begin{figure}[h]
	\centering\includegraphics[]{chapters/01_jugar/path4865.png}
	\caption{Ejemplo TimeSaver}
	\label{fig:timesaver} % Unique label used for referencing the figure in-text
	%\addcontentsline{toc}{figure}{Figure \ref{fig:placeholder}} % Uncomment to add the figure to the table of contents
\end{figure}

En la maqueta se colocan una máquina (tractor de maniobras ) y varios vagones, siguiendo una disposición inicial. Además se tiene un plano con la situación final.

El juego se trata que moviendo el tractor, enganchado y desenganchado vagones, cambiando agujas y demás hasta se llegue a la situación final.

La reglas son sencillas, los trenes van por la vía y no se puede usar la mano (salvo desenganches y descarrilamientos). Se puede jugar en solitario, o compitiendo contra otros, en dicho caso, gana quien tarde menos (en tiempo o en pasos).

\subsection{Cartas o Sistema Americano}
Se llama Sistema Americano porque es el preferido en EEUU, se trata de tener todo el material en la maqueta y que todo el mismo se mueva.

A cada vagón se le asigna una tarjeta. En la cual se le asignan 4 destinos y se trata que todos los vagones recorran sus cuatro destinos. El último destino se considera su base y debe ser el punto de inicio.

Para facilitar el juego en cada destino se pone un cajetín con los vagones que tiene. Las cartas son pequeñas y los destinos se escriben de tal forma que girándola carta quede visible en el cajetín el próximo destino del vagón. El operador de dicho destino tiene que hacer una nueva composición y enviar el nuevo tren por una línea que acerque cada vagón a su siguiente destino.

\begin{figure}[h]
	\centering\includegraphics[width=\textwidth]{chapters/01_jugar/HowToPlay-cartas.png}
	\caption{Ejemplo de Carta}
	\label{fig:cartas} % Unique label used for referencing the figure in-text
	%\addcontentsline{toc}{figure}{Figure \ref{fig:placeholder}} % Uncomment to add the figure to the table of contents
\end{figure}

Es necesario tener una maqueta grande, donde incluir múltiples destinos. Y en esos destinos tener una zona para maniobrar donde crear nuevas composiciones. En estados unidos, es mas usual vivir en unifamiliares y tener un sótano más grande donde poner la maqueta.

También se pueden jugar varios encargándose cada uno de una zona (varios destinos)

\subsection{Explotación Real o Sistema Europeo}
Como el anterior, es el preferido en Europa y por eso se llama sistema Europeo. En este caso la maqueta se diseña simulando una línea “real”. El jugador o maquetista sera el dueño de una compañía ficticia. O al menos el responsable de gestionar los trenes para dicha compañía.

Se parte de un contexto o motivo que justifique la misma, sus elementos y su paisaje. Por ejemplo una ciudad pesquera con su estación de termino con comunicaciones a la capital y un pueblo intermedio. En este ejemplo la industria pesquera recibe los pescados del pueblo y los enviá a la capital. La gente de la capital utiliza el pueblo como destino turístico. 

Con estos datos se planificaran las estaciones, dos términos y un apeadero en medio. Así mismo se incluirán las naves de mantenimiento, bases, playas de vías, etc …. Para alargar un poco el juego, se incluirá un ovalo que permita alargar el recorrido. En general es una linea punto a punto, como sucede en la realidad. Si bien se añadido un ovalo en pos de la jugabilidad.

\begin{figure}[h]
	\centering\includegraphics[width=\textwidth]{chapters/01_jugar/HowToPlay-EU.png}
	\caption{Ejemplo de explotación}
	\label{fig:explotacion} % Unique label used for referencing the figure in-text
	%\addcontentsline{toc}{figure}{Figure \ref{fig:placeholder}} % Uncomment to add the figure to the table of contents
\end{figure}

Para jugar con esta maqueta sera conveniente así mismo pensar o definir las reglas de operaciones:

\begin{itemize}
	\item Prioridades de trenes (Alta Velocidad, Pasajeros, Mercancías perecederas, …)
\item Reglas de paso (los trenes mercantes pasan por vías sin anden, la vía no desviada no tiene anden, se reserva para pasos sin parada,…)
\item Reglas de dirección (en doble vía por la derecha, los que van hacia el pueblo pesquero paran en vías pares,…)
\item …
\end{itemize}
También se tendrá que pensar en los trenes regulares (Expresos nocturnos, regionales mercancías, …) como se nombran e identifican. Con esto se podrá crear una tabla de horarios para cada estación.

En ultimo lugar, pero no por ello menos importante, unas reglas para la maqueta o de compresión de la realidad serán necesarias. En el ejemplo, se define que para ir del pueblo a la capital hay que dar 5 vueltas al ovalo, y el apeadero esta en la vuelta 3. Otra regla de compresión, es el reloj acelerado (1 hora en la realidad son 3 en el juego, por ejemplo) o que tal vagón puede cargar X pasajeros o Y toneladas y, por lo tanto, necesita un tiempo definido para carga y descarga.

El objetivo es siempre el mismo: optimizar el uso del material siendo capaces de cumplir la tabla de horarios que se haya definido. Pero se puede complicar todo lo que se quiera (mantenimiento periódico, costes de carburante,…)

Este sistema se prefiere en Europa ya que poca gente tiene un gran espacio donde montar su maqueta y es muy fácil ajustarlo al espacio disponible:

Si tenemos poco espacio podemos diseñar una estación que ocupe lo máximo posible y preocuparlos solamente por gestionar los trenes que entran y salen de ella. Para que el juego sea interesante debe ser una estación principal donde haya que entrar o sacar un tren cada poco y ademas montar y desmontar composiciones, fuera de la estación puede ser una playa de vías con varias composiciones preparadas o montarlas a mano.

Si tenemos mucho espacio podemos hacer una linea completa con varias estaciones, industrias,…

Nos ayudara mucho hacernos un cronograma de donde va estar cada tren en cada momento.

\subsection{Exhibición}

\section{¿Que tenemos ademas?}

\subsection{Libre}
Obviamente lo anterior son las categorías que he ido viendo por internet, luego cada uno se pude organizar como quiera.

En el juego libre movemos nuestros trenes sin una razón ni reglas concretas. Ahora muevo el talgo, luego el mixto, paro este aquí,….

En esta categoría incluyó la exhibición. Y es mover los trenes con el objetivo de que alguien los vea. No solo el propio tren, sino también la maqueta.

Ademas en este apartado podemos hablar de rodaje técnico, o mover las piezas para que la mecánica no se oxide…Paragraph

Esta categoría agrupa todas las formas de jugar que moveremos los trenes sin un sistema de reglas o sin objetivo claro.

\subsection{Otras Formas Regladas}
Aquí agrupo otras formas, no tan extendidas de jugar, pero regladas. Al final cada uno tiene su maqueta y juega como quiere.

Por ejemplo en las exposiciones y encuentro de modelos, se suele hacer una tabla de horarios y unas composiciones y se trata de que cada modulista se encarga de una zona (varios módulos) y se debe cumplir el horario dicho. Amen de otras reglas que dicte el organizador.

Si vamos a un encuentro de módulos como visitantes veremos a un grupo de “amigos” que se mandan trenes de uno a otro, siempre atentos a los trenes y a los controles. Si queremos preguntar algo de la maqueta, nos dirigiremos, o nos responderá alguien que no este a los mandos en ese momento.

Por otra parte si vamos a una exposición, las maquetas las habra hecho una sola persona, o un grupo reducido de personas (en general). Habra un tren dando vueltas, para que se vea como queda. Pero básicamente el tren andará solo, no se parara en las estaciones y el dueño estará vigilando que los visitantes no metan las zarpas (digo las manos) en medio de la maqueta y resolviendo dudas y preguntas de los visitantes.

\section{Resultados o Datos de interés}(Opcional) 
Si es un experimento incluir los datos o resultados obtenidos, sin valorar ni judgar. Es buen lugar para incluir otros detalles encontrados durante la escritura, búsqueda de información,....
\section{Discusión}
Este el punto para valorar los resultados y dar opiniones.
\section{Conclusiones}
Resumir y agrupar los resultados obtenidos
\section{Próximos pasos}
Escribir aquí un breve texto de lo que se hablara en otros capítulos (y que tenga referencia con este), o cosas que se dejan para realizar en un futuro fuera de este PDF.
\section{Bibliografía y Referencias}
\printbibliography[heading=subbibliography]

\part{Ingenieria en la maqueta}
\chapterimage{chapter_head_1.pdf} % Chapter heading image
\chapter{¿Ingenieria?¿en la maqueta?}
% !TeX encoding = UTF-8
% !TeX spellcheck = es_ES
% !TeX root = ../../main.tex


\epigraph{Su función principal es la de realizar diseños o desarrollar soluciones
    tecnológicas a necesidades sociales, industriales o económicas.
    Para ello el ingeniero debe identificar y comprender los obstáculos más importantes
    para poder realizar un buen diseño. Algunos de los obstáculos son los recursos disponibles,
    las limitaciones físicas o técnicas, la flexibilidad para futuras modificaciones y
    adiciones y otros factores como el coste, la posibilidad de llevarlo a cabo, las
    prestaciones y las consideraciones estéticas y comerciales. Mediante la comprensión de
    los obstáculos, los ingenieros deciden cuáles son las mejores soluciones para afrontar
    las limitaciones encontradas cuando se tiene que producir y utilizar un objeto o sistema.}
{El ingeniero segun Wikipeda}

\epigraph{Alguien que resuelve un problema que no sabías que tenías de una manera
    que no se comprende.}{Chiste anonimo en Internet}

\begin{abstract}
    Veamos que podemos entender por ingenieria y como se puede aplicar a la construccion
    de maquetas. Y quizas podamos llegar a darnos cuenta de que quizas si se estaba aplicando
    ya alguna forma de la misma. Más alla de las tecnicas aplicadas a construccion.
\end{abstract}

\section{Introducción}
Quizas la pregunta más dificil de resolver, y que nos dara un contexto para entender
este capitulo, es definir ``que es la ingenieria'' y con ello podremos ver como aplicarla
en un maqueta.

La otra pregunta a responder, es si ``merece la pena aplicarla'' a una maqueta. Tras varios
puntos de vista, o que se puede entender por ingenieria, comprobraremos que de una forma u
otra ya se estaba haciendo, aunque de forma inconsciente.

Durante la vida profesional del autor se ha encontrado con situaciones (proyectos, clientes,
estudios,...) donde se veia la parte de ingeneria desde diferentes puntos de vista. Iremos
estudiandolos en los diferentes puntos de este partado relacionandolos, cuando sea
posible, con la definicion de Wikipeda.

\subsubsection{Solucionar un/os problema/s}
Tambien se puede decir que \textit{es producir un objeto o sistema}. Es el más obvio, aplicamos
ingenieria para conseguir un objeto, una maqueta en este caso, que antes no teniamos.

Si bien es cierto que la idea que se tiene, es que, solo es para cosas complejas y si es
sencillo no es ingeniria. Es decir debe ser un problema nuevo y complejo y si no lo es,
no es ingenieria.

Pues todo lo contrario, no por ser sencillo o estar ya solucionado no deja de ser ingeniria,
pues el hecho de aprender una solucion de otros y adaptarla a nuestras necesidas ya puede
ser considerado como ingeneria.

Debemos darnos cuenta, que todo gran proyecto, con su problema global, podemos partirlo en
problemillas más pequeños, ``divide y venceras'', y a su vez, nos iremos encontrando con otros
que van surgiendo conforme vayamos avanzando en la construccion de la maqueta.

Desde este punto de vista, el hecho de tener una maqueta al final, cumple para ser ingeneria.

\subsection{Valorar alternivas}
Dentro de la definicion es \textit{decidir cuáles son las mejores soluciones}. Desde esta
prespectiva, ingenieria es pensar en diferentes posiblidades. Ya no solo posibles soluciones,
sino tambien de los posibles problemas que puedan sugir.

Erronenamente hay gente que limita la ingeniria a tener "un" documento que diga hay que hacer
esto que es la mejor opcion de estas planteadas. El error esta en limitarse a esto, ya que en
la relalidad solo es una parte.

Lo más obvio haciendo maquetas, es plantear varios esquemas de vias, y ver cual nos gusta.
Esto en si mismo, ya puede ser considerado como ingeneria, pero para ser más realistas,
ingenieria es llevar al limite esta idea. Como por ejemplo, arboles, comprarlos, hacerlos, si
se hacen, de hilo enrollado, de madera,\dots y asi con todo.

Vemos que desde el momento que plantemos o pensemos en dos opciones para un mismo caso, ya
cumplimos este punto.

\subsection{Proceso}
La realizacion de un proyecto de ingeniria requiere de un proceso bien definido y regulado.
Este proceso garantiza que se \textit{comprenden los obstaculos} y se genera un producto
teniendo en cuenta \textit{los recursos disponibles,las limitaciones físicas o técnicas,
    la flexibilidad  para futuras modificaciones y \dots}.

Estos procesos ademas garantizan que se van alcanzando diferentes hitos y para cada uno se
genera documentacion u otros elementos necesarios para la correcta realizacion del proyecto.
A veces esa documentacion, solo es por motivos regulatorios (nos lo exige alguien, una ley,\dots)
Otras veces son para dar instrucciones a otras personas (planos, lista de materiales, \dots)

Al proceso tambien se le puede llamar metodologia. La idea es que sea algo repetible, metodico
que par nuevos proyectos se aplique ``igual'', para que de esta forma los resultados tengan
calidad\footnote{Habria que definir calidad, pero eso es algo fuera de este capitulo, pues
    la idea se entiende} similar.

A veces que la diferencia entre hacer ingeneria o no, es el proceso. Sobre todo en el universo
DIY o Maker\footnote{DIY: Do It Yourself, o haztelo tu mismo}. Si nos ponemos a hacer cosas
sin una metodologia, sin pensar antes, no es ingeneria. Por ejemplo, saber programar no te hace
ingenerio informatico, hacen falta más cosas. Si siguendo la idea Maker/DIY haces mesas y
no sigues un proceso, cada mesa sera diferente, pero si te defines un proceso, generas
documentacion (medidas, pasos,\dots) podras hacer mesas muy similares y cada vez mejores.

\subsection{Planificacion}
Se dice que un buen proyecto require de una planificacion exquisita, y a veces se piensa que la
ingeneria es hacer una planificacion para que todo el proyecto se haga en el minimo tiempo
posible sin que haya paradas por falta de material, herramientas, \dots.

En nuestra maqueta si un fin de semana no podemos poner las vias porque no tenemos más,
no pasada nada. Pero no se puede parar la construccion de un puente por que nos quedamos
sin cemento. Es labor de alguien preever cuando nos quedaremos sin el y pedir con
antelacion sabiendo lo que se tarda en recibirlo.

Obviamente en nuestra maqueta no debemos ser tan criticos como en un proyecto de ingenieria.
Pero siempre tenemos una lista de tareas por hacer y su secuencia. No es necesario utilizar
herramientas de planing, como el PERT o GANTT\footnote{Formas graficas de ver la dependencia
    entre tareas y su duración}, pero si llevar un pequeño control de tareas o calendario de
cuando pensamos lo que necesitaremos, para no estar parados.

Mientras tengamos un control de los pasos a realizar, tambien cumplimos con este punto.
\subsection{Optimizacion}
Una de las primeras definiciones de ingenieria incluia la optimizacion del uso de los recursos
como su coste economico y temporal. En un lenguaje mas mundano, es minimizar los costes y
maximizar los resultados.

Por ejemplo, imagenemos los tornillos comprados en grandes cantidades, se suelen hacer
descuentos por cantidad. Si nuestro diseño no requiere de sufucientes tornillos para el
siguiente nivel (supongamos que necesitamos 99 y con 100 el precio baja un 25\%) pero estamos
cerca, puede ser interesante añadir ese tornillo que falta y asi rebajar el precio por unidad,
ganando, a priori, mas resitencia al diseño\footnote{Aunque esto dependera de donde se pone y
    algunos factores más.}.

En la practica, la optimizacion en la ingeniria, suele asociarse a planificar para acabar en
el menor tiempo posible y tener el minimo numero de elementos que cumplen los requisitos y sea
seguro. Evitando asi, la sobre-ingeneria.

En una maqueta es tipico ver la optimizacion, en terminos de meter el maximo posible de
elementos en el espacio que tenemos. Ya sean temas de vias, desvios, zonas o escenas  que se
desean. Tambien es escoger el minimo numero de elementos que conforman una escena sin perder
detalle. Un pueblo se puede representar con 2 casas modeladas y un papel de fondo, por ejemplo.
\subsection{Pensar mucho}
El autor ha odido alguna vez que los ignenieros estan todos calvos de tanto pensar, que se les
queman los pelos por el calor de pensar. La verdad es que algun calvo hay, pero como en todos
los trabajos no son tantos y mucho menos como para decir que todos\dots.

De todas formas como dice la definicion es necesario \textit{identificar y comprender los
    obstáculos más importantes}, para ello la forma más facil es haberlos sufrido en anteriodidad
o ponerse a pensar. Pero en la practica lo que hace un ingenerio es pensar antes de actuar.

En nuestras maquetas, algunas veces actuamos sin pensar, nos ponemos a hacer cosas en ella y
si nos gusta, lo dejamos asi, si no nos gusta lo cambiamos. Pero otras pensamos, buscamos
documentacion, fotos, planos de estacion y probamos antes de hacer nada. Ambas opciones son
valida, para una maqueta. Esto no quiere decir que ahora no vamos a probar cosas, ver si nos
gustan o no. Quiere decir que no vamos a pensar antes de probar.

Por ejemplo, no cojemos nuestra caja de vias y desvios para cojer al azar un elemento y
conectarlo con lo que ya tengamos. Por muy creativo que sea este ejercicio, esta muy lejos
del proceso de ingeneria general\footnote{Que bien puede hacerse ante un bloqueo creativo,
    en fases de estudios de alternativas o como pruebas de concepto}.

Pero si lo hacemos en fase de analisis y diseño, partiendo de una idea y cambiando alguna
cosa, si que lo estaria. En este ejemplos, en vez de cojer las vias al azar, podriamos poner
un diseño de playa con 3 vias y ver como añadir 2 más, jugando con las posiciones de los
desvios hasta que nos guste el resultado.
\subsection{Ciencias y tecnicas}
La defincion de la RAE, indica que la ingeneria es aplicar ciencias y tecnicas. Claramente en
al hacer una maqueta se aplican ciencias establecidas y tecnicas concretas. Pero cada ingeneria
requiere de unas ciencias y tecnicas diferentes para cada una, compartiendose entre algunas.

Intentaremos ir recopilando Ciencias y tecnicas a lo largo de diferentes capitulos, que puedan
servir al lector.

\subsection{Finalemente ¿Que es?}
Podemos simplificar la ingeniera en ``Pensar antes de actuar y ser conscientes'', ya que de ahi
se pueden derivar los pilares basicos.
\begin{itemize}
    \item \textbf{Pensar antes de actuar}: O no ir a lo loco, documentarnos de como en la
          realidad se han resuelto los problemas que estamos simulando. O como otros han hecho su maqueta.
          Pensar alternativas y evaluarlas,\dots.
    \item \textbf{Planificar, para el proyecto y para el futuro}: Sin llegar a hacer un plan de
          accion, con todo el analisis que lleva un proyecto de ingeneria. Pero si tener una lista de
          tareas, y pensar que vamos a hacer en el futuro con la maqueta, (ampliarla, hacerla de nuevo,\dots)
    \item \textbf{Realizar un proceso Repetible}: Esto es más importante si vamos a hacer la
          maqueta por modulos\footnote{En este caso secciones que se pueden hacer independiente del resto,
              no solo modulos siguendo algun standard modular}. En cada modulo conviene seguir siempre el mismo
          proceso, para facilitarnos el proceso.
    \item \textbf{Documentar}: para facilitarnos el trabajo en el futuro. Dentro de 10 años no
          nos acordarmos de que cable es que va al carril derecho de la playa de vias. Pero si hemos
          documentado el codigo de colores y el esquema electrico sera más facil.
    \item \textbf{Revisar alternativas}: para tener varias opciones y verificarlas, en linea con
          pensar antes de actuar, veremos una fase de analisis y diseño, cuya mision es esto.
    \item \textbf{Ser conscientes de lo que hacemos}: y para lo que lo hacemos. Es decir tener
          en mente para que hacemos esta "sobrecarga" que es hacer cosas de ignenieros para un "juguete"
    \item \textbf{Ser consecuentes}: Con lo que se decida durante el proyecto y con lo que nos
          obliga a hacer, es decir documentar, que algunas veces podra ser aburrido.
\end{itemize}


\section{Estado del arte}
Explicar como esta actualmente el hobby o las diferentes publicaciones respecto al tema

Maquetas de iniciacion con expansiones

Maquetas hechas por ampliciones

Maquetas Ampliables

Maquetas modulares

Vuelta a empezar

\section{Experimento o Texto principal}
Bien, pensemos en un ejemplo. Hacemos una maqueta y al final tenemos un ovalo con una estacion
y una pequeña playa de vias representando una industria maderera. Pero a partir de tres
aproximaciones diferentes, sin pensar y a lo loco, pensando pero sin seguir un proceso como tal
y por ultimo usando una metodologia de ingenieria.

Este es un ejercicio mental de lo que el autor de este articulo cree que es lo más probable que vaya sucediendo con los fallos apreciados en la maqueta.

\subsection{Recién acabada la maqueta}
Una vez acabada la maqueta en cualquier caso estaremos contentos con nuestra maqueta y en los tres casos tendremos una maqueta muy similar. Para este caso vamos a suponer que los tres casos dan lugar a una maqueta exactamente igual\footnote{En realidad habría pequeñas diferencias, en este apartado queremos estudiar la forma de pensar a largo del tiempo y enseñar que aun siendo iguales, son diferentes} y veremos una serie de cosas particulares para cada caso.
\begin{multicols}{3}
	\textbf{Montaje a lo loco}
	
	Seguramente habrá sido la maqueta más rápida, pues no se habrá parado pensado mucho, quizás algunas pruebas rápidas, pero sin mucho tiempo perdido.
	
	Igualmente habrá varios detalles menores del que pensáramos que seria mejor cambiarlos ligeramente.
	
	\columnbreak
	
	\textbf{Montaje Pensado}
	
	En esta situación, nos habrá llevado más tiempo completar la maqueta, pero al haber pensado lo que queríamos, no tendremos esa sensación de cambiar algunas cosas, en todo caso alguno puntual, pero pocos.
	
	\columnbreak
	
	\textbf{Montaje siguiendo un proceso}
	
	Este caso nos va a llevar mas tiempo, ya que pensamos lo mismo o mas que en el caso anterior y al igual, es posible tener alguna duda sobre algún detalle puntual.
\end{multicols}

\subsection{Después de jugar un tiempo prudencial}
Al pasar el tiempo y después de haber jugado con la maqueta veremos cosas que no nos acaban de gustar del todo, o que mejoraríamos. Este tiempo, podemos definir como lo suficiente para ver los fallos\footnote{Pasar la fase de enamoramiento} y lo suficientemente corto como para recordar nuestras decisiones, si tuviéramos que dar una duración, podríamos decir un año.

En este momento ya podremos ver diferencias de como vemos nuestra maqueta, incluso siendo exactamente iguales.

\begin{multicols}{3}
	\textbf{Montaje a lo loco}
	
	Vamos a ver muchos fallos y cada vez más, nuestro malestar va a empezar a crecer.
	
	Seguramente hagamos algunos cambios, a lo que consideremos lo más grave, pero recordando lo que hicimos y por que, tampoco serán muchos.
	
	\columnbreak
	
	\textbf{Montaje Pensado}
	
	Aun habiendo pensado la maqueta veremos los fallos nuevos, en este momento recordaremos las razones de como se ha llegado a esta maqueta y los aceptaremos tal como son. Quizás, y solo quizás, realicemos algún cambio a alguna cosa  
	
	\columnbreak
	
	\textbf{Montaje siguiendo un proceso}
	
	Este caso, es esencialmente el anterior, pero más pensado y documentado, por lo que seria exactamente lo mismo: Veríamos los fallos, pero recordaríamos o leeríamos las razones y nos quedaríamos igual. Muy probablemente sin cambios.
\end{multicols}
	
Resumiendo, en este momento, para la maqueta pensada y siguiendo un proceso, no tendremos la necesidad (psicológica) de hacer cambios en la maqueta y aceptaremos los fallos tal y como son. 
En todo caso buscaremos cambios menores, modificar ligeramente el trazado de vias, cambiar alguna decoración,... si se hace algun cambio.

Mientras que para la construcción rápida y sin pensar ya empezamos a necesitar cambios. Inicialmente serán menores, pero el cuerpo nos pide más cambios.

\subsection{La madurez de la maqueta}
El tiempo ha seguido pasando, ya no nos acordamos de las razones que nos llevaron a tener la maqueta exactamente así, pero aun no es tan vieja como para desmontarla y hacer una nueva aun le queda vida. Aventurandonos a dar un tiempo nos atrevemos a decir 5 años vista desde la construcción.

En general, los fallos que hemos llegado a aceptar siguen rascándonos la nariz y nuestro malestar va creciendo. No es que no nos divirtamos con la maqueta, simplemente le vamos viendo más fallos. También habremos crecido en el hobby y la maqueta ya no se ajusta a nuestros nuevos intereses. Quizás encontremos que preferíamos material prusiano de época II mientras que la maqueta esta más pensado para material francés de época IV.

Ha llegado un momento critico para la maqueta, nos preguntaremos que haremos a continuación con la maqueta: hacer algún cambio menor, cambio mayor, ampliarla o hacer una de nuevo completamente.


\begin{multicols}{3}
	\textbf{Montaje a lo loco}
	
	Los fallos que vemos cada vez nos molestan más, no sabemos las razones de porque y cada vez las recordamos menos, pero si que fuimos montando-la como quedaba mejor.
	
	El descontento nos pedirá hacer cambios mayores o pensar en hacer una nueva.
	
	A partir de ahora cada vez que juguemos con la maqueta veremos los fallos y el cuerpo nos pedirá ir cambiándola. No tenemos nada, ni recuerdos ni documentación, con los que justificar mantenerlos.
	
	\columnbreak
	
	\textbf{Montaje Pensado}
	
	En este caso ya no nos acordamos de las razones, solo que pensamos algunas variaciones y llegamos a la conclusión de que esta era la mejor. 

	Según nuestra nueva situación nos planteáramos hacer cambios, estos serán grandes en función de nuestras posibilidades. Nuestro descontento ira creciendo, pero como sabemos que todo esta pensado tampoco nos afectara mucho.
	
	Justificaremos los fallos que veamos sabiendo que están ahí y como hemos pensado al crear la maqueta los toleraremos. Siempre nos quedara la duda si no había más opciones y cada vez nos planteamos soluciones que nos guardamos para una futura maqueta.
	
	\columnbreak
	
	\textbf{Montaje siguiendo un proceso}
	
	En esta situación tenemos la documentación con la que hemos ido haciendo la maqueta, por lo que podemos recordar las razones y decisiones que nos han llevado ha este momento. tenemos la certeza de que esta es la mejor opción, aunque despues de unos años podríamos tener más alternativas o nuevas ideas fruto de la experiencia ganada.
	
	Igualmente y según nuestra nueva situación nos planteáramos hacer cambios, estos serán grandes en función de nuestras posibilidades. Nuestro descontento ira creciendo, pero con la documentación justificaríamos los fallos y tampoco nos afectara mucho.
	
	
\end{multicols}
Llegados a este momento la maqueta creada de forma rápida, el cuerpo nos pedirá a gritos un cambio mayor, incluso empezar de nuevo con ella, pero tiene poco tiempo como para que lo veamos como una opción, nos plantearemos cambios mayores (cambiar una parte significativa de las vías, añadir una sección,\dots)

Mientras que para la pensada, seguiremos teniendo la sensación de necesitar el cambio, no tan grande como la versión no pensada, pero si nos plantearemos cambios. La versión documentada, no nos plantearemos cambios, puesto que veremos las alternativas en la documentación y sabremos que ya las habíamos descartado.



\subsection{Los grandes cambios}
Hemos llegado a un momento que vamos hacer grandes cambios, ya sea cambiar drásticamente la maqueta, ampliarla o \dots.

Puede ser por que algunos fallos ya no los toleremos más sin que veamos necesario hacer una maqueta totalmente nueva. Pero lo más probable es que al haber evolucionado en la materia, nuestras necesidades y conocimiento han crecido y la necesidad del cambio en la maqueta sea para actualizarla a nuestros nuevos gustos. O incluso por cambios externos al hobby (más espacio, mudanzas, razones familiares,\dots). 



\begin{multicols}{3}
	\textbf{Montaje a lo loco}
	
	En este caso, la primera idea que nos vendrá a la mente, sera olvidarnos de la maqueta y construir una nueva. Si es que ha llegado a sobrevir hasta este momento\dots
	
	Pero en el caso de hacer grandes cambios, no sabremos como van las cosas. Por ejemplo, los cables que van a las vías serán cada uno de un color, por lo que no sabremos cual va a cual. En una expansión nos tocara ir con el tester probando que puntos conecta cada cable para conectar bien las nuevas vías.
	
	\columnbreak
	
	\textbf{Montaje Pensado}
	
	Seguramente en este caso, nos plateramos hacer cambios o una expansion si nuestras posibilidades lo permiten. Si no se han hecho ya dichos cambios.
	
	En cualquier caso, sera relativamente facil de expandir o cambiar, puesto que se habra llevado una pequeña coordinacion con lo que los diferentes elementos seran identificables, por ejemplo los cables llevaran un esquema de colores, y el carril de la derecha siempre sera del mismo color,\dots
	
	\columnbreak
	
	\textbf{Montaje siguiendo un proceso}
	
	Finalmente, y en esta situacion, seguirmeos el mismo proceso, plantenandos nuevos objetivos y requerimientos. Utilizaremos las lecciones aprendidas, los fallos que hemos encontrado y nos platearemos la primera alternativa, una nueva, expandir o \dots
	
	En cualquier caso tendremos documentacion que nos haran facil reusar los elementos de la maqueta.
\end{multicols}

En resumen, segun que escenario estemos lo tendremos más facil o dificil para avanzar en el hobby.

\subsection{Fallos y Tipos en la maqueta}
En los apartados anteriores hemos hablado de los fallos que vamos a tener en nuestras maquetas, por que efectivamente, todas tendran fallos. Pero el quid de la cuestion, es un fallo para quien juega o para un mero expectador puntual de la misma. 

Empezaramos definiendo un fallo como \emph{aquello que lo vemos y podemos decir ``eso esta mal'' o ``eso nos aburre''}. Bajo esta definicion las unicas personas que pueden decir que es un fallo son el maquetista y las juegan habitualmente con ellas. 

Para entender los fallos, corregirlos y evitarlos en un futuro necesitamos compender cuatro cosas sobre los mismos: que tipo de fallos es, cuando se ha creado, que lo ha causado y que efecto tiene. 

Para el ejercicio mental solo se tiene en cuenta el cuando se han creado, por que es donde puede haber más diferencias segun el estilo de creacion de la maqueta.

Los fallos los podemos clasificar por multidud de categorias como por ejemplo:
\begin{itemize}
	\item \textbf{Funcionales}: Cosas que no nos gustan de como funciona la maqueta, como podrian ser los semaforos, iluminacion, trazado de las vias,\dots Aqui entraria tambien descarrilamientos por poner mal las vias.
	\item \textbf{Esteticas}: Relacionado de como queda visualmente algo. Quizas poner casas de la montaña en zona llana caribeña no queda muy bien,\dots. Si vien la estetica es muy personal, pero si a las personas interesadas les molesta es un fallo\footnote{Si eres un visitante ocasional, podras comentarlo, pero que no coincida con tus gustos no siginifca que es un fallo}.
	\item \textbf{Realismo}: Esto es cuando la maqueta no representa algo que pueda suceder en la realidad. Podemos divir en muchas subcategorias, como de explotacion (mover los trenes como en la realidad), coherencia de epocas y lugares (mezclar materia español epoca II con aleman VI) o de coherencia escenica (mezclar edificios, no cooncordar trazado con elementos). Recordemos que los fallos solo son tales si las personas interesadas lo dicen. Especialemente en este caso, si la falta de realismo no molesta nunca puede ser un fallo. 
	
	\item \textbf{Otros}: Cada persona puede tener su propia clasificacion, y podriamos poner más, pero las quejas más oidas sobre maquetas suelen entrar en esas tres anteriores. Como por ejemplo podemos añadir aqui, el no cumplimento de objetivos planteados para la maqueta.
\end{itemize}

La causa del fallo, la consideramos practicamente irrelevante, en general va a ser por falta de habilidad en el momento de ejecucion (y se solucionara con practica o con cuidado) o por falta de informacion durante la ejecucion o que es lo mismo hemos evolucionado y tenemos nueva formacion y por ello detectamos un fallo. 

El efecto, lo podemos resumir por cuanto nos molesta dicho fallo.

Y el cuando segun se introduce el fallo, ya sea en diseño, construccion o explotacion. Dicha esta clasificación rapida, para el ejercicio mental nos vamos a quedar unicamente la idea de cuando se crea el fallo. Por que al final, los fallos van a existir independientemente de su razon y efecto. Añadiremos un cuarto momento quedando los siguientes tipos:
\begin{itemize}
	\item \textbf{Fallos de diseño}: Son aquellos que se crean cuando se esta pensando la maqueta y, para el ejercicio, se tenia informacion para evitarlo. Por ejemplo un trazado que nos aburre o una playa de vias que no permite manejar el material que queremos. 
	\item \textbf{Fallos de diseño por evolucion}: Categoria añadida para el ejercicio, Son los que se crean en el diseño, pero los vemos cuando hemos aprendido cosas nuevas o refinado nuestros gustos. Por ejemplo la playa de vias no nos sirve con el material que hayamos adquirido a posteriori. O por que hemos leido un libro que explique algo que no tuvimos en cuenta en su momento\footnote{Si ya teniamos el libro en el momento de diseñar es de la categoria anterior}.
	\item \textbf{Fallos de construccion}: Son los que se producen durante la creacion pero no es causado por el diseño. Lo más facil de entender es el ejemplo de poner mal las vias, pero tambien entraria tener que mover un desvio por que justo coincide con un nervio de la estructura y ya no nos convence su funcionomiento/posicion.
	\item \textbf{Fallos de explotacion}: Esta es la categoria par indicar que los fallos son daños que ha recibido la maqueta por la razon que sea (mudanza, mascotas, algo que se cae encima, polvo que no se puede quitar,\dots).
\end{itemize}

Es importante separar los fallos de diseño por evolucion de los que no, por que mentalmente racionalizamos los primeros con el conocimiento, \emph{tenemos este fallo, pero no teniamos forma de evitarlo por que no sabiamos...} en contra \emph{este fallo lo podriamos haber evitado por que lo sabiamos}


De estos cuatro tipos de fallos, para el ejercicio, los fallos de explotacion, son los más aletorios y dificiles de preever por que se presuponen accidentales al igual que de los fallos de diseño por evolución, no van a depender de como se haga la maqueta del resto lo vemos como en los apartado anteriores:


\begin{multicols}{3}
	\textbf{Montaje a lo loco}
	
	Se cometeran muchos errores de diseño, puesto que no hay diseño. Igualmente al ir haciendo y deshaciendo sera muy facil introducir fallos de construccion.
	
	\columnbreak
	
	\textbf{Montaje Pensado}
	
	En este caso, no habra muchos fallos de diseño, si lo hay. Al haber pensado las cosas tampoco habra muchos fallos de construccion.
	
	Pero cuando evolucionemos y veamos fallos, es probable que no podamos separa los fallos de diseño por evolucion de los de diseño real. Con la consiguiente carga mental aumentando el descontento con dicho fallo.
	
	\columnbreak
	
	\textbf{Montaje siguiendo un proceso}
	
	Seguir un proceso de ingenria no garantiza que no haya fallos de diseño, pero si que van a ser minimos, menos que el anterior. Y lo mismo para los de construccion, nos minimiza pero no elimina todos, sobre todo ya que este tambien depende de nuestra habilidad de construccion.
	
	Pero si, cuando veamos un fallo, podremos ver el diseño en la documentacion que hayamos hecho y la clasificacion, sera más facil ver si se deben a evolucion o no. Y en el caso de serlo no nos cargara tanto.
	
\end{multicols}

\section{Resultados o Datos de interés}(Opcional)
Si es un experimento incluir los datos o resultados obtenidos, sin valorar ni judgar. Es buen lugar para incluir otros detalles encontrados durante la escritura, búsqueda de información,....
\section{Discusión}
Este el punto para valorar los resultados y dar opiniones.
\section{Conclusiones}
Resumir y agrupar los resultados obtenidos
\section{Próximos pasos}
Escribir aquí un breve texto de lo que se hablara en otros capítulos (y que tenga referencia con este), o cosas que se dejan para realizar en un futuro fuera de este PDF.
\section{Bibliografía y Referencias}
\printbibliography[heading=subbibliography]
\chapter{Requisitos}
% !TeX encoding = UTF-8
% !TeX spellcheck = es_ES
% !TeX root = ../../main.tex

\epigraph{El mundo entero se aparta cuando ve pasar a un hombre que sabe a dónde va}{Antoine de Saint-Exupéry }

\begin{abstract}
    En un proyecto de ingeniería el primer paso es la captura de requisitos.
    Seguramente un perfil mas comercial dirá que el primero es venderlo, pero para venderlo hay que dar un precio
    y para dar ese precio hay que estimar cuanto costara, para ello se necesita saber lo que se quiere,
    ergo los requisitos diría el perfil más técnico, luego seguramente acabarían comentando que si toma o captura
    de requisitos. Según como sea podrán estar así horas y horas para decir lo mismo con distintos términos.
\end{abstract}

\section{Introducción}

En la realidad hay dos momentos donde se recogen los requisitos, en una fase pre-venta se recogen a alto nivel
(por ejemplo que el puente soporte el peso de 10 coches) y una vez contratado el proyecto se realiza otra fase con
más detalle (10 coches coches se convierten en X toneladas suponiendo la media de los que pasan es Y más un margen
de Z\%,…).
Estos dos momentos se suelen llamar toma y captura, pero cada metodología y/o empresa define cual de esos términos
es cada fase, incluso cambiando el nombre.

Es importante realizar esta captura lo antes posible y con el mayor nivel de detalle posible.
Cambiar los requisitos (modificarlos, quitar o añadir alguno) significa revisar todas las decisiones y acciones
realizadas hasta el momento que se puedan ver afectadas por dicha modificación.
Por lo que si se debe realizar los cambios, costara más cuanto más nos acercaremos al final.

Por ejemplo, si inicialmente tenemos un espacio de 3x4 metros y luego cambiamos a 2x3 metros.
No es lo mismo que estemos pensando el diseño de la maqueta, a tener ya las vías pegadas y empezando a hacer
la decoración escénica. Ya que esta es un requisito que afecta a muchas acciones y decisiones.
Por otra parte, otros requisitos pueden cambiarse casi sin afectar.
Imaginémonos que en un principio queremos representar una escena invernal, pero antes de hacer el paisaje se
decide cambiar por una veraniega y el impacto será mínimo puesto que aun no se había empezados.
Ojo, que quizás alguna decisión pudiera ser sido diferente, como podría ser añadir una playa con un puerto
si se hubiera decidido antes.

\section{Estado del arte}
Centrándonos en que esto es para una maqueta y no un proyecto de ingeniería al uso,
no necesitamos toda una disertación de fases, costes, análisis, gestión del cambio, impacto, etc.
Nos tenemos que quedar con la idea de definirlos lo antes posible, puesto que cambiarlo más tarde costara.
Es decir necesitamos tener el listado de requisitos y pensados con cabeza.

Para poder hablar de los requisitos debemos primero, ver lo que son y sus características. Despues en la seccion
de discusion hablaremos en más detalle\footnote{NdA: Siguiendo una estrucutra más academica}.

\subsection{¿Qué son los requisitos?}

En resumen los requisitos son únicamente el listado de cosas que tiene que cumplir la maqueta, o los objetivos
de la misma. En un proyecto de ingenierías se suelen dividir en tipos, como funcionales (Lo que tiene que hacer)
y técnicos (restricciones físicas) para la proyectos informáticos.
Cada ingeniería tiene su propia categoría y aquí estamos en un hobby de maquetas.
Por tanto los organizaremos y clasificaremos por lo que más nos interese.
Si que podría ser interesante tenerlos organizados por intencionalidad (que se quiere lograr),
escénico (tener tal y tal cosa) y físicos (donde debe caber u otras limitaciones realacioandos con espacio).

Es muy importante tener en cuenta que los requisitos son lo que luego que dictaminarán si una maqueta es un
éxito y por tanto buena. Dicho de otra forma, será una buena maqueta si cumple los requisitos u objetivos
planteados. Así que dichos requisitos deben estar escritos en algún sitio.

Los requisitos siempre responden a preguntas del tipo ¿Qué quiero…? ¿Dónde quiero…?
O ¿Qué debe …? O ¿Dónde debe…?

Algunos requisitos estarán muy relacionados entre si, quizás siendo aclaraciones o detalles de uno.
Por lo que podremos considerarlos como Sub-requisitos.

\subsection{Objetivos y Prioridades}

Los objetivos son los requisitos que nos parecen más importantes y son lo que consideramos básicos que
la maqueta debe cumplir. Son particulares de cada maquetista y ninguno es trivial. Básicamente responden
¿Qué quiero conseguir con mi maqueta? Aunque al final serán los que se consideren más importantes.

Como se puede intuir, no todos los requisitos será igual de importantes, y los menos importantes podremos
no cumplirlos o modificarlos para que se ajusten a lo que tenemos creado. Pero los más importantes deberemos
mantenerlos lo más fijos posibles.

\subsection{Detalle de los requisitos}

Los requisitos deberían ser breves, concisos, concretos y claros. Pero a su vez deben tener el mayor detalle
posible para que sea lo más fácil posible tomar las decisiones futuras. Por ejemplo veamos posibles requisitos,
que son el mismo, pero con diferente nivel de detalle.

\begin{enumerate}
    \item Quiero un parque de atracciones.
    \item Quiero un parque de atracciones inspirado en el de mi ciudad
    \item Quiero un parque de atracciones con una noria y un lago.
    \item Quiero un parque de atracciones con una noria modelo tal y un lago que tenga 3 barcas modelo tal.
    \item Quiero un parque de atracciones inspirado en el de mi cuidad, con al menos las atracciones … y siguiendo el mapa …
\end{enumerate}

Como podemos ver el número 1 es el más genérico y abstracto, pero también nos permite más flexibilidad en
decisiones futuras. Por otra parte el 4 y el 5 son los más concretos y con más detalle, son más rígidos
pero por otra parte nos fija cosas que luego nos evitamos pensar.

En el caso del 5, incluso convendría, partir el listado de atracciones en Sub-requisitos para que no quede muy
largo el mismo, pero considerarlos todos como uno, si va a ser un objetivo de la maqueta.

\subsection{Cuando tener los requisitos}

Los requisitos deben estar lo antes posible y siempre antes de que comiencen a necesitarse.
Los requisitos serán la plantilla para la toma de decisiones como la elección entre alternativas.
Por lo tanto no es necesario tener todos al principio, pero si los objetivos.

Volviendo al ejemplo anterior, suponiendo que hacemos una maqueta por módulos y vamos a tratar el modulo
donde pondremos el parque de atracciones. Realmente hasta este momento no hemos necesitado el requisito hasta
este momento, por lo tanto podremos no tenerlo o modificarlo ”sin coste” hasta ahora.
Pero si lo modificamos después de hacer el modulo, puede que tengamos que rehacerlo.

Es decir necesitaremos tener el requisito lo más detallado junto antes de usarlo, pero en el listado de
requisitos debería estar, aunque sea con baja definición, desde el momento se nos pase por la cabeza.

En el ejemplo del parque de atracciones, tendríamos el 1 al principio, más adelante cuando se planifiquen
los módulos, algo un poco más detallado como el 2 o 3, para dar una idea más clara.
Pero el momento de diseñar el modulo concreto el 4 o 5.
\subsection{Requisitos de espacio}

\section{Ejemplo Practico}
\section{Texto principal}

\section{Resultados}
%\subsection{Los requisitos de una maqueta}

La maqueta que queremos desarrollar será sobre una empresa ficticia DanielBahn y el resultado final será
el resultado de varios años de trabajo, en este momento y estos artículos se describe el proceso para un
maqueta más pequeña DanielTeppichBahn\footnote{TeppichBahn es como llaman en Alemania a las maquetas de
    tren que se montan sobre el suelo para jugar y se desmontan cuando ya se ha acabado el juego.}.
Aunque, teniendo que esta maqueta será para probar técnicas/tecnologías para la versión final.
Se tendrán en cuenta en los requisitos.

El proceso para tener los requisitos es iterativo, es decir se van escribiendo en iteraciones,
intentando ir ampliando poco a poco para cada maqueta.

Para DanielBahn tenemos los objetivos  siguientes:
\begin{itemize}
    \item Representar tres escenas inspiradas en sitios reales, por importancia sentimental
          \begin{itemize}
              \item La estación del puerto de La Coruña hasta la playa de lazareto
              \item La estación del Santo Sepulcro de Zaragoza
              \item Una estación de montaña como la de Canfranc
          \end{itemize}
    \item Desarrollar módulos electrónicos en LCC para desvíos, y paneles de control.
    \item DCC para las vias y para los mandos, cualquiera con soporte para JMRI
    \item Tener un una vía continua para realizar fotos, sin preocupaciones de tener que evitar choques contra fin de vía
    \item Tener una zona de puzzle de clasificación
    \item Ser más o menos realista en trazado y operación.
\end{itemize}


Como se puede ver son lo suficientemente concretos para poder ir diseñando y pensando alternativas,
pero tan genéricos que dan lugar multitud de opciones. Así mismo es tan a largo plazo, que es más una lista de
deseos que un listado de requisitos al uso. Tampoco hay restricciones de espacio, a la espera de realizar diseños
y buscar alternativas.

DanielTeppichBahn, por su parte será una maqueta para poner en el suelo, y de aprendizaje por los que sus
objetivos son:
\begin{itemize}
    \item Tener una maqueta pequeña para:

          \begin{itemize}
              \item Rodar los trenes
              \item Aprender técnicas
              \item Probar nuevas ideas
          \end{itemize}
    \item Debe caber “escondida” detrás de un mueble:
    \item Debe ser fácil de montar
    \item Con Electrónica para controlar desvíos, y luces hecha por mi.
    \item Debe tener los problemas típicos(Si es posible):

          \begin{itemize}
              \item Playa de vías
              \item Vías de escape
              \item Cruces
              \item Bucles
          \end{itemize}
    \item DCC para las vías y los desvíos
    \item Loconet para los mandos, módulos propios para JMRI
    \item Paneles y módulos propios en Loconet (para ahorrar cables)
    \item Basada en montaje de caja inicial
\end{itemize}

Como se puede apreciar, en esta maqueta ya aparece un requisito donde debe estar guardada, porque es el compromiso
que hemos podido llegar, para tener una maqueta. Y son más concretos, con lo que nos permite ir desarrollando
esta maqueta.

Seguramente, iremos desarrollando maquetas según el tiempo y el espacio disponible vayan variando.
\section{Discusión}
\subsection{Requisitos de espacio}

Las limitaciones de espacio, y por ende sus requisitos, se suelen tener en cuenta antes si quiera de empezar
a pensar en la maqueta. Esto es un reflejo de las situación de cuando podemos hacer la maqueta.
Normalmente cuando tenemos ya la vida resuelta y el espacio ocupado ya en la casa por estanterías, muebles
y demás.

Esto nos obliga a hacer compromisos, con otras personas, el espacio y lo que queríamos hacer, por lo que
muchas veces acabamos con una maqueta que no nos satisface del todo. Desde este apartado, abogamos por
retrasar estos requisitos lo máximo posible hasta una fase de análisis y diseño con los requisitos que se
realmente se quiere.

Para poder retrasar estos requisitos necesitamos una fase de análisis y diseño, donde veremos el espacio
que requiriéramos para el resto de requisitos y nos obligará a pensar alternativas de espacio
(alquilar un local, usar la casa del pueblo, mudarse, …), pensar en diseños para varias habitaciones, o
un cambio de diseño (pasar a escenas en varios niveles, segmentada, de maleta,…).

También nos permite probar varios diseños, sobre el papel o sobre pruebas, donde llegaremos a compromisos,
con la seguridad que hemos intentando todo lo posible por tener todo aquello que queríamos necesariamente.
Y que ademas ha sido imposible tener más.

Esto no quiere decir que tener requisitos de espacio al principio sea malo. Los tendremos si es algo ”fijo”
puesto que se ha acordado previamente con otras personas del hogar, o porque es la única posibilidad real.
Pero recordemos que luego un cambio de esto, tiene grandes implicaciones, por lo que cuanto más tarde 
lo tengamos mejor. Aunque implicara pensar más alternativas y por lo tanto más trabajo intelectual.

\section{Conclusiones}
Como hemos comentado, lo importante de los requisitos es tener un documento o listado de lo que se quiere
tener. Este listado debe ser lo mas detallado posible, pero cuando se necesite. Al principio lo tendremos 
con poco detalle y lo iremos detallando a necesidad. Nos preocuparemos más de lo que queremos y las
limitaciones intentaremos retrasarlas para buscar alternativas.

También hemos presentado los objetivos de la maqueta final, a largo plazo (DanielBahn) y otra mas cercana
en el tiempo para ir aprendiendo y tener un sitio donde rodar (DanielTeppichBahn)

Como conclusión, tengamos sentido común, y hagamos una lista de lo que realmente queramos y luego
estudiemos opciones para ver como podemos llegar a ellas. No tengamos miedo a pensar a largo plazo y
hacer maquetas intermedias para aprender, o probar cosas. 
\section{Próximos pasos}

\section{Bibliografía}
\printbibliography[heading=subbibliography]


\part{Normativa}
\chapterimage{chapter_head_2.pdf} % Chapter heading image

\chapter{Introducción a las normativas}
%!TeX encoding = UTF-8
%!TeX spellcheck = es_ES
%!TEX root=../../main.tex

\epigraph{Las normas están para cumplirlas, pero cuando se hacen por el beneficio o mejora de todos, si no, es el capricho de alguien que no está trabajando por el colectivo}{Ulises Barrera}

\begin{abstract}
Ponemos unas reglas y disponemos de ellas como una herramienta para facilitarnos el desarrollo de nuestra maqueta. ¿Que nos motiva a tener unas reglas?, ¿Son necesarias?
\end{abstract}

\section{Introducción}
Si bien la cita de Ulises Barrera se refiere a un evento deportivo, ante decisiones arbitrarias de sobre que coches pueden o no disputar una carrera, es un buena explicación de por que existen la reglas, para el beneficio del colectivo y no propio. No sin razón podemos preguntar ¿Que colectivo? si total, la maqueta es para mi mismo y para nadie más.

En el futuro, tendremos que modificar la maqueta, ya sea por mantenimiento o por que queramos ampliarla. El colectivo seremos nuestra versión futura y seguramente no nos acordemos de porque hicimos tal cosa o que cable es el que lleva la alimentación a la vía. Ya que como dice un gran filosofo:

\epigraph{Cuando hice este código solo yo y Dios sabíamos lo que hacia, ahora solo Dios lo sabe}{Comentario anonimo en internet}

Sabias palabras que medio en broma, medio en serio nos muestra la debilidad de nuestra memoria.

\section{Estado del arte}
Hoy por hoy existen muchas normas a la hora de hacer una maqueta. Prácticamente cualquier persona con un blog, canal de youtube o en un foro, expone sus normas, algunos humildemente pero otros de manera tajante. En este apartado trataremos de categorizar y recopilar las normas mas importantes que hemos encontrado.

Las categorías las organizamos según su rango de aplicación, de mas global a más especifica. Dando se la casualidad, que serán de las que menos apliquemos a menos y en caso de seguirlas mal de las que tienen más efecto a menos. Siendo estas:
\begin{itemize}
	\item \textbf{Legislativas}: Las que pone un gobierno o autoridad, que puedan afectar a nuestra maqueta. Suelen ser de seguridad y de sentido común.
	\item \textbf{Para fabricantes}: Son las normas que las asociaciones de fabricantes han puesto para que sus productos sean compatibles, alguna de ellas nos impactara en el diseño.
	\item \textbf{Para Módulos}: Son normas para hacer una maqueta de módulos intercambiables. debemos seguirlas si queremos ir a encuentros y que se pueda unir al resto.
	\item \textbf{Especificas o locales}: Estas las estableceremos para una maqueta en concreto, o si estamos en alguna asociación, las que ponga para poder hacer la maqueta entre varios socios.
\end{itemize}

\subsection{Normativas Legislativas}
El marco legal vigente nos establece una serie de normas en cuanto las actividades que se pueden realizar en según que sitios. No en todos los sitios, aun siendo de nuestra propiedad, podremos construir una maqueta de tren. En general estas normas son de seguridad, y como suele pasar con las leyes sobre seguridad, se ponen tras accidentes donde la gente ha resultado herida. 

Para una maqueta personal y pequeña (de una habitación normal) casi seguro que no haya muchas leyes que nos impacten, mas allá de las normas de convivencia. Aun conviene conocer ciertas normas que nos puedan impactar. 

Conviene conocer las normas que las autoridades locales tengan, podemos ver las siguientes:

\begin{itemize}
	\item \textbf{Reglas de Convivencia}: Básicamente, son el ruido máximo que podemos hacer y en que horas. Pero depende de como queramos "explotar" la maqueta algunas afecten más o menos.
	\item \textbf{Reglas de Construcción}: Aqui tendremos que mirar, si existe alguna ley o normativa que nos indique como debemos construir la maqueta, cuanto puede pesar. En que zonas de una casa. También en este apartado vemos los materiales que se pueden usar o no, por si resultan ser tóxicos en caso de incendio.
	\item \textbf{Reglas Eléctricas}: Puesto que vamos hacer una instalación eléctrica debemos conocer la normativa, para no sobrecargar los conductores. Seguramente simplemente con usar varios enchufes de la habitación sea suficiente.
	\item \textbf{Reglas Sanitarias}: Desde la ventilación que deba tener nuestra habitación, hasta los sanitarios que deba tener.
	\item  \textbf{Normativa de Actividades Económicas}: Si se va realizar una actividad económica en torno a la maqueta es necesario conocerlas. No es el objetivo de estos artículos desarrollar un plan de negocio, y si el lector esta planteándose montar un negocio, seguramente la parte de construcción ya la tenga más que superada.
	
\end{itemize}
Es cierto que esta lista se desarrolla no descartando ninguna para abarcar desde maquetas pequeñas a grandes como "Miniature Wünderland" pasando por el profesional que se dedica a construir maquetas o módulos para otros. Y que por ello muchas normas de esta categoría no se aplicaran, o podemos simplificarlas. También es cierto que ignorarlas (hasta el punto de hacer lo contrario) puede ser fatal.

En general, un maquetista que usa una habitación de su casa, o como mucho un anexo de la casa del pueblo. Solo tiene que preocuparse por no poner materiales peligrosos, no pasarse de peso (para que el suelo no se caiga), de que los cables de luz sean lo suficientemente grandes y de no hacer mucho(pero mucho) ruido por las noches.

Si somos un grupo con un local, deberemos tener en cuenta alguna norma más, como la sanitaria, pero en general, con un conocimiento básico y de sentido común sera suficiente. 

Estas normas alimentaran nuestra normativa de construcción y de explotación.

\subsection{Normativas Para Fabricantes}
En el mundo del modelismo ferroviario hay dos asociaciones de fabricantes, la NMRA de Estados Unidos y MOROP de Europa y que a su vez se han coordinado para que sean compatibles y referenciándose entre si.

Estas normas básicamente se establecen para poder correr material de cualquier fabricante sobre maquetas hechas con piezas de diferentes fabricantes. De tal forma podemos usar maquinas de Piko sobre vías Rocco y mezclar coches de varios fabricantes.

Estas normas se dirigen a lo siguiente:
\begin{itemize}
	\item Enganches
	\item Protocolos DCC y LCC
	\item Características eléctricas
	\item Propiedades de las escalas (dimensiones del material)
	\item Distancias de vías (galibo, curvas,...)
\end{itemize}
Para nuestras normativas, realmente solo necesitamos seguir las distancias de vías como una recomendación para ajustarnos a radios para que puedan pasar nuestro material.
\subsection{Normativas Para Módulos}
Otra forma de hacer maquetas es por modulos normalizados, esto son por partes y luego unir cada modulo para formar una maqueta más grande usando piezas de varias personas, pudiendo organizarlas de formas diferentes cada vez.

Los modulos normalizados estan pensados para realizar encuentros de maquetistas llegados de varias ciudades y montar una maqueta nueva cada vez.

\begin{figure}[h]
	\centering\includegraphics[scale=0.10]{chapters/0X_Normativas_01_Intro/IMG_0017.JPG}
	\caption{Modulo FreeMo TT}
	\label{fig:modulott}
\end{figure}


Es necesario tener definido una serie de cosas para que sea posible conectarlos entre si en cada encuentro y a su vez no haya modulos depentientes entre si. Estos puntos se recogen en normativas y los modulos se conocen por los nombres de dichas normativas, Modulos Maquetren, Free-Mo, T-Track,... . 

La organizacion de cada encuentro decide que normativa usar y si hay alguna varicion sobre las normas oficiales. Dicha organizacion tambien suele ser la responsable de tener modulos especiales, que  se salen de la normativa pero son necesarios, como las curvas, bucles o similares.

Como minimo las normativas modular debe definir:
\begin{itemize}
	\item \textbf{Perfil de conexion}: Esto es el perfil que debe mostar un modulo para que al menos las vias coincidan al juntar. Y asi los trenes pasar de modulo a modulo.
	\item \textbf{Conexion Mecanica}: O la forma de unir y anclar dos modulos entre si. De esta manera no se podra desplazar un modulo sin mover el otro.
	\item \textbf{Conexion Electrica}: Es decir los conectores para pasar la corriente a las vias.
\end{itemize}

\begin{figure}[h]
	\centering\includegraphics[scale=0.5]{chapters/0X_Normativas_01_Intro/PERFILMQ40.jpg}
	\caption{Perfil Maquetren MQ-40}
	\label{fig:perfilmq40}
\end{figure}

Pero normalmente suelen definir tambien:

\begin{itemize}
	\item \textbf{Perfil escenico}: Ya no solo el perfil sirve para que las vias coincidan, sino tambien el paisaje, de tal forma que se vea una continuidad escenica\footnote{Al menos sin saltos bruscos, pues cada modulo tendra colores diferentes}.
	\item \textbf{Altura de los modulos}: Desde el suelo.
	\item \textbf{Dimensiones}: Esto es cuanto debe medir un modulo, una dimension ya la tenemos fijada por el perfil, pero la otra puede ser más o menos libre. Fijarla a un valor permite, al encuentro, facilitar la organizacion de la maqueta y cambiar modulos de sitio a voluntad, puesto que todos miden lo mismo. Dejarla libre, dota de mayor creatividad al creador del modulo, pero la maqueta montada requerira de un esfuerzo mayor de montaje.
	\item \textbf{Forma de construir}: Hay normativas que indican como exactamente hay que crear el modulo. Suelen ser más una recomendacion que una obligacion, pero en caso de concurso puede ser razon de descalificacion. 
	\item \textbf{Conexion Electrica}: Ademas de la conexion de la via, la conexion de energia a los accesorios. Tambien en esta punto pude ser el tipo señal (DCC, Analogica,...) a las vias
	\item \textbf{Otros}: Una normativa puede ademas definir otras cosas que considere importante como puede ser el frontal, inclusion de logotipos, cartel identificador,... .

\end{itemize}

Por ultimo algunos encuentros permiten libertad en los modulos mientras se provea de algun lado normalizado para que se puede conectar a otros modulos.
Por ejemplo permiten tener un conjunto largo de modulos propios, pudiendo conectarse entre si como el maquetista quiera siempre y cuando haya al menos un lado normalizado. Como por ejemplo una estacion larga.
  
\subsection{Normativas Especificas o locales}
Por ultimo, a manera muy local, se pueden poner otras normativas que haya que cumplir en la maqueta. En esta categoria podrian entrar tanto las que se pongan para una maqueta concreta o una asocicion ponga en su maqueta de tal forma que los socios puedan hacer partes por su cuenta y luego juntarlas en el local social.

En este ultimo caso difiere de los modulos en que el objetivo es poder partir una maqueta concreta en segmentos "fijos" y hacerlo diferentes personas. Una vez montada va a ser permanente y un segmento conectara siempre con los mismos compañeros, no tienen que ser intercambiables.

\section{Motivacion de las Reglas}
Como podemos suponer las reglas se han ido creando y modificando por una u otra razon. A veces estas razones se olvidan o desaparece la necesidad, pero la regla sigue. Lo que nos lleva al refranero popular y sus maravillosas contradicciones:

\epigraph{Las reglas estan por una razon.}{Refranero popular español}

\epigraph{Las reglas estan para romperlas.}{Refranero popular español}

La primera cita nos dice que sigamos las reglas por que tienen una razon, y la seguna nos dice que nos las saltemos, una contradiccion en toda regla. El significado completo es que todas las reglas estan por una razon, si no sabes cual es siguela por si acaso, pero si la sabes y no se aplica su razon saltatela.

Es decir debemos conocer siempre las normas que se nos aplican y su motivacion, y seguirlas siempre al pie de la letra a menos que no se apliquen a nuestro caso.

En general la motivacion para cada tipo de normativa es:

\begin{itemize}
	\item \textbf{Legislativas}: La mayor parte de estas reglas estan relacionadas con la seguridad. Segun lo que queramos hacer tendremos unas u otras.  

Dentro de la creacion de maquetas lo más probable es que nos afecten legislacion para instalaciones electricas domesticas (a menos que sea muy muy grande) y referidas al peso/montaje. Seguramente podremos consultar a un electricista o un carpintero \footnote{Un amigo te cobrara en cafes, pero siempre se puede contratar a un profesional para hacer las adaptaciones pertinentes}. En caso de querer hacer algo más complicado recomendamos contratar una asesoria legal o un gabinete tecnico.  
	\item \textbf{Para fabricantes}: De estas normas nos fijaremos sobre todo en las dimensiones minimas y recomendadas que debemos seguir, como por ejemplo el radio minimo de las curvas. Si las hemos cumplido y luego no nos va bien algun tren, podemos asegurar que el problema viene de fabrica y no por nuestra maqueta.
	\item \textbf{Para Módulos}: Si queremos hacer un modulo para encuentros debemos seguirlas al pie de la letra, pero en caso de duda consultar con la organizacion. Incluso si no vamos a hacer un modulo como tal, nos conviene conocerlas puesto que son una fuente de ideas provadas para conectar varias secciones. 
	\item \textbf{Especificas o locales}: En este punto es donde debemos hacer más esfuerzo puesto que son las que nos ahorran muchos problemas en el futuro. Nos tocara hacer nuestras propias reglas y normas, sera un esfuerzo importante, pero cuando tengamos que hacer cualquier cosa, podremos ir más rapido y perder tiempo intentando averiguar como van las cosas.
\end{itemize}

\section{Resultados y Discusion} 
No hemos querido entrar en detalles de normativas especificas, para poder abarcar muchos más lectores. Puesto que en cada lugar existiran unas normas u otras. Ya bien sea por diferente legalidad o por prefencias de la zona (T-Track vs Free-Mo).
En este apartado presentamos las recomendaciones de los autores, no tanto su clasificacion o su existencia.

\subsection{Legislativas}
En el termino Legistalivo podemos ver las diferentes normas que existen para el cableado electrico de una casa (o de uso industrial), pero estas varian de pais a pais. Estas variaciones podemos pensar que nos afectan a lo que se ve tipo de enchufe o voltaje\footnote{Si viajamos al extranjero tenemos que llevar adaptatores y asegurarnos que nuestros dispositvos soportan 110V y 220V}
pero en la practica hay muchas normativas que cumplir, tamaño del conductor, materiales validos, distanciuas entre los elementos,... Y por suerte o por desgracia varian de pais a pais o incluso de ciudad a ciudad\footnote{Realmente de provicias, condados o como sea la organizacion territorial}.

En este aspecto de normativas legales, proponemos desde este capitulo que pensemos en tres posibles situaciones y actuemos en consecuencia.
\begin{itemize}
	\item \textbf{Maqueta Grande o Negocio}: Si vamos a montar una maqueta tamaño club, para lo cual hemos adquirdo un local o hemos decidido montar un negocio entorno a la maqueta/s. Debemos aseguranos con profesionales de que el local cumple las normativas correspondientes.

Como toda adquisicion de locales requiere una adecuacion al uso que se le va dar a dicho local, recomendamos encarecidamente aprovechar este momento para contratar a los profesionales que correspondand

Al menos se tendra que revisar que el local:
	\begin{itemize}
		\item Puede soportar el peso de la maqueta junto con otros muebles\footnote{Armarios, sillas, mesas, neveras, televisiones, ...} y de todos los visitantes
		\item Cumple con la normativa electrica para la maqueta, iluminacion,...
		\item Existen los elementos sanitarios y de seguridad correspondientes segun normativa
	\end{itemize}
	\item \textbf{Maqueta Mediana Casera o en habitacion Nueva}: En el caso que no entremos en el caso anterior, pero creamos que nuestra maqueta va a ser más pesada que un armario lleno, vamos a necestir más enchufes/circuitos de los que ya tenga la habitacion o tengamos dudas sobre la construccion de la misma. Recomendamos igualmente contratar a algun profesional. 
	\item \textbf{Maqueta Pequeña Casera en habitacion Existente}: Si vamos a montar una maqueta pequeña, de poco peso
\end{itemize}
En resumen si vamos a usar un negocio, o realizar una adaptacion grande, recomendamos encaridamente contratar a profesionales que se encargen. Pero si vamos a utilizar un lugar conocido y seguro es una sugerencia, segun la confianza que tengamos en la construcción.
Ya que en general, si una vivienda es valida para habitarla, podemos creer que ya se cumplen estas normativa y por lo tanto no deberemos preocuparnos de mas.
\subsection{Para fabricantes}
Estas normas son las que desde el punto de vista de un maquetista son las que podemos decir que menos importancia tienen. Salvo las distancias minimas de margen que esas si son un poco más importantes.

Recordemos que estas normas, indican el tamaño de las ruedas, distancia entre vias, Altura de los enganches, su forma,... . Muchas de estas normas, por no decir todas, son para que el material y/o componentes de diversos fabricantes funcionen en una sola maqueta sin fallos. En esta idea, nos convertimos en "fabricantes" en el sentido de que ese material va a circular por nuestra maqueta, y ahi es donde tenemos que ver cual es el radio minimo de curvas.
Alturas entre plantas, distancias entre carriles, galibos,... .
\subsection{Para Módulos}
Los módulos han definido una serie de normas para poder crear un modúlo y llevarlo a los encuentros. Pero existen varios standares, incompatibles entre si y realmente ninguno mejor que otro. Porque cada uno se enfoca en problemas especificos de cada uno y en las preferencias de cada zona. Si hemos de elegir alguno deberia ser aquel que se ajuste a dos Características:
\begin{itemize}
	\item Nos permita acudir al mayor número encuentros.
	\item Nos acerque al mayor número de personas.
\end{itemize} 
Y como se puede ver, ninguna de ambas es de razones tecnicas, sino sociales. Los modulos son el objetivo y la consecuencia de los encuentros. Para una maqueta personal, o incluso para una grupo reducido, no tiene sentido utilizar una normativa modular estricta, pero si diseñar la maqueta pensando en juntar y separar rapidamente. Para estos casos es mucho mejor un sistema segmentado. Donde cada "Modulo" o segmento puede modificar cosas para que se ajueste mejor al tema de la Maqueta, como puede ser el perfil, dimensiones,\dots Lo que tampoco quita que los segmentos se inspiren en normativas modulares.
\subsection{Especificas o locales}
Queremos volver a recalcar lo imporante de tener una serie de normas para hacer una maqueta, y crearnoslas para nosotros mismos. Una norma puede ser el tipo de cables y sus colores para el bus DCC, esta norma es sencilla. Y, si la respetamos a lo largo de toda la maqueta, cuando tengamos un problema identificaremos rapidamente cuales son los cables del bus DCC. de los que iluminan la maqueta.
\section{Conclusiones}\section{Próximos pasos}

\section{Bibliografía}
\printbibliography[heading=subbibliography]

\chapter{Normativa Minima Legal}
%!TeX encoding = UTF-8
%!TeX spellcheck = es_ES
%!TEX root=../../main.tex

\epigraph{Quien desconoce el motivo de las normas está condenado a respetarlas.}{Valérie Tasso}

\begin{abstract}
Ya hemos hablado de las normas y su porque general. En este capitulo veremos una serie de reglas que siempre se deberan cumplir si o si. Veremos que son muy logicas y basadas en cuestiones tecnicas sin ser muy arbirarias.
\end{abstract}

\section{Introducción}
Hemos visto que a nivel legal siempre hemos recalcado al menos tres tipos de normas:
\begin{itemize}
	\item \textbf{Estructurales}: Son las normas que dictan cuanto peso soporta una habitacion y lo que debe pesar nuestra maqueta.
	\item \textbf{Electricas}: Como debe de ser la instalacion electrica para ser segura.
	\item \textbf{Sanitarias}: En caso de hacer un negocio los requisitos sanitarios y de seguridad que deben segir.
\end{itemize}
De las cuales, no siempre vamos a crear un negocio, por lo que solamente nos centraremos en las dos primeras.

Asi mismo estas dos normativas legales se han ido creando en el tiempo debido a accidentes y su objetivo es minimizar el riesgo a las personas.
Su no cumpliemento, ademas de una multa, puede ocasionar graves daños personales (incendios, derrumbes,\dots).

\section{Estado del arte}
Cada pais tiene un reglamento de como se debe construir una casa y que pesos puede soportar un edificio. Y lo mas sorprendente, donde puede o no puede instalarse un armario.

Igualemnte para la instlacion electrica, cada pais tiene su reglamento que indica que cables se pueden usar y donde colocarlos.
\subsection{Estructura y Pesos}
Los libros y la ropa pesan mucho más de lo que nos pensamos, y si no que les pregunten a las lineas aereas. Si nos fijamos cuidadosamente, las paredes para armarios se situan entre columnas, lo que hara que el peso recaiga sobre una viga y no sobre el centro de una habitacion.

Asi mismo, una reforma del baño hay que tener cuidado por que la zona de la bañera esta reforzada para poder soportar el peso de varios adultos y el agua que pueda entrar, si no fuera asi, las escenas donde cae una bañera se verian más que en el cine.

Cuando se compra una casa, al nuevo dueño se le deberia incluir un manual donde se indiquen los kilos soportados por metro cuadrado y donde colocar los armarios pesados.

No poner el peso en el lugar correcto, o ubicar más peso donde no se deberia, hace que corramos el riesgo de que el suelo se caiga. Y nuestro suelo es el techo del vecino de abajo.
\subsection{Instalacion Electrica}
El reglamento de las instalaciones electricas varia de pais a pais, pero todos van a tener unos apartados similares:
\begin{itemize}
	\item \textbf{Tipos de cables}: Segun el Amperaje que debe soportar el circuito, que cable usar (seccion y materiales aislantes)
	\item \textbf{Voltajes}: Voltajes del sistema, normalmente si es 110V o 220V y que amperajes maximos deben ir por cada circuito, 16A o 20A, pòr lo general\footnote{Se recomienda revisar en su pais, lo que dice concretamente la normativa}.
	\item \textbf{Elementos fijos}: Como son los enchufes, los interruptores, tanto desde el punto de vista mecanico (forma y materiales), como de su numero, y ubiocacion en una habitacion. 
	\item \textbf{Canalizaciones}: Como deben ser los tubos que llevan los conductures, puesto que entre enchufe y enchufe, el cable debe ir en un tubo, no puede ir al aire. La norma dice tambien cuantos conectores pueden ir por un tubo concreto.
	\item \textbf{Distancias de seguridad}: Relacioando con la ubicacion de los elementos fijos  pero referido a distancias minimas de seguridad para no causar problemas con otros elementos
\end{itemize}
Estos temas son los mas tipicos que nos podremos enconctrar, sobre todo relacionados con un piso normal. Pero la norma debe abacar todo tipo de viviendas.

Desde un punto de vista tecnico no seguir estas reglas pueden producir tres tipos de problemas:
\begin{itemize}
	\item \textbf{Chispas}:Cuando dos conductores con cargas(voltajes, resumidamente) diferentes se acercan mucho entre si, salta una chispa entre ellos. Esta chispa puede producir un incendio si toca algun materiar inflamable.
	\item \textbf{Calor}: La corriente que pasa por un conductor lo calienta. Segun la seccion se calentara más o menos (a misma corriente). La normativa indica que conductor se debe usar segun la corriente máxima que pasa por el. Tambien se puede leer al reves, segun la seccion como mucho puede pasar tanta corriente. Esta combinacion es la que hace que el conductor se caliente como mucho a 70 grados, y que se considera segura para los aislantes usados. Si se utilzar para más corriente se corre el riesgo de que se caliente más, ya de por si este calor tiene la posibilidad de iniciar un fuego. Pero ademas este calor extra hace correr el riesgo de que se funda el aislante y exponga un conductor a otros de diferente voltaje, pudiendo provocar una chispa.
	\item \textbf{Descargas no deseadas}: Toda instalacion tiene dos cables, Neutro (sin voltaje) y Vivo (con voltaje Alterno). Pero que no tenga voltaje, no significa que no lleve corriente. En la practica se puede decir\footnote{Simil para entender, la realidad es un poco más complicada} que la corriente nos viene por el cable ``vivo'', la usamos en nuestro electrodomestico y vuelve por ``neutro'' a la central electrica. A veces, por una mala instalacion o un fallo en el electrodomestico, esa corriente vuelve pasando por el usuario del mismo pudiendo causar un paro cardiaco.
\end{itemize}
\subsection{Instalacion electrica Tipica}
La instlacion tipica de una casa empieza con la conexion del contandor al interruptor general. Luego de ahi se va a varios intrreptores de circuito y luego cada circuito va a una o varias cajas de empalmes. Suele a ver una caja por habitacion y de ahi se unen los conductores de los enchues de la habitacion. 

Dibujo mostrando esto

Las normativas indican cuantos interruptores pueden ir a un circuito, si hay circuitos para iluminacio, o para electrodomesticos especificos (como aires acondicinados, hornos, neveras,\dots)

Para complicar un poco más las cosas, las empresas nos cobran por KiloWatios, pero las limitiaciones son por Amperios. Por suerte hay una relacion sencilla.
\section{Peso Máxmimo de la maqueta}
Podemos considerar que en una construccion moderna, va a ser muy dificil que una maqueta de tren (grande superficie plana con "poco" volumen) supere el maximo de ocupacion de una casa, a menos que usemos la parte inferior para almacenaje de objetos pesados.

Todas las construcciones deben soportar una sobrecarga mucho mayor de 150 Kg por Metro cuadrado\footnote{En España es de 300 segun normativa de 2020, 200 en construcciones anteriors a 1965}. Eso significa poner más de 150 KG en cada cuadrado de un metro de largo. En una habitacion de 10m2 serian poner 1500 KG o el equivalente a 20 personas promedio. 

En caso de duda se puede medir los m2 en planta que ocupa la maqueta y multiplcar por 100KG si pesa menos, estaremos tranquilos, si pesa más limitaremos el contenido interior.

\section{Consumo de la maqueta}
\subsection{¿Cuantos Boosters?}
Para saber cuanto consume nuestra maqueta tenemos que saber cuantos booster necesitamos, estos se calculan en funcion de las maquinas (decoders) que vamos hacer correr a la vez en nuestra maqueta. Mirar en documentaciones cuanto consumen y sumar por este lado. Luego veremos cuanto puede suministrar nuestra central y los boosters que tengamos y aseguranos que esta suma es superior.

$A_{central} + \sum_{n=1}^{nBoosters}A_{Booster}(n) \geq \sum_{i=1}^{nDecoders}A_{Decoders}(i)$

En este punto, no vamos a entrar en como saber cuanto nos consumen los trenes en funcion de lo que estan haciendo y nos bastara aplicar la regla simple de esperar 0.5 amperios por maquina. 

$A_{central} + \sum_{n=1}^{nBoosters}A_{Booster}(n) \gg 0.5\times nDecoders $

Como los Decoders nos dicen cuantos ameperios necesitan y los Booster cuantos Amperios pueden suministrar, hasta este punto es facil, solo es cuestion de usar el Booster que nos de los suficientes y estos facilmente los encontramos de 2, 3, 10, 15, 20,\dots.
\subsection{¿Cuanto nos sonsumen nuestros boosters?}
El siguente problema es saber cuanta energia vamos a pedir a nuestra instalacion. 
Si nos fijamos en los enchufes tipicos\footnote{O en las alargaderas}, veremos que solo soportan 16A como mucho\footnote{Segun el fabricante, no me atreveria a meter más de 12A en algunas bases multiples\dots}. Eso nos puede llevar a pensar que no podemos tener Boosters de más de 15A, pero alguno hay de 20A. ¿Como puede ser esto posible?\dots. Para simplificar el texto, vamos a suponer que tenemos un Booster de 10 Amperios y 15 Voltios de salida.

La razon es que, desde el punto de vista de consumo, un Amperio a 15V no es lo mismo que a 220V. Lo importante es energia utilizada, que se mide en Julios y se calcula como la integral de la potencia instantana durante un periodo de tiempo t. Si el consumo es constante se puede simplificar en la multiplicacion de la potencia por el tiempo.

$ E= \int_{i}^{t}{p(i)dt} \approx P\cdot t$ 

La potencia se mide en Watios, y si la multiplicamos por el tiempo de medida, tenenos la energia usada en ese periodo\footnote{En el S.I. Watios Segundo, o Julios}. Si nuestro periodo lo medimos en horas, podemos multiplicar directamente los Watios por esas horas y tenemos la energia en Watios Hora, que es lo que nos cobran\footnote{Las compañias cobran por KiloWatioHora, pero solo es dividir por 1000}. Lo bueno de usar esta unidad es que nos permite entender una cosa, si nuestra consumo es 10 Watios-Hora nos da igual que hayan sido 10 Watios en una hora que 1 W en 10 horas, el resultado es el mismo.

Volviendo a los boosters, se puede decir que son un sistema transformador de energia, de 220V en alterna a 15V en continua por lo que la energia de salida (10 Amperios a 15 Voltios) debe ser la misma que de entrada (X Amperios a 220 Voltios). En la practica van a haber perdidas\footnote{En forma de calor} por lo que la energia de entrada sera superior. Como regla general, podemos decir que perdemos un 20%

$E_{entrada} = E_{salida} + E_{perdida}$

$E_{entrada} > E_{salida}$

$E_{entrada} \approx 1.20 \cdot E_{salida}$

Pero esta energia de entrada debe ser en el mismo periodo y ademas para saber cuanto puede consumir nuestra maqueta, hay que ponerse en el peor caso. Siendo este que el booster este dando al maximo durante todo el tiempo\footnote{Si hacemos nuestros calculos seguros para ese caso, sabremos que sera seguro para la realidad, ya que siempre sera menor nuestro consumo}, lo que es consumo constante asi que podemos sustuir por la multiplicacion y simplificar el termino t en la equivalencia

$E_{entrada} \approx 1.20 \cdot E_{salida}$

$P{entrada}\cdot t \approx 1.20 \cdot P_{salida} \cdot t$

$P_{entrada} \approx 1.20 \cdot P_{salida}$

Lo que quiere decir que nos podemos fijar solo en los Watios y olvidarnos del tiempo.

Pero como calculamos los Watios de nuestros boosters, pues la potencia en un momento dado es la multiplicacion del voltaje por la corriente (Voltios por Amperios). En corriente Continua, siempre tenemos los mismos voltios y la misma corriente, por lo que se simplifican los calculos para calcular la potencia real:

$p(i) = v(i) \cdot a(i)$

$P = V \cdot A$

En corriente alterna, donde el voltaje y la corriente forman una señal periodica que se repite un tiempo, el calculo es mas complicado y se hace la media sobre el tiempo de cada periodo. Por suerte para las señales tipicas se han calculado simplificaciones con dos factores, uno de forma y otro de potencia.

$ P_{ac} =\frac{1}{T} \int_{i=0}^{i=T}{v(i)\cdot a(i)\cdot dt}$

$ P_{ac} = V \cdot A \cdot F_{forma} \cdot F_{potencia} $

Ambos factores siempre son menores o iguales a 1 y dependen del tipo de señal (Sinosuidal, triangular, cuadrada,...) y del desfase entre corriente y volataje. Por suerte para el factor de forma se tiene ya calculado y el Factor de potencia en los sistemas domesticos debe ser cercano a 1. Para las ondas sinusoidales es $\frac{1}{\sqrt{2}} \approx 0.707$, conociendo este dato se puede asignar un factor de potencia equivalente de tal forma que de $\frac{1}{2}$

$F_{forma} \cdot F_{potencia} \approx  \frac{1}{2} $

Dicho esto ya podemos ver tres formas de como saber cuantos amperios circularan por nuestro circuito electrico

\subsubsection{La Matematica}
Sabemos que la energia de entrada debe ser la de salida, pero podemos simplificar a la potencia de entrada debe ser la misma ya que es siempre en la mismo tiempo. 

$ E_{entrada} = E_{salida} $

$ P_{entrada} = P_{salida} + P_{perdida} $ (teniendo en cuenta perdidas)

$ P_{entrada} = 1.20 \cdot P_{salida} $

$ V_{ac} \cdot A_{ac} \cdot F_{forma} \cdot F_{potencia} = 1.20 \cdot V_{salida}\cdot A_{salida}$
$ A_{ac} = \frac{1.20 \cdot V_{salida}\cdot A_{salida}}{V_{ac} \cdot F_{forma} \cdot F_{potencia}} $ (Reorganizando)

$ A_{ac} = \frac{1.20}{F_{forma} \cdot F_{potencia}} \cdot\frac{V_{salida} }{V_{ac}} \cdot A_{salida} $

Hemos añadido las perdidas, puesto que la informacion que tenemos de la potencia de salida, es la potencia que se entrega a las vias, no la que  el mismo booster necesite para sus operaciones.

Ademas hemos reorganizado los terminos para encontrarnos con dos factores sobre el amperaje de salida:

$\frac{1.20}{F_{forma} \cdot F_{potencia}}$ y $\frac{V_{salida} }{V_{ac}} $

El primero es un factor que depende de las perdidas  y el segundo segun la relacion entre los voltajes.

Si sustimos la informacion que tenemos nos queda:

$A_{ac} =  \frac{1.20}{0.5}\cdot\frac{15}{220}{10}$

$A_{ac} \approx  2,40\cdot\frac{7}{100}\cdot {10}$

$A_{ac} \approx \frac{17}{100}\cdot{10} \approx 0,17\cdot 10 \approx 1.7$

Si se revisan los calculos se podra observar que esta cifra obtenida es ligeramente mayor al resutaldo ``real''. Ademas el primer factor depende de las perdidas y de que el transformador tenga un factor de potencia tal que duplique ese 1.20.
Estos valores, han sido escogidos suponiendo una mal diseño del transformador que de la corriente al booster. En la practica para poder venderse como domestico, estos valores deberan ser más peqeños y ese factor estara más cerca del 1.50 que el 2.40 obtenido.

Estos calculos habra que realizarlos cada voltaje de salida y de entrada (en caso de no ser 220V, como en europa). Por ejemplo para 18V, nuestros 10A se convierten en 2A en 220V y 4A en 110V.

Nota: Los voltajes recomenados para H0 y N son 16 y 12V, respectivamente. por lo que 18V estaria por encima de las recomenaciones\footnote{Otras escalas pueden usarlo, recomendamos rehacer estos calculos a necesidad}.

En este metodo, podemos ajustarnos más al tope que puede suministrarnos\footnote{Repitiendolos para los datos concretos y siempre dejando al menos 0.5A como margen}, con los datos de los calculos podriamos poner 9 Boosters y aun nos sobrarian 0,7 Amperios\footnote{Aunque esto no deja espacio para otras cosas, como iluminacion de la habitacion, ordenadores, accesorios,...}. Y eso serian 90 amperios a 15V con lo que podriamos tener 180 locomotoras al maximo a la vez, con lo que no estariamos hablando de una maqueta pequeña.

\subsubsection{La rapida}

El objetivo de estos calculos es saber cuantos booster podemos poner en nuestro circuito de casa. Lo que quiere decir que si nos equivocamos en el consumo por arriba, suponiendo que nos consume más de lo que luego consumen realmente, y aun asi no alcanzamos el limite que nos puede suministrar, no estaremos poniendo en peligro la instalacion.

Por lo que podemos hacer un calculo rapido suponiendo que cada Amperio de salida nos consume 0.2 Amperios de entrada, o que es lo mismo, dividir por 5. Esto para el caso de 220V, para 110 bastaria 0,4 Amperios o dividir por 2.5. 

Este calculo rapido vemos que nos sirve para los voltajes recomendados de H0 y de N, incluso llendonos a 18V.

En todo caso si utilizamos este metodo se deberia dejar al menos 1A margen

Con este sistema, por cada Booster de 10A necesitamos 2Amperios, por lo que en nuestra liena de 16A podriamos poner 7, dejando de margen 2A\footnote{8 nos deja sin margen}.

\subsubsection{La tecnica}
Si nos fijamos en las fichas tecnicas de los transformadores que se usan en los boosters, vermos que hay una tabla que dice el voltaje y los amperios de salida y los de entrada.

(Foto de un transformador)

Lo bueno de estos datos es que ya tienen en cuenta las posibles perdidas y los factores de potencia, por lo que los datos son más reales.

En la foto incluida se puede ver que nos produce XXA y necesitamos YYY Amperios en corriente alterna, Si hacemos los calculos matematicos vemos que nos produce un error de ZZZ, y ahora podemos tener

\subsubsection{Pensar en Watios}

Otra forma de calcular lo que nos va a consumir es pensar en Watios. Si sabemos el voltaje y su corriente maxima, podemos tener en cuenta las formulas, perdidas y demas podemos decir:

$P_{ac}=V\cdot A \cdot F_{potencia}\cdot P_{forma}=220 \cdot 16 \cdot \frac{1}{2} = 1760 \approx 1800W$

$P_{salida}= V\cdot A +P_{perdidas}\approx 1.20\cdot V\cdot A =15 \cdot 10 \cdot 1.20=180W$

A partir de aqui es cuestion de pasar a W todos los dispositivos y sumar hasta que nos llege a 1800W.

Nota: ¿Como es posible que haya planchas de más 2.200 W? 

Esto es muy sencillo, hemos realizado los calculos suponiendo que el factor de forma y de potencia nos de como combinados 0.5 y ya hemos dicho que estamos considerando una situacion mala que nos limita los Watios que nos suministra la instalacion. Una plancha es una carga resistiva cuyo factor de potencia es 1\footnote{O tan cercano que se simplifca a 1} en ese caso, $P_{forma}=\frac{1}{\sqrt{2}}\approx=0.707$. A 220V nos suministra unos 2500W, pero ademas, para asegurar un minimo las compañias envian la coriente a 240V, en esta situacion el maximo es 2800W.

Esto suponiendo que la plancha es lo unico conectado a ese circuito y que el cable por el interior es de 16A y no de otra capacidad.


\subsection{¿Que nos soporta nuestra Red Electica?}
A nuestra casa llega un cable de la suminstradora que puede virtualmente soportar infinitos amperios. Esto quiere decir que esta sobredimensionado y que facilmente puede soportar el doble del maximo de potencia contratable por un individual.

Este cable entra a una ``caja de fusibles'' que no son más que interruptores ``automaticos'' que cortan la corriente si detectan mucho consumo.
Se suelen llamar tambien como diferenciales, magnetotermicos,generales\dots pero estos nombres solo son como controlan cuanta corriente pasa o su funcion.

(Poner diagrama)

Por norma general estos interruptores se miden en amperios y es lo maximo que permiten pasar a la siguiente fase.

No vamos a entrar en la normativa domestica y vamos a simplificar nuestra caja de fusibles a dos tipos de interuptores:
\begin{itemize}
\item \textbf{General o IGP}: Este interruptor lo pone la suministradora y es el primero de la linea, limita la corriente a lo contratable. Puede que permita más corriente de la contratado pero no es lo normal\footnote{Un cambio de potencia contratada se supone que implica cambiar este elemento, pero se pueden ahorrar costes si el que hay superior}. Este interruptor no lo podemos cambiar, esta sellado. 

Este interruptor conecta el cable de entrada a los Automaticos residenciales y limita la corriente máxima que podemos consumir en toda la casa. Estos interruptores son de corrientes ``altas'' (40, 50,\dots)
\item \textbf{Automaticos residenciales}: Estos interruptores son de corrientes ``bajas'' (6,10,16,\dots) y lo que sale de ellos es un circuito, del que cuelgan varios enchufes, luces,\dots y esos amperios estaran compartidos por todos los enchufes conectados a ese interruptor.

Estos interruptores pueden ser cambiados por el dueño de la casa, asi que seria posible poner un interruptor de 20A para conectar a dos enchufes de 16 y aumentar un poco la capacidad, siempre y cuando la normativa lo permita.
\end{itemize}

En una casa moderna, existiran varios circuitos, iluminacion, baños, habitaciones, salon,\dots Pero en cualquier caso podemos asegurar que un circutito va tener varios enchufes. Y si queremos usar más de los 16A, usando varios dispositivos, debemos asegurarnos de que los conectamos a diferentes circuitos.

Asi mismo, en las casas modernas, no va un cable desde cada enchufe hasta el interruptor. En cada habitacion hay una caja de conexiones, donde se une el cable que viene del interruptor a varios cables, uno por cada enchufe. Y donde hay una union, la corriente se reparte, si el interruptor es de 16 amperios, esos 16 amperios se repartiran por los enchufes, asi que no podremos tener dos aparatos de 10 Amperios conectados en ese circuito.

Otra cosa que hay que tener en cuenta es la capacidad de los cables. Es posible tener cada vez cables más pequeños, segun la necesidad y lo que permita la normativa. Por ejemplo, podemos poner un interruptor de 20A que vaya a una caja de conexion y de ahi a dos enchufes de 16A, pero en este caso debemos aseguranos de que el cable desde el interruptor soporte esos 20A y de que luego no conectemos nada que supere 16A en los enchufes.

\subsection{Que podemos hacer}
En una maqueta normal seguramente nos baste con la instalacion normal de una casa. Y sino con una modificacion muy simple de añadir un circuito solo para una habitacion, no seria muy dificil añadir un interruptor y llevar cable nuevo hasta la caja de conexion de la habitacion. Reutilizando las canalizaciones y la caja existente a este nuevo circuito en vez de uno compartido con otras. Si somos habilidosos lo podemos hacer nostros mismos, pero un electricista lo puede hacer sin obra.

(Ejemplo)

Otra solucion puede ser usar, el circuito de iluminacion, para iluminar la maqueta. En vez de usar lamparas conectadas a enchufes, tomar la corriente de una lampara del techo. Pero ojo, este circuito es de menor potencia que el de los enchufes. Nos serviria para ganar unos pocos Watios. Esto aun podemos hacerlo sin contratar a nadie.

La siguiente modificacion, seria pedir a un electricista que nos ponga un enchufe de 25 Amperios, o de cocina. Esta solucion implica que un profesional añada un interruptor, canalizacion hasta la habitacion y obre para poner un nuevo enchufe. Pero sera más barato que la otra opcion.

Foto de un enchufe 25A

De este enchufe de 25A nos podemos hacer nuestra propia caja de fusibles para la maqueta, y asi organizar como queramos los nuevos enchufes. En realidad nada nos impide hacernos nuestra caja de fusibles para 16A y poner interruptores, para iluminacion, DCC, accesorios,\dots

(Dibujo enchufe y caja fusible para la maqueta)

La ultima opcion es realizar una obra con profesionales indicando bien nuestras necesidades y ellos se encargen de todo.
\section{Resultados} 
\section{Discusión}
\section{Conclusiones}
Hemos visto un par de normas que debemos tener en cuenta cuando hagamos nuestra maqueta. En general y para una maqueta domestica no debemos preocuparmnos mucho, pero si esta crece, o nos queremos construir una nueva habitacion estas normas las tendremos que tener en cuenta.



\section{Próximos pasos}

\section{Bibliografía}
\printbibliography[heading=subbibliography]
	

\part{Original}

%%% EXAMPLES
\input{Examples}
\chapter{Base}
% !TeX encoding = UTF-8
% !TeX spellcheck = es_ES
% !TeX root = ../../main.tex


\epigraph{Los hombres no crecen, solo cambian el precio y tamaño de sus juguetes}{Cita anonima en Internet}

\begin{abstract}
Hay varias formas de jugar con una maqueta de tren, en este capitulo revisaremos algunas de las más comunes
\end{abstract}

\section{Introducción}
Incluir aquí una introducción al capitulo, estableciendo un contexto para centrar al lector
\cite{ackerberg2006} Es una prueba solo para jugar

\section{Estado del arte}
Explicar como esta actualmente el hobby o las diferentes publicaciones respecto al tema
\section{Experimento o Texto principal}
Describir de la manera más aseptica posible lo que se quiere avanzar
\section{Resultados o Datos de interés}(Opcional) 
Si es un experimento incluir los datos o resultados obtenidos, sin valorar ni judgar. Es buen lugar para incluir otros detalles encontrados durante la escritura, búsqueda de información,....
\section{Discusión}
Este el punto para valorar los resultados y dar opiniones.
\section{Conclusiones}
Resumir y agrupar los resultados obtenidos
\section{Próximos pasos}
Escribir aquí un breve texto de lo que se hablara en otros capítulos (y que tenga referencia con este), o cosas que se dejan para realizar en un futuro fuera de este PDF.
\section{Bibliografía y Referencias}
\printbibliography[heading=subbibliography]
%----------------------------------------------------------------------------------------
%	BIBLIOGRAPHY
%----------------------------------------------------------------------------------------

\chapter*{Bibliography}
\nocite{*}
\addcontentsline{toc}{chapter}{\textcolor{ocre}{Bibliography}} % Add a Bibliography heading to the table of contents

%------------------------------------------------

\section*{Articles}
\addcontentsline{toc}{section}{Articles}
\printbibliography[segment=*,heading=bibempty,type=article]

%------------------------------------------------

\section*{Books}
\addcontentsline{toc}{section}{Books}
\printbibliography[heading=bibempty,type=book]

\section*{Paginas Web}
\addcontentsline{toc}{section}{Paginas Web}
\printbibliography[heading=bibempty,type=online]

%----------------------------------------------------------------------------------------
%	INDEX
%----------------------------------------------------------------------------------------

\cleardoublepage % Make sure the index starts on an odd (right side) page
\phantomsection
\setlength{\columnsep}{0.75cm} % Space between the 2 columns of the index
\addcontentsline{toc}{chapter}{\textcolor{ocre}{Index}} % Add an Index heading to the table of contents
\printindex % Output the index

%----------------------------------------------------------------------------------------

\end{document}
