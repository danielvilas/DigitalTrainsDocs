% !TeX encoding = UTF-8
% !TeX spellcheck = es_ES
% !TeX root = DB-Conectors.tex
Daniel Bahn es una empresa ficticia para una maqueta de trenes personal. Esta empresa simula ser la gestora de la red ferroviaria dispuesta en dicha maqueta.
Es por tanto el hilo conductor que nos permite entender dicha maqueta.

Para esta maqueta se ha decidido definir una serie de reglas u normativas para facilitar la evolucion de la maqueta propiamente dicha y una metodologia
reutilizable para futuras maquetas.

Este documento es la normativa que deben seguirlas diferentes conexiones electricas entre los elementos principales

La maqueta ha sido diseñada mediante modulos y segmentos que pueden ser conectados y desconectados a voluntad.
Por lo que es importante tener una serie de conectores estandarizados y seguros que faciliten su utilidad

\subsection{Arquitectura}
La maqueta DanielBahn esta dividida en varios segmentos, una version sencilla y reducida
\sidenote{Para entender lo que se quiere explicar aqui} de la misma es:

\begin{figure}[H]
    \centering
    % !TeX encoding = UTF-8
% !TeX spellcheck = es_ES
% !TeX root = ../Esquema.tex
% !TEX root = ../Esquema.tex

\newcommand{\paintBoard}[1][Melon!15]{

    %Base Board
    \draw [fill=#1] (-6,0) rectangle +(3,3)
        (-3,1.5) rectangle +(2,1.5)
        (-1,1.5) rectangle +(2,1.5)
			(1,1.5) rectangle +(2,1.5)
        (6,0) rectangle +(-3,3);
    \draw [fill=Melon!15] (-6,0) rectangle +(3,-3)
        (-3,-1.5) rectangle +(2,-1.5)
        (-1,-1.5) rectangle +(2,-1.5)
        (1,-1.5) rectangle +(2,-1.5)
        (6,0) rectangle +(-3,-3);

}


\newcommand{\paintMain}[2][black]{
    \draw[color=#1,line width=#2] (3,2.1) arc (90:-90:2.1)
    -- (-3,-2.1) arc(-90:-270:2.1) --(3,2.1);
}

\newcommand{\paintStation}[2][black]{
    \draw[color=#1,line width=#2] (5.1,0) --(5.1,-0.3)
        arc(0:-90:2.1) -- (-3,-2.4)
        arc (-90:-180:2.1) -- (-5.1,0);
}

\newcommand{\paintTerminus}[2][black]{
    \draw[color=#1,line width=#2] (-1,-1.8) -- (3,-1.8) arc(-90:90:1.8)
    -- (2.5,1.8) -- (1.2,2.1);

\draw[color=#1,line width=#2] (1.2,-1.8) -- (2.8,-2.1);

}


\newcommand{\paintYard}[2][black]{
	\draw [color=#1,line width=#2] (-2.8,2.1) -- (-1.5,2.4)
		-- (-0.2,2.7);
	\draw [color=#1,line width=#2](-1.5,2.4)
		-- (5,2.4);
	\draw [color=#1,line width=#2](-4,2.7)
		-- (5,2.7);
}

    \caption{Maqueta Simple}
    \label{fig:MaquetaSimple}
\end{figure}

Esta version de la maqueta tiene como objetivo probar diferentes tecnicas
y es por lo tanto muy sencilla en cuanto a diseño de circuito ferroviario.

Estos segmentos necesitan un bus DCC para manejar los trenes, un Bus de corriente
continua para alimentar accesorios y un bus LCB\sidenote{Layout Control Bus} para 
manejar los trenes y recibir informacion de los accesorios.

\begin{figure}[H]
    \centering
    

\begin{tikzpicture}

    %\draw [very thin, green]  (-6,-3) grid (6,3);
    \begin{scope}[shift={(-4.5,0)}] 
        \draw [fill=Melon!15] (-2,-1) rectangle +(4,2);
        \pic[]() at (0,0.25) {RailStraigh=black};
        \node[]() at (0,-0.5){Segmento 1};
    \end{scope}

    \begin{scope}[shift={(0,0)}] 
        \draw [fill=Melon!15] (-2,-1) rectangle +(4,2);
        \pic[]() at (0,0.25) {RailStraigh=black};
        \node[]() at (0,-0.5){Segmento 2};
    \end{scope}

    \begin{scope}[shift={(4.5,0)}] 
        \draw [fill=Melon!15] (-2,-1) rectangle +(4,2);
        \pic[]() at (0,0.25) {RailStraigh=black};
        \node[]() at (0,-0.5){Segmento 3};
    \end{scope}

\end{tikzpicture}
    \caption{Modulos y Buses}
    \label{fig:ModulosBuses}
\end{figure}
