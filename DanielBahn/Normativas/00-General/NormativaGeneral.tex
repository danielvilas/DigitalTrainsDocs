% !TeX encoding = UTF-8
% !TeX spellcheck = es_ES

\documentclass{DccDiyTools/DccDiyTools}
\usepackage[spanish]{babel}
\usepackage[
type={CC},
modifier={by-sa},
version={4.0},
]{doclicense}


\title{Normativa General}
\author{Daniel Bahn}
\date{Febrero 2022}

\dbHeaderTitle{Normativa General}
\dbType{PR}
\dbDate{22}
\dbCode{000}
\dbStatus{Draft}
\dbVersion{0.1}

\begin{document}
%\pagestyle{fancy}
\maketitle
\newpage
%\thispagestyle{fancy}
\section{Introduccion}
Daniel Bahn es una empresa ficticia para una maqueta de trenes personal. Esta empresa simula ser la gestora de la red ferroviaria dispuesta en dicha maqueta.
Es por tanto el hilo conductor que nos permite entender dicha maqueta.

Para esta maqueta se ha decidido definir una serie de reglas u normativas para facilitar la evolucion de la maqueta propiamente dicha y una metodologia
reutilizable para futuras maquetas.

Este documento es la normativa que deben seguir el resto de documentos, siendo realativo a estos su tipologia, formato, estructura a seguir,...

\section{Motivacion} 
Una maqueta que cuenta una historia es más facil de planear. Una maqueta documentada es más facil de diseñar y Mantener. A pesar de que hacer estas
tareas conlleva más esfuerzo, a la larga del resultado es más placentero. Ademas es un esfuerzo que se puede ir haciendo en ratos muertos, en los que 
no es posible estar fisicamente haciendo cosas en la maqueta.

Recordemos tambien que una forma de jugar con las maquetas es simular una compañia, creando documentos de regimen interno, tales como tipos de señales
restriciones, hojas de rutas y horarios,... para utilizarlos en las sesiones de juego. Para esta maqueta no se pretende llegar tan lejos, pero si
utilizar la idea de simular una compañia como hilo conductor y generando tanta documentación como sea posbile.


\newpage
\section{Objetivos}
\subsection{Objetivos de este documento}
\subsubitem{Otra}

% \begin{tip}{Test}
% Ejemplo de tip
% \end{tip}
\end{document}