% !TeX encoding = UTF-8
% !TeX spellcheck = es_ES

\documentclass{DccDiyTools/DccDiyTools}
\usepackage[spanish]{babel}
\usepackage[
type={CC},
modifier={by-sa},
version={4.0},
]{doclicense}


\title{Normativa General}
\author{Daniel Bahn}
\date{Febrero 3}

\dbHeaderTitle{Normativa General}
\dbType{D}
\dbDate{23}
\dbCode{000}
\dbStatus{Draft}
\dbVersion{0.1}

\begin{document}
%\pagestyle{fancy}
\maketitle
\newpage
%\thispagestyle{fancy}
\section{Introduccion}
Daniel Bahn es una empresa ficticia para una maqueta de trenes personal. Esta empresa simula ser la gestora de la red ferroviaria dispuesta en dicha maqueta.
Es por tanto el hilo conductor que nos permite entender dicha maqueta.

Para esta maqueta se ha decidido definir una serie de reglas u normativas para facilitar la evolucion de la maqueta propiamente dicha y una metodologia
reutilizable para futuras maquetas.

Este documento es el registro del esquema de la maqueta

\begin{figure}[H]
    \centering
    % !TeX encoding = UTF-8
% !TeX spellcheck = es_ES
% !TeX root = ../Esquema.tex
% !TEX root = ../Esquema.tex

\newcommand{\paintBoard}[1][Melon!15]{

    %Base Board
    \draw [fill=#1] (-5,0) rectangle +(3,3)
        
        (-2,1.5) rectangle +(2,1.5)
			(0,1.5) rectangle +(2,1.5)
        (5,0) rectangle +(-3,3);
    \draw [fill=Melon!15] (-5,0) rectangle +(3,-3)
        (-2,-1.5) rectangle +(2,-1.5)
        (0,-1.5) rectangle +(2,-1.5)
        (5,0) rectangle +(-3,-3);

}


\newcommand{\paintMain}[2][black]{
    \draw[color=#1,line width=#2] (2,2.1) arc (90:-90:2.1)
    -- (-2,-2.1) arc(-90:-270:2.1) --(2,2.1);
}

\newcommand{\paintStation}[2][black]{
    \draw[color=#1,line width=#2] (4.1,0) --(4.1,-0.3)
        arc(0:-90:2.1) -- (-2,-2.4)
        arc (-90:-180:2.1) -- (-4.1,0);
}

\newcommand{\paintTerminus}[2][black]{
    \draw[color=#1,line width=#2] (-2,-1.8) -- (2,-1.8) arc(-90:90:1.8)
    -- (1.5,1.8) -- (0.2,2.1);

    \draw[color=#1,line width=#2] (0.2,-1.8) -- (1.8,-2.1);

}


\newcommand{\paintYard}[2][black]{
	\draw [color=#1,line width=#2] (-1.8,2.1) -- (-0.5,2.4)
		-- (0.8,2.7);
	\draw [color=#1,line width=#2](-0.5,2.4)
		-- (4,2.4);
	\draw [color=#1,line width=#2](-3,2.7)
		-- (4,2.7);
}

\newcommand{\paintMainDa}[2][black]{

	\draw[color=#1,line width=#2]
	(-4.1,0) arc(-180:-270:2.1) --(2,2.1) arc (90:0:2.1);

}

\newcommand{\paintMainDb}[2][black]{

	\draw[color=#1,line width=#2]
	(4.1,0) arc(0:-90:2.1) --(-2,-2.1) arc (-90:-180:2.1);

}

\newcommand{\paintTerminusDa}[2][black]{
	\draw[color=#1,line width=#2] (3.8,0) arc(0:90:1.8)
    -- (1.5,1.8) -- (0.2,2.1);
}

\newcommand{\paintTerminusDb}[2][black]{
	\draw[color=#1,line width=#2] (-2,-1.8) -- (2,-1.8) arc(-90:0:1.8);

	\draw[color=#1,line width=#2] (0.2,-1.8) -- (1.8,-2.1);
}


\newcommand{\paintYardZones}[2][black]{
	%\draw[color=#1,line width=#2] (-0.95,2.4) -- (0.95,2.4)
	%(1.05,2.4) -- (2.95,2.4)
	%(3.05,2.4) -- (4.95,2.4);

%	\draw[color=#1,line width=#2] (0.05,2.7) -- (0.95,2.7)
%	(1.05,2.7) -- (2.95,2.7)
%	(3.05,2.7) -- (4.95,2.7);

%	\draw[color=#1,line width=#2] (-1.05,2.7) -- (-2.95,2.7) 
%	(-3.05,2.7)--(-3.95,2.7); 
%	\draw[color=#1, line width=#2]
%	($(-1.5,2.4)!0.475!0:(-2.8,2.1)$) -- (-1.5,2.4) --(-0.2,2.7)
%	(-0.05,2.7) --(-0.95,2.7)
%	(-1.05,2.4) --(-1.5,2.4) 
;
}

\newcommand{\paintMainZones}[2][black]{
% 	\draw[color=#1,line width=#2] (-0.95,2.1) -- (0.95,2.1)
% 	(2.05,2.1) -- (2.95,2.1)
% 	($(3,0)!1!89:(5.1,0)$) arc (89:1:2.1)
%    ($(3,0)!1!-1:(5.1,0)$) arc (-1:-89:2.1)
%    (5.1,-0.04) -- ($(3,-0.3)!1!-1:(5.1,-0.3)$) arc (-1:-89:2.1)
%    (1.75,-2.1) -- (1.05,-2.1)
%    (-0.95,-2.1) -- (0.95,-2.1)
%    (-1.05,-2.1) -- (-2.95,-2.1)
%    ($(-3,-0.3)!1!-1:(-3,-2.4)$) arc(-91:-179:2.1) --
%    ($(-3,-0)!1!1:(-5.1,-0)$) arc(181:269:2.1)
% 	($(-3,-0)!1!-1:(-5.1,-0)$) arc(179:91:2.1)
% 	(-1.95,2.1) -- (-1.05,2.1)
%    (2.8,-2.1) -- ($(2.8,-2.1)!.475!0:(1.2,-1.8)$)
% 	(2.95,-2.1) -- (1.85,-2.1)
%    (1.05,2.1) -- (1.95,2.1)
%    (1.2,2.1) -- ($(1.2,2.1)!.475!0:(2.5,1.8)$)
% 	(-2.95,2.1)--(-2.05,2.1)
%    (-2.8,2.1) -- ($(-2.8,2.1)!0.475!0:(-1.5,2.4)$);
}
\newcommand{\paintStationZones}[2][black]{
% 	\draw[color=#1,line width=#2]
%    (2.95,-2.4) -- (1.05,-2.4)
%    (-0.95,-2.4) -- (0.95,-2.4)
%    (-1.05,-2.4) -- (-2.95,-2.4);
}

\newcommand{\paintTerminusZones}[2][black]{
% 	\draw[color=#1,line width=#2]
%  ($(3,0)!1!1:(4.8,0)$) arc(1:89:1.8)
%     (2.95,1.8) -- (2.5,1.8) -- ($(2.5,1.8)!0.475!0:(1.2,2.1)$);
	
%    \draw[color=#1,line width=#2] (-0.95,-1.8) -- (0.95,-1.8)
% 	(2.25,-1.8) -- (2.95,-1.8)
%    ($(3,0)!1!1:(3,-1.8)$)arc(-89:-1:1.8);
% 	\draw [color=#1,line width=#2]
% 	(1.05,-1.8) -- (2.15,-1.8)
%    (1.2,-1.8) -- ($(1.2,-1.8)!.475!0:(2.8,-2.1)$)
% ;
	%\draw[color=#1,line width=#2] (1.2,-1.8) -- (2.8,-2.1);
}

    \caption{Segmentos y Buses}
    \label{fig:ModulosBuses}
\end{figure}

\end{document}