% !TeX encoding = UTF-8
% !TeX spellcheck = es_ES
% !TeX root = Esquema.tex
% !TEX root = Esquema.tex


\section{Sectores}
Los sectores son las diferentes agrupaciones en las que se divide la maqueta desde unos punto de vista electricos. Los fabricantes suelen llamar a cada una de elementos agruados como \textbf{Districtos} y \textbf{Bloques} \sidenote{Solo para Digital}.

La principal caracterisitica de cada elemento es que esta aislado electricamente del resto\sidenote{En maquetas DC o Analogicas, estos  conceptos serian equivalentes a circutitos que puede ser controlado por un mando}. Cada uno de estos elementos (distritos o bloques) va estar precedido por un aparato que conecte dicho elemento a la fuente de energia DCC\sidenote{Ya sea un booster o una central}. La diferencia radica en el objetivo que se quiere obtener con la segmentacion y por lo tanto el dispositivo a conectar. 

\begin{itemize}
\item \textbf{Bloque}: tambien llamado \textit{<<Bloque de deteccion>>}, monitoriza el consumo de corriente y con esa informacion puede inferir si el bloque esta ocupado. 

Puede incluir un detector/lector RailComm y asi poder obtener informacion de la maquina/decoder que esta ocupando el bloque y enviandola al bus LCB.

El bloque debe ser lo más pequeño posible para que solo pueda entrar un elemento a detectar, pero lo suficientemente grande como para detectar el material más grande que se quiera detectar.

\item \textbf{Districto Electrico}: Protege al resto de \textit{districtos} de corto circuitos que puedan exitir en la seccion protegida. El dispositivo conectado controla que el consumo en un   \textit{districto} no supere un cierto umbral, a partir del cual considera que hay un corto y desconecta el distrito de la fuente de potencia.

Hoy en dia las centrales y los booster tienen protecciones ante corto, por lo tanto podria parecer que la division en distritos no es necesaria. Pero si no existen esta particion, un corto, como por ejemplo en la playa de vias, provocaria que toda la maqueta se parara, mientras que con los districtos, en el ejemplo, solo la playa estara sin energia y el resto de la maqueta seguiria funcionando.
\end{itemize}


La maqueta se ha realizado por modulos delimitadas por lo tableros, asi que estos mismos nos ayudan a delimitar los bloques y distritos de la maqueta.


\begin{figure}[H]
    \centering
\begin{tikzpicture}

    %\draw [very thin, green]  (-6,-3) grid (6,3);
	\paintBoard 

	\paintStation[BlueGreen!25]{5pt}
	\paintMainDb[BlueGreen!25]{5pt}  
	\paintYard[BrickRed!25]{5pt}
	\paintTerminusDb[BlueGreen!25]{5pt}
	
	\paintMainDa[BrickRed!25]{5pt}
	\paintTerminusDa[BrickRed!25]{5pt}
	

	\paintStation[gray]{2pt}
   \paintTerminus[gray]{2pt}
	\paintYard[gray]{2pt}
	\paintMain[gray]{2pt}

	\paintYardZones[cyan!50]{1.5pt}
	\paintMainZones[white]{1.5pt}
   \paintStationZones[red]{1.5pt}
   \paintTerminusZones[green]{1.5pt}

	\node[draw=black, text width=4em, line width=1pt](central) at (-7.5,0) {Central Booster};

	\node[draw=black, text width=2em, line width=1pt] (dco1) at (-6,1.5) {DCO};

	\node[draw=black, text width=2em, line width=1pt] (dco2) at (-6,-1.5) {DCO};
	\draw[line width=4pt] (central.east) -- (-6,0)
	-- (dco1.south) -- (dco2.north);

\draw[line width=3pt, BrickRed!25] (dco1.east) -- (-4.5,1.5);
\draw[line width=3pt, BlueGreen!25] (dco2.east) -- (-4.5,-1.5);
	
\end{tikzpicture}

    \caption{Distritos y bloques de la maqueta}
    \label{fig:particionSectores}
\end{figure}

Como se puede apreciar la maqueta se ha divido en dos sectores (Estacion y Playa de vias),
de tal forma que un corto en uno de ellos no afecte al otro. En el diagrama es el color de base (Rosa o Azul)


Asi mismo cada modulo de la misma va a tener un bloque de deteccion por via que tenga. Con la excepcion de los desvios, los cuales se separan en dos bloques si permiten tener dos trenes en el sector. Para diferneciarlos todo el esquema se ha pintado de gris sobre los dos distritos y los bloques con el color de la zona seprados por pequeños huecos.