% !TeX encoding = UTF-8
% !TeX spellcheck = es_ES
% !TeX root = ../Thermal.tex
%!TEX root=../Thermal.tex

\begin{table}[H]
    \centering
    \renewcommand\theadfont{\bfseries}
    \setlength{\tabcolsep}{10pt}
    \renewcommand{\arraystretch}{1.5}
    \begin{tabular}{|l|c|c|c|c|c|c|c|c|c|}
        \hline
        test   & $V_{load}$ & $I_{load}$        & $V_{src}$      & $I_{src}$         & $P_{load}$ & $P_{src}$ & $Eff$ & $T_{Calc}$         & $T_{real}$ \\ \hline
        50\%   & \textcolor{red}{\SI{11,13}{\volt}}& \SI{0.5}{\ampere} & \SI{12}{\volt} & \textcolor{red}{\SI{578}{\milli\ampere}}                   & \textcolor{blue}{\SI{5,565}{\watt}}                       & \textcolor{blue}{\SI{5,565}{\watt}} & \textcolor{blue}{88,33\%}       & \SI{96}{\celsius}  & \textcolor{red}{\SI{50}{\celsius}}\\ \hline

        100\%   & \textcolor{red}{\SI{11}{\volt}}& \SI{1}{\ampere} & \SI{12}{\volt} & \textcolor{red}{\SI{1037}{\milli\ampere}}                   & \textcolor{blue}{\SI{11}{\watt}}                       & \textcolor{blue}{\SI{12,444}{\watt}} & \textcolor{blue}{88,40\%}       & \SI{173}{\celsius}  & \textcolor{red}{\SI{66}{\celsius}}\\ \hline

        120\%   & \textcolor{red}{\SI{11}{\volt}}& \SI{1.2}{\ampere} & \SI{12}{\volt} & \textcolor{red}{\SI{1221}{\milli\ampere}}                   & \textcolor{blue}{\SI{13,2}{\watt}}                       & \textcolor{blue}{\SI{14,652}{\watt}} & \textcolor{blue}{90,09\%}       & \SI{213}{\celsius}  & \textcolor{red}{\SI{71}{\celsius}}\\ \hline
       Limite &            & \SI{1.5}{\ampere}                 & \SI{12}{\volt} & &            &           &       & \SI{272}{\celsius} &            \\
        \hline
    \end{tabular}

    \caption{Resultados CC}
    \label{tab:CcResTable}
\end{table}

En este caso tambien evitamos realizar la prueba de maxima corriente, para evitar posibles daños al diodo GS1510FL.
