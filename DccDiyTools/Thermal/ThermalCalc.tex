% !TeX encoding = UTF-8
% !TeX spellcheck = es_ES
% !TeX root = Thermal.tex
%!TEX root=Thermal.tex

La forma de modelar/estimar que temperatura alcanzara el silicio en un chip es considerar
la potencia que disipa como una fuente de corriente y el camino que tiene hasta el aire como
una resistencia modelo simplificado(a)\ref{fig:ThermalEquivalent}:

\begin{figure}[H]
    \centering
    % !TeX encoding = UTF-8
% !TeX spellcheck = es_ES
% !TeX root = ../Thermal.tex
%!TEX root=../Thermal.tex

\begin{tikzpicture}[american]
    %\draw (0,0) circle[radius=1pt];
    \begin{scope}[shift={(-5,0)}]
        %\draw (0,0) circle[radius=1pt];
        %\draw (-2.5,-3.5) rectangle +(5,7);
        
        \draw (-2,2.5) to[isource, l=$W$ ] (0.5,2.5)
            node[right,red]{$T_j$} 
            to[R, l2=$R_{thJA}$ and $\si{\degree\kelvin\per\watt}$] (0.5,-0)
            node[right,red]{$T_{amb}$} 
            to[vsource=$T_a$] (0.5,-2.5)
            node[right,red]{$\SI{0}{\celsius}$} 
            node[ground](gnd){};
        \node at(0,-3.5) {a) Simplificado};
    \end{scope}


    \begin{scope}[shift={(-0,0)}]
        %\draw (0,0) circle[radius=1pt];
        %\draw (-2.5,-4.75) rectangle +(5,9.5);
        
        \draw (-2,3.75) to[isource, l=$W$ ] (0.5,3.75)
            node[right,red]{$T_j$} 
            to[R, l2=$R_{thJC}$ and $\si{\degree\kelvin\per\watt}$] (0.5,1.25)
            node[right,red]{$T_{case}$} 
            to[R, l2=$R_{thCA}$ and $\si{\degree\kelvin\per\watt}$] (0.5,-1.25)
            node[right,red]{$T_{amb}$} 
            to[vsource=$T_a$] (0.5,-3.75)
            node[right,red]{$\SI{0}{\celsius}$} 
            node[ground](gnd){};
        \node at(0,-4.75) {b) Normal};
    \end{scope}

    \begin{scope}[shift={(5,0)}]
        %\draw (0,0) circle[radius=1pt];
        %\draw (-2.5,-3.5) rectangle +(5,7);
        
        \draw (-2,3.75) to[isource, l=$W$ ] (0.5,3.75)
            node[right,red]{$T_j \leq \SI{125}{\celsius}$} 
            to[R, l2=$R_{thJC}$ and $\si{\degree\kelvin\per\watt}$] (0.5,1.25)
            node[right,red]{$T_{case} \leq \SI{100}{\celsius}$} 
            to[R, l2=$R_{thHS}$ and $\si{\degree\kelvin\per\watt}$] (0.5,-1.25)
        
            node[right,red]{$T_{amb}\approx \SI{25}{\celsius}$} 
            to[vsource=$T_a$] (0.5,-3.75)
            node[right,red]{$\SI{0}{\celsius}$} 
            node[ground](gnd){};
        \node at(0,-4.75) {c) Usable};
    \end{scope}

\end{tikzpicture}

    \caption{Circuito Equivalente}
    \label{fig:ThermalEquivalent}
\end{figure}

En la practica se puede modelar como dos resistencias, una $R_{thJC}$ :
Junction\sidenote{El silicio} a la case\sidenote{Carcasa} y $R_{thCA}$ de la 
carcasa al ambiente, tal y como se representa en el caso (b) \ref{fig:ThermalEquivalent}.
En este modelo es importante mantener la temperatura del silicio $T_j$ por debajo de
$\SI{125}{\celsius}$ lo que se corresponde con $T_c=\SI{100}{\celsius}$ en la carcasa.
En la realidad, el modelo es más complejo, con resistencias en paralelo segun el dispador
que se ponga, pero se simplifica por la diferencia valores y se puede ignorar $R_{thCA}$
por $R_{thHS}$ del disipador\sidenote{HeatSink}.

Estos modelos los podemos complicar un poco más\sidenote{Acercandose más
a la realidad}, dependiendo de si usamos un dispador sobre un componente o 
usamos la propia PCB como disipador. En cuyo caso la resitencia termica es
de $\SI{100}{\degree\kelvin/\square{inch}}$ o $\SI{645.16}{\degree\kelvin/\square\cm}$.
Aunque estos valores dependeran en gran medida de los materiales usados en la 
fabricacion de la PCB.


\begin{figure}[H]
    \centering
    % !TeX encoding = UTF-8
% !TeX spellcheck = es_ES
% !TeX root = ../Thermal.tex
%!TEX root=../Thermal.tex

\begin{tikzpicture}[american]
    %\draw (0,0) circle[radius=1pt];
    \begin{scope}[shift={(-5,0)}]
        %\draw (0,0) circle[radius=1pt];
        %\draw (-3.25,-5.5) rectangle +(6.5,11);
			
			\draw (-2.75,4.5)
				to[isource, l=$W$ ] ++(2.5,0) 
				node [red, right] {$T_{j}$}
				to [R,l=$R_{thJC}$] ++(0,-2.5) 
					coordinate(TC)
				node[red,left]{$T_{case}$}
				to [R,l=$R_{thCA}$] ++ (0,-4)
					coordinate(TA)
				node[red,left]{$T_{amb}$}
	to[vsource,v=$T_{amb}$] ++(0,-2.5)
			node[ground]{};
			\draw (TC) to[short,*-] ++(2,0)
				to[R,l=$R_{thS}$] ++(0,-2.)
				to[R,l=$R_{thHS}$] ++(0,-2.)
				to[short,-*](TA); 
     \end{scope}
\end{tikzpicture}
    \caption{Diseño más complejo}
    \label{fig:ThermalEquivFull}
\end{figure}

Ahora es momento de implificar el circutito teniendo en cuenta la regla del 10
\sidenote{En realidad depende de la tolerancia}. pudiendo eliminar una resistencia
si hay otra 10 veces mas grande o pequeña segun sea el caso:
\begin{itemize}
    \item \textbf{En serie}: Si $R_p$ es 10 menor que $R_g$, podemos eliminarla puesto que 
    $R_p$ se esconderia en la tolerancia de $R_g$.

    En nuestro caso sucede con $R_{thS}$,$R_{thSn}$, siendo, repsecivamente, estas la pasta
    termica que se pone entre un chip y el dispador, y el estaño que une un pad a la PCB.
    ambas cercanas a \SI{1}{\degree\kelvin/\watt}
    \item \textbf{En paralelo}: Si $R_g$ es los suficientemente grande, su valor se esconderia
    en la tolerancia de $R_p$, por lo que puede eleminarse. Para el calculo de temperaturas, donde nuestro
    origen es una fuente de calor representada como fuente de corriente, lo que hace es "<robar">
    un poco de corriente a $R_p$, por lo que al eleminar $R_g$ tendremos un error al alza. 
    Habiendo calculado una temperatura superior a la real, y si esta es segura, la real tambien.
    
    Asi pues con un disipador correcto ($R_{thHS}$, o la red que representa la PCB), podremos quitar
    $R_{thCA}$.
    \item \textbf{Otras}: Como la conexion en delta, se pueden llegar a simplifcar
    pensando en cuanta corriente se llevan, pero lo normal es que los fabricantes ya
    hayan echo los calculos y no sean necesarios. Como el es el caso de $R_{thLC}$.    
\end{itemize}