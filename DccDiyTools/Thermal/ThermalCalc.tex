% !TeX encoding = UTF-8
% !TeX spellcheck = es_ES
% !TeX root = Thermal.tex
%!TEX root=Thermal.tex

La forma de modelar/estimar que temperatura alcanzara el silicio en un chip es considerar
la potencia que disipa como una fuente de corriente y el camino que tiene hasta el aire como
una resistencia modelo simplificado(a)\ref{fig:ThermalEquivalent}:

\begin{figure}[H]
    \centering
    % !TeX encoding = UTF-8
% !TeX spellcheck = es_ES
% !TeX root = ../Thermal.tex
%!TEX root=../Thermal.tex

\begin{tikzpicture}[american]
    %\draw (0,0) circle[radius=1pt];
    \begin{scope}[shift={(-5,0)}]
        %\draw (0,0) circle[radius=1pt];
        %\draw (-2.5,-3.5) rectangle +(5,7);
        
        \draw (-2,2.5) to[isource, l=$W$ ] (0.5,2.5)
            node[right,red]{$T_j$} 
            to[R, l2=$R_{thJA}$ and $\si{\degree\kelvin\per\watt}$] (0.5,-0)
            node[right,red]{$T_{amb}$} 
            to[vsource=$T_a$] (0.5,-2.5)
            node[right,red]{$\SI{0}{\celsius}$} 
            node[ground](gnd){};
        \node at(0,-3.5) {a) Simplificado};
    \end{scope}


    \begin{scope}[shift={(-0,0)}]
        %\draw (0,0) circle[radius=1pt];
        %\draw (-2.5,-4.75) rectangle +(5,9.5);
        
        \draw (-2,3.75) to[isource, l=$W$ ] (0.5,3.75)
            node[right,red]{$T_j$} 
            to[R, l2=$R_{thJC}$ and $\si{\degree\kelvin\per\watt}$] (0.5,1.25)
            node[right,red]{$T_{case}$} 
            to[R, l2=$R_{thCA}$ and $\si{\degree\kelvin\per\watt}$] (0.5,-1.25)
            node[right,red]{$T_{amb}$} 
            to[vsource=$T_a$] (0.5,-3.75)
            node[right,red]{$\SI{0}{\celsius}$} 
            node[ground](gnd){};
        \node at(0,-4.75) {b) Normal};
    \end{scope}

    \begin{scope}[shift={(5,0)}]
        %\draw (0,0) circle[radius=1pt];
        %\draw (-2.5,-3.5) rectangle +(5,7);
        
        \draw (-2,3.75) to[isource, l=$W$ ] (0.5,3.75)
            node[right,red]{$T_j \leq \SI{125}{\celsius}$} 
            to[R, l2=$R_{thHS}$ and $\si{\degree\kelvin\per\watt}$] (0.5,1.25)
            node[right,red]{$T_{case} \leq \SI{100}{\celsius}$} 
            to[R, l2=$R_{thJC}$ and $\si{\degree\kelvin\per\watt}$] (0.5,-1.25)
        
            node[right,red]{$T_{amb}\approx \SI{25}{\celsius}$} 
            to[vsource=$T_a$] (0.5,-3.75)
            node[right,red]{$\SI{0}{\celsius}$} 
            node[ground](gnd){};
        \node at(0,-4.75) {c) Usable};
    \end{scope}

\end{tikzpicture}

    \caption{Circuito Equivalente}
    \label{fig:ThermalEquivalent}
\end{figure}

En la realidad se puede modelar como dos resistencias, una $R_{thJC}$ : Junction\sidenote{El silicio} a la case\sidenote{Carcasa} y $R_{thCA}$ de la carcasa al ambiente,
tal y como se representa en el caso (b) \ref{fig:ThermalEquivalent}.
En este modelo es importante mantener la temperatura del silicio $T_j$ por debajo de $\SI{125}{\celsius}$ lo que se corresponde con $T_c=\SI{100}{\celsius}$ en la carcasa. En la realidad, el modelo es más complejo, con resistencias en paralelo segun el dispador que se ponga, pero se simplifica por la diferencia valores y se puede ignorar $R_{thCA}$ por $R_{thHS}$ del disipador\sidenote{HeatSink}. 

\begin{figure}[H]
    \centering
    % !TeX encoding = UTF-8
% !TeX spellcheck = es_ES
% !TeX root = ../Thermal.tex
%!TEX root=../Thermal.tex

\begin{tikzpicture}[american]
    %\draw (0,0) circle[radius=1pt];
    \begin{scope}[shift={(-3,0)}]
        %\draw (0,0) circle[radius=1pt];
        %\draw (-3.25,-5.5) rectangle +(6.5,11);
			
			\draw (-2.75,4.5)
				to[isource, l=$W$ ] ++(2.5,0) 
				node [red, right] {$T_{j}$}
				to [R,l=$R_{thJC}$] ++(0,-2.5) 
					coordinate(TC)
				node[red,left]{$T_{case}$}
				to [R,l=$R_{thCA}$] ++ (0,-4)
					coordinate(TA)
				node[red,left]{$T_{amb}$}
				to[vsource,v=$T_{amb}$] ++(0,-2.5)
				node[ground]{};
			\draw (TC) to[short,*-] ++(2,0)
				to[R,l=$R_{thS}$] ++(0,-2.)
				to[R,l=$R_{thHS}$] ++(0,-2.)
				to[short,-*](TA);
			\node[] at (0,-5.5){a) Componente TH};
     \end{scope}
	  \begin{scope}[shift={(5,0)}]
			%\draw (-5.25,-5.5) rectangle +(10.5,11);
			
			\draw (-4.75,4.5)
				to[isource, l=$W$ ] ++(2.5,0)
					coordinate(TJ) 
				node [red, above] {$T_{j}$}
				to [R,l=$R_{thJC}$] ++(0,-2.5) 
					coordinate(TC)
				node[red,left]{$T_{case}$}
				to [R,l=$R_{thCA}$] ++ (0,-4)
					coordinate(TA)
				node[red,left]{$T_{amb}$}
				to[vsource,v=$T_{amb}$] ++(0,-2.5)
				node[ground]{};
			\draw (TJ) to[short,*-] ++(2.5,0)
				to[R,l=$R_{thJL}$] ++(0,-2.5)
					coordinate(TL)
					node[red,right]{$T_l$}
				to[R,l=$R_{thLC}$] (TC);
			\draw (TL) to [R,l=$R_{thS}$] ++ (0,-2)
					coordinate(Ttop)
					node[red,left]{$T_{top}$}
				to[R,l=$R_{thTopPcb}$]++(0,-2)
				to[short,-*](TA);
			\draw (Ttop) to [vsource,v_=$\SI{10}{\degree\kelvin}$] ++(3,0)
			node[red,right ]{$T_{bott}$}
			to[R,l=$R_{thBottPcb}$] ++(0,-2)
			to[short,-*] (TA)
			;

			%\draw (TC) to[short,*-] ++(2,0)
			%	to[R,l=$R_{thS}$] ++(0,-2.)
			%	to[R,l=$R_{thHS}$] ++(0,-2.)
			%	to[short,-*](TA);
			
			\node[] at (0,-5.5){b) SMD usando PCB como Disipador};
     \end{scope}

\end{tikzpicture}
    \caption{Diseño más complejo}
    \label{fig:ThermalEquivFull}
\end{figure}