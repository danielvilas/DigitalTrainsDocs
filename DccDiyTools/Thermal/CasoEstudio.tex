% !TeX encoding = UTF-8
% !TeX spellcheck = es_ES
% !TeX root = Thermal.tex
%!TEX root=Thermal.tex

Como caso practico de estudio vamos a usar la placa DccDiyTools DccPowerDistributor y los elementos que
pueden disipar suficiente potencia como para dañar el componente. En dicha placa los 
componetes que cumplen ese requisito son:
\begin{itemize}
    \item Dos diodos GS1510FL
    \item Un Mosfet BSS308
    \item Un puente de diodos KMB26STR
    \item Un Controlador de Buck Converter TPS54331D
    \item Una Bobina del Buck Converter SRP1038A-220M
    \item Un Regulador LDO AMS1117-3.3
\end{itemize}
A grandes rasgos la potencia entra o bien por la señal DCC o bien por un conector jack y luego pasa por un convertidor buck y un regulador LDO. Como podemos ver en la figura siguiente\ref{fig:Bloques}
\begin{figure}[H]
    \centering
    % !TeX encoding = UTF-8
% !TeX spellcheck = es_ES
% !TeX root = ../Thermal.tex
%!TEX root=../Thermal.tex

\begin{tikzpicture}[]


	\begin{scope}[shift={(0,5)}]
		\node[gray, right] at (-6.25,2){Entrada};
		\draw[dashed,gray](-6.25,-2.25) rectangle +(12.5,4.5);
		\begin{scope}[shift={(-4.25,1)}]
			\pic[](DccBr) at (2,0){DiodeBrigde};
			\node[above] at (DccBr-up){KMB26STR};

			%\draw (1,0) to[short,o-] (DccBr-up);
			\draw (DccBr-ac.north -| 0,0)to[short,o-](DccBr-ac.north);
			\draw (DccBr-ac.south -| 0,0)to[short,o-](DccBr-ac.south);
			\draw (DccBr-cc.center) to [D,l=GS1510FL,n=DccD] (6,0)
			node[pigfetd,anchor=S, rotate=90, bodydiode,](DccSw){}
			(DccSw.east |- DccD.north) node[above = 1mm] {BSS308} (DccSw.D)
			to[short] (8,0)
			coordinate(DccVdrive);
			\node[left]at (0,0) {Señal DCC};

		\end{scope}

		\begin{scope}[shift={(-4.25,-1.5)}]

			\node[left]at (0,0) {Entrada CC};

			\draw (0,0) to[short,o-] (DccSw.S |- 0,0)
			to [D,l=GS1510FL] (DccSw.D |- 0,0)
			to[short] (DccVdrive |- (0,0)
			coordinate(CcVdrive)
			(DccD.center |- 0,1) node {Controla};
			\draw[dashed,-stealth] (DccD.center |- 0,0) |- (DccSw.G);
			\draw (DccVdrive) to[short,-*] (CcVdrive) to[short,-o] +(1,0)
			coordinate(OutVdrive)
			node[right]{VDrive};
		\end{scope}
	\end{scope}
	%\draw[red](-6.25,-2.25) rectangle +(12.5,4.5);
	\begin{scope}[shift={(0,0)}]
		\node[gray, right] at (-3,1.25){Buck Converter};
		\draw[dashed,gray](-3,-1.5) rectangle +(6,3);

		\begin{scope}[shift={(-1.5,-0.25)}]
			\draw (-1,0.5) coordinate(BuckIn)
				to[short,-o] (-0.7,0.5)
				node[scale=0.5,right]{$V_{in}$};
			\draw (1,0.5) coordinate(BuckOut)
				to[short,-o] (0.7,0.5)
				node[scale=0.5,left]{$P_h$};
			\draw (1,-0.5) coordinate(BuckFb)
				to[short,-o] (0.7,-0.5)
				node[scale=0.5,left]{$F_b$};
			\node[scale=0.75] at (0,0){TPS54331D};
			\draw (-0.9,-0.75) rectangle + (1.8,1.5);
		\end{scope}
		\draw (BuckOut) to[cute inductor,l=SRP1038A-220M] +(3,0)
			coordinate(BuckInductor)
			to[short,-o] (BuckInductor -| OutVdrive)
			node[right]{+5V};
		\draw[dashed,-stealth] (BuckInductor) |- (BuckFb);
		
	\end{scope}

	\draw (CcVdrive) |- +(-7.5,-1.5) coordinate(PwrRef) |- (BuckIn);

	\begin{scope}[shift={(0,-4.25)}]
		\node[gray, right] at (-3,1.25){LDO};
		\draw[dashed,gray](-3,-1.5) rectangle +(6,3);

		\begin{scope}[shift={(0,.25)}]
			\draw (-1,0.5) coordinate(LdoIn)
				to[short,-o] (-0.7,0.5)
				node[scale=0.5,right]{$V_{in}$};
			\draw (1,0.5) coordinate(LdoOut)
				to[short,-o] (0.7,0.5)
				node[scale=0.5,left]{$P_h$};
			\draw (0,-0.85) coordinate(LdoAdj)
				to[short,-o] (0,-0.55)
				node[scale=0.5,above]{$A_{dj}$};
			\node[scale=0.75] at (0,0){AMS1117-3.3};
			\draw (-0.9,-0.75) rectangle + (1.8,1.5);
		\end{scope}
		\node[ground] at(LdoAdj){};
		\draw (LdoOut) to[short,-o] (LdoOut -| OutVdrive)
		node[right]{+3.3V};

	\end{scope}

	\draw (CcVdrive |- BuckOut) |- +(-7.5,-2.5) |- (LdoIn);
	\node[draw=gray,text width=3.5cm, scale=0.75] at (-5,-5) {* Se han omitido componentes extra como Condensadores};

\end{tikzpicture}
    \caption{Diseño de bloques}
    \label{fig:Bloques}
\end{figure}

Como se aprecia la potencia DCC pasar por un Diodo GS1510FL y por un Mosfet BSS308. La mision de dicho mosfet es bloquear\sidenote{No se muestra
la circuiteria} la potencia si la entrada CC esta presente. El mosfet esta siempre abierto o cerrado.

Si esta el jack de CC esta conectado, se bloqueara el mosfet y solo pasara la corriente por un Didodo GS1510FL.

A partir de aqui se genera el rail VDrive del que se conectan ya las cargas que necesiten unos 12V o similar y un Convertidor Buck para sacar 5V.

De los 5V se utilizar un Regulador LDO para tener un rail 3.3V.

Se debera llenar la siguiente tabla para cada opcion

\begin{table}[H]
    \centering
    \renewcommand\theadfont{\bfseries}
    \setlength{\tabcolsep}{10pt}
    \renewcommand{\arraystretch}{1.5}
    \begin{tabular}{|l|c|c|c|c|c|c|c|c|c|}
        \hline
         test & $V_{load}$ & $I_{load}$& $V_{src}$& $I_{src}$ & $P_{load}$&$P_{src}$&$Eff$ &$T_{Calc}$ &$T_{real}$ \\ \hline
			50\% & & \SI{500}{\milli\ampere}& \SI{12}{\volt}&& & & & XX &\\ \hline

100\% & & \SI{1000}{\milli\ampere}& \SI{12}{\volt}&& & & & XX &\\ \hline

120\% & & \SI{1200}{\milli\ampere}& \SI{12}{\volt}&& & & & XX &\\ \hline

Limite & & \SI{1500}{\milli\ampere}& \SI{12}{\volt}&& & & & XX &\\

			\hline

	\end{tabular}

    \caption{Tabla a rellenar por cada prueba}
    \label{tab:exampleDataLab}
\end{table}


\subsection{Diseño de las pruebas}
El objetivo de las pruebas es triple, por una parte verificar las ecuaciones, otra verificar que funciona con los requisitios decididos y la eficiencia del sistema.
\subsubsection{Rail DCC}
El rail DCC se probara poniendo una carga programable en la salida del rail VDrive y una fuente de laboratorio configurada a 12V y limite 1.5A



