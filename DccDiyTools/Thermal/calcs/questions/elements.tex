% !TeX encoding = UTF-8
% !TeX spellcheck = es_ES
% !TeX root = ../../Thermal.tex
%!TEX root=../../Thermal.tex
Algunas veces conviene pararnos a pensar si nuestro diseño actual es seguro
para el chip, debemos ver a que temperatura llega cada parte del diseño. 

Asi que para calcular la temperatura de diversos puntos del chip, como la carcasa, la
parte superior de la pcb,\dots, se debe escoger el modelo que más se adecue a
los puntos a calcular y resolver.

Partiremos siempre de la premisa de que la placa se ha diseñado para disipar
calor ya sea por tener un dispador o por suficiente superficie en la placa
que sea disipadora. Y a partir de ahi simplifcar el circuito termico. Todo
aquello que no se haya pensado como dispador, debe ser excluido del modelo.

La primera simplificacion.
