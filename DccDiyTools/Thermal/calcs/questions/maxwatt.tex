% !TeX encoding = UTF-8
% !TeX spellcheck = es_ES
% !TeX root = ../../Thermal.tex
%!TEX root=../../Thermal.tex

Para calcular la potencia maxima que se puede disipar para una configuracion dada conviene
calcular la $R_{thJA}$ equivalente, ya sea realizando toda la simplificacion o empiricamente.
Una vez calculada dicha Resistencia es cuestion de despejar:

\begin{align*}
	T_j=T_{amb}+W_{max} \cdot R_{thJA} & & \text{Original} \\
	125&=25+W_{max} \cdot R_{thJA} & & \text{Sustituimos limite} \\
\frac{125-25}{R_{thJA}}&=W_{max} & & \text{Despejamos}
\end{align*}
Con lo que llegamos a la ecuacion \ref{eq:ResMaxWatt}:
\begin{equation}
\label{eq:ResMaxWatt}
W_{max}=\frac{100}{R_{thJA}}
\end{equation}
