% !TeX encoding = UTF-8
% !TeX spellcheck = es_ES
% !TeX root = ../Thermal.tex
%!TEX root=../Thermal.tex
El modelo más simplificado es una resistencia, que podremos usarlo cuando este dato nos sea dado
por el fabricante en el DataSheet.
El más realista son dos resistencias en serie, una representado la resistencia desde el silicio al
disipador (ya sea, la carcasa, la pcb o un trozo de metal) y otra de este ultimo al aire.



\begin{figure}[H]
    \centering
    % !TeX encoding = UTF-8
% !TeX spellcheck = es_ES
% !TeX root = ../Thermal.tex
%!TEX root=../Thermal.tex

\begin{tikzpicture}[american]
    %\draw (0,0) circle[radius=1pt];
    \begin{scope}[shift={(-5,0)}]
        %\draw (0,0) circle[radius=1pt];
        %\draw (-2.5,-3.5) rectangle +(5,7);
        
        \draw (-2,2.5) to[isource, l=$W$ ] (0.5,2.5)
            node[right,red]{$T_j$} 
            to[R, l2=$R_{thJA}$ and $\si{\degree\kelvin\per\watt}$] (0.5,-0)
            node[right,red]{$T_{amb}$} 
            to[vsource=$T_a$] (0.5,-2.5)
            node[right,red]{$\SI{0}{\celsius}$} 
            node[ground](gnd){};
        \node at(0,-3.5) {a) Simplificado};
    \end{scope}


    \begin{scope}[shift={(-0,0)}]
        %\draw (0,0) circle[radius=1pt];
        %\draw (-2.5,-4.75) rectangle +(5,9.5);
        
        \draw (-2,3.75) to[isource, l=$W$ ] (0.5,3.75)
            node[right,red]{$T_j$} 
            to[R, l2=$R_{thJC}$ and $\si{\degree\kelvin\per\watt}$] (0.5,1.25)
            node[right,red]{$T_{case}$} 
            to[R, l2=$R_{thCA}$ and $\si{\degree\kelvin\per\watt}$] (0.5,-1.25)
            node[right,red]{$T_{amb}$} 
            to[vsource=$T_a$] (0.5,-3.75)
            node[right,red]{$\SI{0}{\celsius}$} 
            node[ground](gnd){};
        \node at(0,-4.75) {b) Normal};
    \end{scope}

    \begin{scope}[shift={(5,0)}]
        %\draw (0,0) circle[radius=1pt];
        %\draw (-2.5,-3.5) rectangle +(5,7);
        
        \draw (-2,3.75) to[isource, l=$W$ ] (0.5,3.75)
            node[right,red]{$T_j \leq \SI{125}{\celsius}$} 
            to[R, l2=$R_{thJC}$ and $\si{\degree\kelvin\per\watt}$] (0.5,1.25)
            node[right,red]{$T_{case} \leq \SI{100}{\celsius}$} 
            to[R, l2=$R_{thHS}$ and $\si{\degree\kelvin\per\watt}$] (0.5,-1.25)
        
            node[right,red]{$T_{amb}\approx \SI{25}{\celsius}$} 
            to[vsource=$T_a$] (0.5,-3.75)
            node[right,red]{$\SI{0}{\celsius}$} 
            node[ground](gnd){};
        \node at(0,-4.75) {c) Usable};
    \end{scope}

\end{tikzpicture}

    \caption{Circuito Equivalente}
    \label{fig:ThermalEquivalent}
\end{figure}

En la practica es mejor modelar como dos resistencias, una $R_{thJC}$ :
Junction\sidenote{El silicio} a la case\sidenote{Carcasa} y $R_{thCA}$ de la
carcasa al ambiente, tal y como se representa en el caso (b) \ref{fig:ThermalEquivalent}.
De esta forma podremos variar $R_{thCA}$ con un dispador más grande.

En este modelo es importante mantener la temperatura del silicio $T_j$ por debajo de
$\SI{125}{\celsius}$ lo que se suele corresponde con $T_c=\SI{100}{\celsius}$ en la carcasa.
Como hemos visto, en la realidad, el modelo es más complejo, con resistencias en paralelo segun el dispador
que se ponga, pero se simplifica por la diferencia valores y se puede ignorar $R_{thCA}$
por $R_{thHS}$ del disipador\sidenote{HeatSink}.
