% !TeX encoding = UTF-8
% !TeX spellcheck = es_ES
% !TeX root = ../Thermal.tex
%!TEX root=../Thermal.tex
Generalmente querremos saber si nuestro diseño es viable, pero podemos concretar
en que queremos responder una de estas preguntas:
\begin{itemize}
    \item ¿Cuanta es la potencia maxima que podemos usar en este componente?
    \item ¿Que temperatura alcanzara el componente con la potencia que va a pasar?
          o cada parte (carcasa, pcb, \dots)
    \item ¿Cual es el disipador minimo que tenemos que poner? o que es lo mismo
          ¿Que resistencia maxima debe tener el disipador?
    \item ¿Que area minima debemos usar en la PCB para disipar X potencia?
\end{itemize}

La mejor forma de conseguir esto es diseñar una PCB pequeña, de tamaño similiar
al standart JESD51 de JEDEC, pero modificada para que se parezca a nuestra PCB
final y probar con una fuente de laboratorio y una camara de infrarojos. La segunda
mejor forma es sobre nuestras PCB.

Aunque hoy por hoy es barato obtener una PCB, es mucho mas rapido hacer unos pocos
calculos y luego comprobarlos sobre la PCB. Los componentes siguen teniendo un precio

Para cualquier pregunta, deberemos trabjar con el modelo de una o dos resistencias.
Para ello trabajaremos solo con un camino de disipacion y obviaremos el otro.

En el caso de los componentes through hole solo tenemos dos opciones, con o sin
disipador. Si no tenemos dispador solo tendremos el camino por la carcasa al
aire, pero si ponemos un disipador tenemos  dos caminos. En este caso podemos
eliminar el camino que va por la carcasa.

En cambio, para los componentes SMD, siempre tendremos la disipacion por PCB,
aunque no añamos diseñado la PCB como disipador. Si no hemos diseñado la PCB
a conciencia\sidenote{Podria ser que sin querer sea un buen disipador} solo
tenedremos en cuenta la disipacion por la carcasa. Y si nuestra intencion es
usar la PCB como disipador, es mejor descartar la carcasa y diseñar la PCB
como unica opcion de disipador.

Al omitir un camino lo que conseguimos es un error al alza en la temperatura
por lo que si nuestro diseño es seguro\sidenote{$T_j< \SI{125}{\celsius}$}
segun los calculos lo sera en la realidad.

Como ejemplo vamos usar un chip 78XX de onsemi, MC7800 cuyo datasheet nos dice
lo siguiente:
\begin{table}[H]
    \centering
    \renewcommand\theadfont{\bfseries}
    \setlength{\tabcolsep}{10pt}
    \renewcommand{\arraystretch}{1.5}
    \begin{tabular}{|l|c|c|c|c|}
        \hline
                                  &                          &            & \multicolumn{2}{|c|}{Empaquetado}                      \\ \hline
        desc                      & PadSize                  & Symbol     & TO-220                            & \makecell{TO-252-3 \\ DPAK } \\ \hline
        Th R, Junction-to-Case    &                          & $R_{thJC}$ & 5                                 & 5                  \\ \hline
        Th R, Junction-to-Ambient &                          & $R_{thJA}$ & 65                                & 96                 \\ \hline
        Th R, Junction-to-Ambient & $5*\SI{5}{\square\mm}$   & $R_{thJA}$ & -                                 & 67                 \\ \hline
        Th R, Junction-to-Ambient & $10*\SI{10}{\square\mm}$ & $R_{thJA}$ & -                                 & 56                 \\ \hline
        Th R, Junction-to-Ambient & $20*\SI{20}{\square\mm}$ & $R_{thJA}$ & -                                 & 49                 \\ \hline
        Max Junction Temperature  &                          & $T_j$      & 150                               & 150                \\
        \hline
    \end{tabular}
    \caption{Datos suministrados por el DataSheet}
    \label{tab:MC7800_ds_thermal_data}
\end{table}

Como podemos ver no nos dice $R_{thCA}$ ni $R_{thPCB}$ y tenemos diferentes $R_{thJA}$
segun el tamaño del pad. Con esta informacion podemos averiguar los datos que nos faltan
usando las siguentes formulas.


\begin{subequations}
    \label{eq:basicRth}
    \begin{align}
        \label{eq:RthSinDisipador} R_{thJA}          & =R_{thJC}+R_{thCA}                     &  & \text{Sin disipador }     \\
        \label{eq:RthConDisipador} R_{thJA}(a_{pcb}) & =R_{thJC}+R_{thCA}||R_{thPCB}(a_{pcb}) &  & \text{Con disipador(PCB)}
    \end{align}
\end{subequations}

De las cuales podemos deducir las siguientes ecuaciones:

\begin{subequations}
    \label{eq:ResRth}
    \begin{align}
        \label{eq:RthJA} R_{thCA}            & =R_{thJA}-R_{thJC}                                                    \\
        \label{eq:RthPcb} R_{thPcb}(a_{pcb}) & =\frac{1}{\frac{1}{R_{thJA}(a_{pcb}) - R_{thJC}} -\frac{1}{R_{thCA}}}
    \end{align}
\end{subequations}

Siguiendo los siguientes pasos:

\begin{align*}
    R_{thCA}                               & =R_{thJA}-R_{thJC}                                                    &  & \text{Despejando de \ref{eq:RthSinDisipador}}      \\
    \\
    R_{thJA}(a_{pcb})                      & =R_{thJC}+\frac{1}{\frac{1}{R_{thCA}}+\frac{1}{R_{thPcb}(a_{pcb})}}   &  & \text{Applicando  || en  \ref{eq:RthConDisipador}} \\
    R_{thJA}(a_{pcb}) - R_{thJC}           & =\frac{1}{\frac{1}{R_{thCA}}+\frac{1}{R_{thPcb}(a_{pcb})}}            &  & \text{Despejando}                                  \\
    \frac{1}{R_{thJA}(a_{pcb}) - R_{thJC}} & = \frac{1}{R_{thCA}}+\frac{1}{R_{thPcb}(a_{pcb})}                     &  & \text{Aplicando $1/x$}                             \\
    \frac{1}{R_{thPcb}(a_{pcb})}           & =\frac{1}{R_{thJA}(a_{pcb}) - R_{thJC}} -\frac{1}{R_{thCA}}           &  & \text{Despejando}                                  \\
    R_{thPcb}(a_{pcb})                     & =\frac{1}{\frac{1}{R_{thJA}(a_{pcb}) - R_{thJC}} -\frac{1}{R_{thCA}}} &  & \text{Aplicando $1/x$}
\end{align*}

Con estas ecuaciones obtenemos los datos que nos faltan:

\begin{table}[H]
    \centering
    \renewcommand\theadfont{\bfseries}
    \setlength{\tabcolsep}{10pt}
    \renewcommand{\arraystretch}{1.5}
    \begin{tabular}{|l|c|c|c|c|}
        \hline
                              &                      &            & \multicolumn{2}{|c|}{Empaquetado}                      \\ \hline
        desc                  & PadSize              & Symbol     & TO-220                            & \makecell{TO-252-3 \\ DPAK } \\ \hline
        Th R, Case-to-Ambient &                      & $R_{thCA}$ & 60                                & 91                 \\ \hline
        Th R, Case-to-Ambient & 5x5\si{\square\mm}   & $R_{thCA}$ & -                                 & 195                \\ \hline
        Th R, Case-to-Ambient & 10x10\si{\square\mm} & $R_{thCA}$ & -                                 & 122                \\ \hline
        Th R, Case-to-Ambient & 20x20\si{\square\mm} & $R_{thCA}$ & -                                 & 86                 \\ \hline
    \end{tabular}

    \caption{Datos suministrados por el DataSheet}
    \label{tab:MC7800_calc_thermal_data}
\end{table}

\subsubsection{Maxima Potencia}
% !TeX encoding = UTF-8
% !TeX spellcheck = es_ES
% !TeX root = ../../Thermal.tex
%!TEX root=../../Thermal.tex

Para calcular la potencia maxima que se puede disipar para una configuracion dada conviene
calcular la $R_{thJA}$ equivalente, ya sea realizando toda la simplificacion o empiricamente.
Una vez calculada dicha Resistencia es cuestion de despejar:

\begin{align*}
	T_j=T_{amb}+W_{max} \cdot R_{thJA} & & \text{Original} \\
	125&=25+W_{max} \cdot R_{thJA} & & \text{Sustituimos limite} \\
\frac{125-25}{R_{thJA}}&=W_{max} & & \text{Despejamos}
\end{align*}
Con lo que llegamos a la ecuacion \ref{eq:ResMaxWatt}:
\begin{equation}
\label{eq:ResMaxWatt}
W_{max}=\frac{100}{R_{thJA}}
\end{equation}


\subsubsection{Temperatura de cada elemento}
% !TeX encoding = UTF-8
% !TeX spellcheck = es_ES
% !TeX root = ../../Thermal.tex
%!TEX root=../../Thermal.tex
Para calcular la temperatura de diversos puntos del chip, como la carcasa, la
parte superior de la pcb,\dots se debe escoger el modelo que más se adecue a
los puntos a calcular y resolver.

