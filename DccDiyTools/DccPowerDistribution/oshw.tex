% !TeX encoding = UTF-8
% !TeX spellcheck = es_ES
% !TeX root = DccPowerDistribution.tex

La gama de dispositivos DccDiyTools estan bajo una licencia OSHW, que se define como:

"<Hardware de Fuentes Abiertas (OSHW en inglés) es aquel hardware cuyo diseño se hace disponible
públicamente para que cualquier persona lo pueda estudiar, modificar, distribuir, materializar
y vender, tanto el original como otros objetos basados en ese diseño. Las fuentes del hardware
(entendidas como los ficheros fuente) habrán de estar disponibles en un formato apropiado para
poder realizar modificaciones sobre ellas. Idealmente, el hardware de fuentes abiertas utiliza
componentes y materiales de alta disponibilidad, procesos estandarizados, infraestructuras abiertas,
contenidos sin restricciones, y herramientas de fuentes abiertas de cara a maximizar la habilidad
de los individuos para materializar y usar el hardware. El hardware de fuentes abiertas da libertad
de controlar la tecnología y al mismo tiempo compartir conocimientos y estimular la comercialización
por medio del intercambio abierto de diseños.">
\href{https://www.oshwa.org/definition/spanish/}{Extracto de la licencia por OSHWA}

\subsection{Licencia Resumida}
La licencia completa se puede obtener de \href{https://www.oshwa.org/definition/spanish/}{Open Software
Hardware Association - (OSHWA)} pero en resumidas cuentas significa que:
\begin{itemize}
    \item Cualquier persona pueda fabricar, modificar, distribuir y usar esos objetos.
    \begin{itemize}
        \item Quien lo haga tiene la oblicacion de indicar que su producto no ha sido
manufacturado, vendido, garantizado o autorizado en cualquier forma por el diseñador original.
        \item Tampoco puede usar ninguna marca registrada del diseñador original.
    \end{itemize}
    \item Se tiene acceso a la documentacion libremente de por no más que un razonable coste de reproducción.
    \begin{itemize}
        \item La documentacion, se refiere principalmente a los documentos CAD necesarios para manufacturar
        el producto.
        \item Debe estar en un formato facilmente modificable
    \end{itemize}
    \item El software necesario\sidenote[][]{Firmware, software de control,...} debe ser libre o bien
    documentado para que cualquiera pueda crearlo.
    \item Cualquiera persona puede crear obras derivadas y redistirbuir (incluyendo venta) con la misma licencia
de la obra original
    \begin{itemize}
        \item No se deberia requerir el pago de licencias, regalias o similares por la venta.
        \item El diseñador original puede exigir la atribucion de autoria y restringir el nombre en las obras
derivadas.
    \end{itemize}
    \item Cualquiera significa Cualquiera. Sin discriminacion de raza, sexo,... ni discriminacion de ambito
de aplicacion. Nadie debe requerir de una licencia especial para disfrutar de los derechos aqui defenidos.
    \item No se puede restringir otros productos que se distribuyan junto al producto licenciado. 

\end{itemize}

En resumen "<Cualquier persona pueda fabricar, modificar, distribuir y usar esos objetos."> y el diseñador 
puede añadir alguna pequeña restriccion.

\subsection{Licencia DccPowerDistribution}
DccPowerDistribution se licencia mediante la definicion publicada por la \href{https://www.oshwa.org/definition/spanish/}{OSHWA}
con las siguientes consideraciones:
\begin{itemize}
    \item Toda la documentacion y ficheros se encuentra en repositorios publicos de github.
    \begin{itemize}
        \item \url{https://github.com/danielvilas/DigitalTrains/tree/master/DccBlocks/DccPowerDistribution/board}
        \item \url{https://github.com/danielvilas/DigitalTrainsDocs/tree/main/DccDiyTools/DccPowerDistribution} 
    \end{itemize}
    \item Esta licencia OSHW se aplica para la placa electronica DccPowerDistribution 
    \item Los documentos anexos, como manuales, read me y similares, estan bajo la licencia Creative Commons By Share-Alike 
    \ccbysa
    \item Para editar la placa es necesario usar KiCAD 6.0.6 o superior. La documentacion se encentra en \LaTeX
    \item Cualquier placa derivada debe cambiar el nombre lo suficiente como para que sepa que es una placa derivada
    \begin{itemize}
        \item Cambiar el numero de version \textbf{no} es suficiente
        \item Seria valido añadir un nombre corto ([Nombre]DccPowerDistribution o DccPowerDistribution[Nombre]) 
    \end{itemize}
    \item Cualquiera puede manufacturar esta placa y venderla, sin pago de licencias ni regalias al diseñador.
Pero considera hacer una donacion al diseñador, sobre todo en el caso de venta y teniendo un beneficio economico
considerable. 
\end{itemize}