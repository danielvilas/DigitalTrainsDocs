% !TeX encoding = UTF-8
% !TeX spellcheck = es_ES
% !TeX root = DccPowerDistribution.tex
\subsection{Objetivos}
Como se ha dicho DccPowerDistribution es modulo cuyo objetivo principal dotar de alimentacion para otros modulos DiY
para facilitar el desarrollo de otros modulos por parte del usuario. Los objetivos tecnicos del modulo son:
\begin{itemize}
    \item \textbf{Autoseleccion DCC - CC} Cuado esten los dos buses conectados debe usar Corriente Continua.
sin que el usuario tenga que cambiar interruptores.    
    \item \textbf{1A Corriente} continua: que permita mover un motor y algunos accesorios
sin problemas
    \item \textbf{1.5A Corriente} maximo. Como margen de seguridad.
    \item \textbf{Tres salidas de Voltaje}:
    \begin{itemize}
        \item \textbf{VDrive}: El Voltaje de entrada para alimentar otros modulos.
        \item \textbf{+5V}: 5 Voltios para arduinos y otros componentes
        \item \textbf{+3.3V} 3.3 Voltios para los nuevos Arduinos ARM y otros microcontroladores similares
    \end{itemize}
    \item \textbf{Bus 12V}: Los buses de 12V de Corriente Continua son los más utilizados para los accesorios
tales como bombillas y otros componentes. DccPowerDistribution esta pensado para utilizar este bus.
\end{itemize}

\subsection{Diagrama de Bloques}
El diseño del modulo en bloques es el siguiente:
\begin{figure}[H]
    \centering
    \begin{tikzpicture}
    %\draw [very thin, green] (-1,-3) grid (12,4);
    \begin{scope}[shift={(0,2)}]
        \node [left] (txtCcB) at (0, 1){CC Bloque};
        \pic(picCcB)[rotate=-90] at (1,1)  {screwTerminal} ;
        \node [left](txtCcJ) at (0, 0){CC Jack};
        \pic (picCcJ) at (1,0)  {jack} ;

        \draw[] (txtCcB.east) -- (picCcB-cable-down.center);
        \draw[] (txtCcJ.east) -- (picCcJ-cable.center);
        \node[draw](proteccion) at (4,0.5) {Protección};

        \draw[] (picCcB-cable-up.center) -- (2.25,1) 
            -- (2.25, 0.5) -- (proteccion.west);
        \draw[] (picCcJ-point.center) -- (2.25,0)
            -- (2.25, 0.5);
        \node[](ptVDrive)  at (proteccion.east-| 6,0) {};
        \node[above](lblVdrive) at (ptVDrive.center)  {VDrive};
        \draw (proteccion.east) -- (ptVDrive.center);
        \draw (2.75, 0.5) -- (2.75,1.25) -- (5.25,1.25) 
            -- (5.25,0.5);
        \pic[] at (proteccion.center |- 0,1.25) {SmallDiode};
        \draw[blue, dashed] (3.5,1.25) -- (3.5,1.5) -- (4.5,1.5)
            -- (4.5,1.25);
        \draw[blue, dashed] (proteccion.center |- 0,0)
            -- (5.25,0) -- (5.25,.5);
    \end{scope}

    \begin{scope}
        \node [left] (txtDccIn) at (0, 0){DCC};
        \pic[rotate=-90] (picDccIn) at(1,0){JstXh2};
        \pic[](DccDiode) at (3,0){DiodeBrigde};

        \draw[] (txtDccIn.east) -- (picDccIn-cable.center);
        \draw[] (picDccIn-pcb.center) -- (DccDiode-ac.center);

        \draw[] (DccDiode-cc.center) -- (DccDiode-cc.center -| proteccion.south)
            -- (proteccion.south);
    \end{scope}

    \begin{scope}[shift={(ptVDrive.center |- 0,0)}]
        \pic[](picOutVdrive)  at (0,0){screwTerminal};
        \node[below](lblOutVdrive)  at (picOutVdrive-cable-down.center){Vdrive};
        \draw[] (ptVDrive.center) -- (picOutVdrive-cable-up.center);
        \pic[rotate=-90,yscale=-1](picLedVdrive)  at (-0.75,0){SmallLed=green};
        \draw (0,0.5) -- (picLedVdrive-p.center |- 0,0.5) -- (picLedVdrive-p.center);
        \node[red, scale=.75]at(-0.4,0.5){X};
    \end{scope}

    \begin{scope}[shift={(ptVDrive.center -| 7.5,0)}]
        \node[draw](buck) at (0,0) {Buck}; 
        
        \node[](pt5V)  at (1.5,0) {};
        \node[above](lbl5V) at (pt5V.center)  {+5V};
        \draw (buck.east) -- (pt5V.center);
        \draw (ptVDrive.center) -- (buck.west);
    \end{scope}

    \begin{scope}[shift={(pt5V.center |- 0,0)}]
        \pic[](picOut5V)  at (0,0){screwTerminal};
        \node[below](lblOut5)  at (picOut5V-cable-down.center){+5V};
        \draw[] (pt5V.center) -- (picOut5V-cable-up.center);
        \pic[rotate=-90,yscale=-1](picLed5V)  at (-0.75,0){SmallLed=green};
        \draw (0,0.5) -- (picLed5V-p.center |- 0,0.5) -- (picLed5V-p.center);
        \node[red, scale=.75]at(-0.4,0.5){X};
    \end{scope}


    \begin{scope}[shift={(pt5V.center -| 10.5,0)}]
        \node[draw](ldo) at (0,0) {LDO};  
        \node[](pt33V)  at (1.5,0) {};
        \node[above](lbl33V) at (pt33V.center)  {+3.3V};
        \draw (ldo.east) -- (pt33V.center);
        \draw (pt5V.center) -- (ldo.west);
    \end{scope}

    \begin{scope}[shift={(pt33V.center |- 0,0)}]
        \pic[](picOut33V)  at (0,0){screwTerminal};
        \node[below](lblOut33)  at (picOut33V-cable-down.center){+3.3V};
        \draw[] (pt33V.center) -- (picOut33V-cable-up.center);
        \pic[rotate=-90,yscale=-1](picLed33V)  at (-0.75,0){SmallLed=green};
        \draw (0,0.5) -- (picLed33V-p.center |- 0,0.5) -- (picLed33V-p.center);
        \node[red, scale=.75]at(-0.4,0.5){X};
    \end{scope}

    \begin{scope}[shift={(lblVdrive.center |- 3,-2)}]
        \pic[](dccIface)at (0,0){DccIface};
        
    \end{scope}
    \draw[] (picDccIn-pcb.center -| 1.9,0) -- (1.9,0 |- dccIface-west-out.center) 
        -- (dccIface-west-out.center);
        \draw[] (4,0) -- (4,0 |- dccIface-west-in.center) 
        -- (dccIface-west-in.center);


    \begin{scope}[shift={(lbl5V.center |- dccIface-west.center)}]
        \pic[rotate=90](dccOut)at (0,0){JstXh4};

        \node[right](lblDccOutGnd) at (1,0.6) {GND};
        \node[right](lblDccOut) at (1,0.2) {DCC-TTL};
        \node[right](lblDccOutACK) at (1,-0.2) {DCC-ACK};
        \node[right](lblDccOut5V) at (1,-0.6) {+5V};
        
        \draw[red] (dccOut-c0.center) -- (lblDccOut5V.west);
        \draw[orange] (dccOut-c1.center) -- (lblDccOutACK.west);
        \draw[blue] (dccOut-c2.center) -- (lblDccOut.west);
        \draw[black] (dccOut-c3.center) -- (lblDccOutGnd.west);


        \draw (dccIface-east-out.center) -- (dccOut-p2.center);
        \draw (dccIface-east-in.center) -- (dccOut-p1.center);
        \draw (dccIface-east.center |- dccOut-p0.center) -- (dccOut-p0.center);
    \end{scope}

    \draw[] (pt5V.center |- 0,1.5) -- +(-1.2,0) -- +(-1.2,-3.3) arc(90:-90:.05)
    -- +(0,-.2)  arc(90:-90:.05) -- +(0,-.25);

    \node[red, scale=.75]at(8.4,1.5){X};
    \node[red, scale=.75]at(2.1,0){X};
    \node[red, scale=.75, rotate=90]at(4,1.5){X};
    
\end{tikzpicture}
    \caption{Diseño por Diagrama de Bloques}
    \label{fig:BlockDiagrama}
\end{figure}
Mediante jumpers y soldaduras es posible configurar la placa para alterar su funcionamiento.
Las lineas discontinuas azules son jumpers para poder hacer derivaciones (por defecto abiertos) y
las cruzes rojas son pistas para romper con puntos de soldaura cercanos para unir soldando.

\subsubsection{Seleccion de corriente}
La señal \textbf{DCC} va directa la \textbf{interface DCC} y a un puente rectificador
\textbf{AC -> CC} para poder ser usado como fuente de corriente, pero antes debe pasar
por el bloque \textbf{Proteccion}. 

Pero la prioridad es utilizar la corriente que venga de un adaptador Corriente Continua (CC), 
ya sea por el \textbf{Jack CC} o por el \textbf{Bloque CC} atornillable. 
Por lo que si hay corriente de una adaptador CC, el bloque \textbf{Proteccion} bloqueara la corriente
proveniente de la señal DCC y la linea \textbf{VDrive} sera alimentada desde el adaptador
pasando por diodo de proteccion en caso de tener corriente DCC y CC a la vez.

Es posbile evitar la proteccion con el jumper de bypass (lineas azules cerca del bloque) y
con los cortes (cruces rojas) cerca del retificador. Mirar el apartado de configuración para
más informacion.

Los dos conectores, \textbf{Jack CC} y \textbf{Bloque CC} atornillable, estan conectados
en paralelo directo, por lo que se debe tener cuidado en caso de conectar los dos.

\subsubsection{Distribucion de corriente}
Este modulo genera tres carriles de potencia, \textbf{VDrive}, \textbf{+5V} y \textbf{+3.3V}.
VDrive es tal cual la fuente que se ha selecionado y el resto se generan a partir de esta.

Los tres carriles acaban en un bloque terminal atornillable para conectar a otros modulos.
Junto al cual se haya un led para visualizar que existe dicho carril. Este led puede ser
inhibido, cortando una pista. Mirar el apartado de configuración para
más informacion.

El carril de 5 Voltios se crea mediante un circuito \textbf{Buck StepDown} generado con la herramienta
de diseño "WEBENCH Power Designer", siendo este más o menos el circuito de referencia del chip usado.

Asi mismo el carril de 3.3 Voltios se genera con un regulador lineal siguiendo el cortocircuito
de referencia del mismo.

\subsubsection{Interface DCC}
La interface DCC se basa en el diseño de referencia con dos optoaisladores, de tal forma que la
señal DCC se convierte en TTL y la linea ACK activa una carga en la red DCC.

El conector DCC incluye los siguientes puntos:
\begin{itemize}
    \item \textbf{+5V}: linea para TTL Alto, se utiliza de vuelta en el pin DCC-TTL. Puede ser configurada
para usar el rail \textbf{+5V}.
    \item \textbf{DCC-ACK}: un 1 logico aqui provacara que se active una carga en la red DCC
notificando un OK a la central (como una programacion correcta).
    \item \textbf{DCC-TTL}: Linea que repite la señal DCC en un valor valido TTL, usando como valor
la señal +5V del conector.
    \item \textbf{GND}: Nivel base GND para unir a otros modulos.
\end{itemize} 

\subsection{Limites}
DccPowerDistribution soporta varios Voltajes y corrientes maximas segun cual sea el conector que
se use y como se configuren los bypass. Los componentes se han elegido pensando en un consumo maximo
continuado de \textbf{1A}. Por especificaciones los elementos pueden soportar de manera continudada al
menos un 50\% más. 

La tablas siguente recogen los Voltajes aceptados recomendados, maximo sin bypass y maximo si activamos
el bypass para evitar la proteccion.

\begin{table}[H]
    \centering
    \renewcommand\theadfont{\bfseries}
    \setlength{\tabcolsep}{10pt}
    \renewcommand{\arraystretch}{1.5}
    \begin{tabular}{c |c |c |c |c |}
        entrada & \thead[b]{item} & \thead[b]{Recomendado} & \thead[b]{Maximo} & \thead[b]{Con Bypass} \\ 
        \Xhline{5\arrayrulewidth}
%Entrada DCC
        \rowcolor{Melon!15}
        & Voltaje &14-20V & \multicolumn{2}{c|}{12-24V} \\
        \cline{2 - 5}
        \rowcolor{Melon!10} \cellcolor{Melon!15}
        \multirow{-2}{*}{DCC}&Corriente & 1A & 1.5A & 2A \\ \Xhline{3\arrayrulewidth}
%Entrada Jack
        \rowcolor{blue!15} & Voltaje & 12-20V & \multicolumn{2}{c|}{10-24V} \\
        \cline{2 - 5}
        \rowcolor{blue!10} \cellcolor{blue!15} \multirow{-2}{*}{ \makecell{ \cellcolor{blue!15} Jack\\ \cellcolor{blue!15} Terminal}} & Corriente & 1A & 1.5A & 3A \\
        \Xhline{5\arrayrulewidth}
    \end{tabular}
    \caption{Limites de entrada}
    \label{tab:limiteEntrada}
\end{table}

El Jack y el Terminal atornillable, estan en paralelo por lo que comparten limites. Y es por eso que solo
uno puede ser usado como entrada.

\begin{table}[H]
    \centering
    \renewcommand\theadfont{\bfseries}
    \setlength{\tabcolsep}{10pt}
    \renewcommand{\arraystretch}{1.5}
    \begin{tabular}{c |c |c |c |c |}
        Salidas & \thead[b]{item} & \thead[b]{Recomendado} & \thead[b]{Maximo} & \thead[b]{Con Bypass} \\ 
        \Xhline{5\arrayrulewidth}
%Salida Vdrive        
        \rowcolor{BlueGreen!15}& Voltaje & \multicolumn{3}{c|}{Mismo que la entrada - 1.0V} \\
        \cline{2-5}
        \rowcolor{BlueGreen!10} \cellcolor{BlueGreen!15}
        \multirow{-2}{*}{Vdrive}&Corriente* & 1A & 1.5A & 3A \\ \Xhline{3\arrayrulewidth}
%Salida +5V        
        \rowcolor{red!15}& Voltaje & \multicolumn{3}{c|}{5V +/- 2\% con ruido} \\
        \cline{2-5}
        \rowcolor{red!10} \cellcolor{red!15}
        \multirow{-2}{*}{+5V}&Corriente* & 1A & 1.5A & 3A \\ \Xhline{3\arrayrulewidth}
%Salida +3.3V                
        \rowcolor{Goldenrod!15}& Voltaje & \multicolumn{3}{c|}{3.3V +/- 0.6\%} \\
        \cline{2-5}
        \rowcolor{Goldenrod!10} \cellcolor{Goldenrod!15}
        \multirow{-2}{*}{+3.3V}&Corriente* & 0.5A & 1A & 1A \\ \Xhline{3\arrayrulewidth}
   \end{tabular}
   \caption{Limites de salida}
   \label{tab:limiteSalida}
   \textbf{*}:La suma de la corriente de salidas no puede exceder el limite de corriente de entrada
\end{table}

DccPowerDistribution ha sido diseñado y probado para poder utilizar hasta 1A en total y lo componentes
escogidos para que segun sus especificaciones puedan soportar al menos 1.5A