% !TeX encoding = UTF-8
% !TeX spellcheck = es_ES
% !TeX root = DccPowerDistribution.tex

\documentclass[spanish]{DccDiyTools/DccDiyTools}
\usepackage[spanish]{babel}
\usepackage[
type={CC},
modifier={by-sa},
version={4.0},
]{doclicense}
\usepackage{DccDiyTools/DccDiyToolsPics}

\title{Dcc Power Distribution}
\subtitle{Manual de Usuario}
\author{Daniel Vilas}
\date{Julio 2022}

\dbHeaderTitle{Dcc Power Distribution}
\dbType{M}
\dbDate{22}
\dbCode{000}
\dbStatus{Draft}
\dbVersion{0.1}
\dbImage{images/front.png}

\tikzset{
    pics/WallAc/.style={
      code = {
    \begin{scope}[shift={(0.25,0)}]
        % \draw [step=0.1,very thin, yellow] (-2,-2) grid (2,2);
        % \draw [step=0.5,very thin, red] (-2,-2) grid (2,2);
        % \draw [very thin, green] (-2,-2) grid (2,2);
        % Mains lead
        \draw[line width=2pt,cap=round, lightgray] (0.05,-0.21) -- (0.05,-0.45);
        \draw[line width=2pt,cap=round, lightgray] (0.275,-0.21) -- (0.275,-0.45);
        % Relief and out cable
        \draw[line width=2pt,cap=round] (-0.5,0.25) -- (-1,0.25);
        \draw[line width=2pt,cap=round,black!75] (-0.55,0.35)--(-0.55,0.15);
        \draw[line width=2pt,cap=round,black!75] (-0.63,0.325)--(-0.63,0.175);
        \draw[line width=2pt,cap=round,black!75] (-0.71,0.3)--(-0.71,0.2);
        \draw[line width=2pt,cap=round,black!75] (-0.79,0.275)--(-0.79,0.225);
        \draw[line width=2pt,cap=round,black!75] (-0.87,0.275)--(-0.87,0.225);

        \draw[line width=2pt,rounded corners=1pt,fill=black!75] (-0.5,0) rectangle +(1,0.5);
        \draw[line width=2pt,rounded corners=1pt,fill=black!65] (-0.1,0) rectangle +(0.5,-0.2);
        % \draw[line width=2pt,cap=round, gray] (0.05,-0.21) -- (0.05,-0.45);
        % \draw[line width=2pt,cap=round,gray] (0.275,-0.21) -- (0.275,-0.45);

        \node[white,scale=0.5] at (0,0.25) {DC Wall};
        \node[](-jack) at (-1.1,0.25) {};
        \node[](-mains) at (.1625,-0.55) {};
    \end{scope}
    }}
}

\tikzset{
    pics/Rj11/.style={
      code = {
        %\draw [step=0.1,very thin, yellow] (-1,-1) grid (1,1);
        %\draw [step=0.5,very thin, red] (-1,-1) grid (1,1);
        %\draw [very thin, green] (-1,-1) grid (1,1);
        %
        \draw [fill=black!90](-0.4,-0.4) rectangle +(0.8,0.8);
        \draw [fill=black!60](-0.3,-0.3) -- (-.3,.2) 
          -- (-0.1,0.2) -- (-0.1,0.3) -- (.1,.3)
          -- (.1,.2) -- (.3,.2) -- (.3,-.3)
          -- (-0.3,-0.3)
        ;
        \foreach \x in {-.25,-.15,...,.26}{
          \draw[gray!50] (\x,-.29)--(\x,-.2);
        }
        
    }}
}

\tikzset{
    pics/HandHeld/.style={
      code = {
        \draw[fill=yellow!05] (-.5,-1.5) arc (270:180:.1)
        -- (-.6,.1) arc (0:90:.2) arc (270:180:.2)
        -- (-1,1.3) arc (180:90:.2) -- (0.8,1.5)
        arc(90:0:.2) -- (1,.5) arc (0:-90:.2)
        arc (90:180:.2) -- (0.6,-1.4) arc (0:-90:.1)
        -- (.5,-1.5) -- (-.5,-1.5);
        \draw [red, fill=red!75] (0,.2) circle(.4);
        \draw [red!65, fill=red!50] (0,.2) circle(.3);
        \draw[fill=black!60] (-0.6,0.75) rectangle (0.6,1.25);
        
        \foreach \y in {-0.4,-.6,...,-1.3}{
          \foreach \x in {-0.3,0,0.3}{
            \begin{scope}[shift={(\x,\y)}]
              \draw[fill=gray!50] (-.05,.05) arc(90:270:.05)
              -- (.05,-.05) arc (-90:90:.05) 
              -- (-.05,.05);
            \end{scope}
          }
        }

    }}
}

\begin{document}

\maketitle
\newpage

\section{Introduccion}
Dcc Power Distribution es un modulo "DCC DiY Tools" que recoje la señal DCC y la utiliza como fuente de alimentacion para otros modulos de una maqueta. 
El objetivo de este modulo es proporcinar una interface DCC para que el usuario de este modulo tenga la señal DCC y energia para sus propios modulos.

Los modulos "DCC DiY Tools" son una serie de "Herramientas DCC Hazlo tu Mismo", pensadas para la gente con conocimiento de las placas Arduino y similares
puedan desarrollar sus porpios modulos sin tener que preocuparse de las complejidades y de los problemas comunes. Cualquiera que tenga un sketch corriendo
sobre una placa blanca de prototipo y quiera moverla a su maqueta se puede beneficiar de este modulo y asi no depender de un ordenador. 

Este modulo de distribucion de energia viene de la necesidad de controlar una serie de desvios con servomotores para hacer un moviento lento y para una 
maqueta no siempre tiene conectado un bus de +12V conectado. Pero no esta limitado a esto. Cualquier modulo que neceiste recibir alimentacion y señal DCC,
 en TTL, puede servirse de este dispositivo. Las caracteristicas generales de este modulo son:

\begin{itemize}
    \item \textbf{Multi Alimentacion}: Se puede alimentar por DCC o por un adaptador de Corriente Continua
    \item \textbf{Auto seleccion de Alimentacion}: En caso de conectar DCC y un adaptador, sin cambiar la configuracion se usara el adaptador.
    \item \textbf{Opciones de configuración}: Mediante Jumpers y pads soldables se puede lograr cierto grado de configuracion.
    \item \textbf{Multiples salidas} de voltaje:
    \begin{itemize}
        \item \textbf{VDrive}: lo mismo de de entrada, menos algunas perdidas por proteccion y auto-seleccion
        \item \textbf{5V}: obtenidos mediante un Buck Converter
        \item \textbf{3.3V}: Ajustados linealmente de los 5V
    \end{itemize}
    \item \textbf{Varias entradas} de voltaje:
    \begin{itemize}
        \item \textbf{DCC}: 12-20V y esta protegido ante los otros dos
        \item \textbf{Barrel Jack}: Centro positivo 12-20V, protegido con DCC y desprotegido con el terminal 
        \item \textbf{Terminal de Tornillos}: Conectado en paralelo con el jack.
    \end{itemize}
    \item Hasta \textbf{1A Corriente maxima}: por salida. Ver apartado de alimentacion.
    \item \textbf{Interface DCC opto-aislada}: incorporda y configurable.
    \item \textbf{Conectores standard}: y cambiables
    \begin{itemize}
        \item \textbf{JST XH} paso 2.54mm: DCC. 
        \item \textbf{806-KLDX-0202-A}: Conector Jack 2mm 
        \item \textbf{Terminal} paso 3mm: Terminales atornillables.
    \end{itemize} 
    \item \textbf{Open Software Hardware}: Este modulo se basa en diferentes diseños OSH y asi mismo se publica como OSH.
\end{itemize}

\subsection{Objetivos}
DccPowerDistribution es modulo cuyo objetivo principal dotar de alimentacion para otros modulos DiY
para facilitar el desarrollo de otros modulos por parte del usuario. Los objetivos del modulo son:
\begin{itemize}
    \item \textbf{Facilitar la toma de energia} al usuario objetivo, que es una persona que sepa
de arduino y quiera mover su circuito a su maqueta.   
    \item \textbf{Funcionar solo con un bus DCC}: Debe ser capaz de recoger la corriente desde el bus
DCC si no hay un bus de Corriente Continua(CC) disponible. Como por ejemplo en maquetas de pruebas
    \item \textbf{Usar un Bus CC}: para no sobrecargar el bus DCC sin ser necesario.
    \item \textbf{Ser robusto ante caidas DCC}: Ante un cortocircuito las centrales y los boosters 
paran de suministrar corriente DCC, por lo que el bus CC debe ser prioritario.
    \item \textbf{Facilitar el uso de DCC}: Para que el usuario tenga una señal DCC en TTL y usarla
con su placa.
\end{itemize}

\newpage
\section{Guia Instalación}
\input{quickStart.tex}

\newpage
\section{Mantenimiento Periodico}
% !TeX encoding = UTF-8
% !TeX spellcheck = es_ES
% !TeX root = DccPowerDistribution.tex
%!TEX root= DccPowerDistribution.tex

Una vez instalado y al menos una vez al trimestre conviene realizar una serie de comprobaciones.

\subsection{Antes de montar en la maqueta}
Una vez soldados los componentes, nos aseguraremos que no hay cortos y funciona como se espera.

Se recomienda hacer estas comprobaciones antes de realizar los cortes/cortos en los Jumpers que correspondan a la configuracion deseada. Y repetir las pruebas que correspondan una vez configurada la placa.

\begin{itemize}
	\item Apagado y con un tester de continuidad asegurase que de no hay corto circuitos entre los siguientes puntos:

\begin{itemize}
\item GND con VIN(+), VDrive(+), +5V(+), +3.3V(+), DCC, Ack y +5V(arduino). Probar secuencialmente, GND puede obtenerse de los agujeros de montaje o de un Vx(-).
\item  VIN(+), VDrive(+), +5V(+), +3.3V(+), DCC, Ack y +5V(arduino). Entre ellos.
\item Entre los pines del conector DCC
\end{itemize}

\item Connectar y alimentar la placa mediante el Jack DC y medir los voltajes de cada salida:
	\begin{itemize}
	\item \textbf{No realizar} si la placa esta configurada como DCC solo. 
	\item \textbf{VDrive}: Debe ser el voltaje del alimentador menos 0.7V aproximadamente, o la misma si esta configurada como CC solo.
	\item \textbf{+5V}: Debe estar cercano a 5V
	\item \textbf{+3.3V}: Igualmente debe ser cercano a 3.3V.
	\item Revisar que se enciendan los LEDs indicadores. Salvo que sus correspondientes Jumpers (JP5, JP6 y JP7) no esten cortados.
	\item Revisar temperatura de D3, U1, U2 y L1. Tocar un dedo, deben estar templados, si queman revisar pistas por cortos.
	\end{itemize}
\item Desconectar el Jack DC y conectar la señal DCC.
	\begin{itemize}

	\item \textbf{No realizar} si la placa esta configurada como CC solo. O si alguno de los  jumpers JP1, JP2, JP3 o JP4 estan cortados.

	\item Repetir las mediciones, \textbf{+5V} y \textbf{+3.3V} deben ser las mismas (o muy similares) a como estaban. La señal \textbf{VDrive} sera aproximadamente 1.5V menos que el adapador para DCC\sidenote{O si la central lo soporta, el voltaje configurado en esta}
	\item Revisar que se enciendan los LEDs indicadores. Salvo que sus correspondientes Jumpers (JP5, JP6 y JP7) no esten cortados.
\item Medir el voltaje entre Gate (TPx) y Source (TPx) de Q4, este debe aproximadamente VDrive. Tambien se puede medir entre Gate y GND, siendo en este caso practicamente 0.

\item Revisar temperatura de D2, Q4, U1, U2 y L1. Tocar un dedo, deben estar templados, si queman revisar pistas por cortos.

	\end{itemize}
\item Conectar ahora el Jack DC y la señal DCC
	\begin{itemize}
		\item \textbf{Solo} si la placa esta configurada como "<auto"> y si los jumpers JP1, JP2, JP3 y JP4 estan conectados

\item Repetir las mediciones, \textbf{+5V} y \textbf{+3.3V} deben ser las mismas (o muy similares) a como estaban. La señal \textbf{VDrive} sera aproximadamente 0.7V menos que el adaptador CC

\item Revisar que se enciendan los LEDs indicadores. Salvo que sus correspondientes Jumpers (JP5, JP6 y JP7) no esten cortados.
\item Medir el voltaje entre Gate (TPx) y Source (TPx) de Q4, este debe aproximadamente 0V. Tambien se puede medir entre Gate y GND, siendo en este caso practicamente VDcc.

\item Revisar temperatura de D3,D2, Q4, U1, U2 y L1. Tocar un dedo, deben estar templados, si queman revisar pistas por cortos.

	\end{itemize}

\item Conectar un modulo DCC al conector y la entrada DCC a la señal de via.
\begin{itemize}
\item  Para que funcione es necesario que JP1 y JP2 esten conectados. Si se quiere probar ACK, es necesario tener JP3 y JP4 conectados igualmente.
\item Se puede usar un arduino con la libreria DCC y el ejemplo de Turnout. Se recomienda "<modificarlo">\sidenote{Habilitar, que ya esta en el ejemplo} para que muestre log DCC e informacion de programacion en la salida UART del mismo.
\item \textbf{Si JP8} esta conectado, el decoder recibira alimentacion mediante conector Salida Dcc.
\item Enviar comandos DCC y ver que el decoder actua en consecuencia. Si se usa el ejemplo que muestra la informacion en la terminal.
\item Cambiar la entrada DCC a la via de programacion y enviar un comando de lectura de CV9 y ver que en la consola del arduino se muestra informacion coherente 

\end{itemize}

\end{itemize}
\subsection{Despues de Montar en la maqueta}
Cuando tengamos la placa DccPowerDistribution instalada en la maqueta se nos habra complicado
el poder realizar unas pruebas exhaustivas, pero en general se intentaran realizar todas las posibles.

Periodicamente se deberan realizar los siguientes pasos:
\begin{itemize}
\item \textbf{Limpieza}: Quitar el polvo con un trapo, pasandolo suavenmente sobre la placa o con un aspirador adecuado para la maqueta.
Si hay suciedad resistente se puede humedecer ligeramente el trapo con algun limpiador apto para
componentes electronicos. La limpieza se debe hacer \textbf{sin estar alimentada}

\item \textbf{Observar}: Por si hay alguna descoloracion aparante. Por el efecto de calor y/o del tiempo puede llegar apreciarse cambios
en el color de los componentes. Si se aprecia un cambio significativo desde la ultimo mantenimiento,
hay que rehacer todo el proceso de pruebas, para asegurar que sigue funcionado correctamente siendo
lo más exahaustivo posible\sidenote{Desmontala si es preciso}. 

Si \textit{solo es un ligero oscurecimiento hacia el amarillo}, puede darse en los primeros usos y
luego quedarse fijo, siendo no más una cuestion estetica. Pero hay que controlar que sea estable y
no afecte a las pistas a la placa misma. En caso de duda, cambiarla por una placa nueva\sidenote{Si
se repite una vez más, revisa que no se la sobrecarga}

Cualquier otro defecto visual (pistas quemadas, componentes rotos,\dots) quitar la placa y cambiarla por una nueva.
\item \textbf{Test rapido}: Una vez que la placa esta limpia y sin defectos aparentes, conviene
realizar toda la bateria de pruebas. Pero debido a que ya esta instalada, puede no ser posible asi
que como minimo hay que combrobar:
\begin{itemize}
\item Que no haya corto en la entrada CC ni DCC. Con un multimetro comprobar que no hay continuidad entre los dos pines de DCC (entre si) ni en los de CC (entre si)
\item Que no haya corto entre GND y las salidas Vdrive, +5V y 3.3V
\end{itemize}

\end{itemize}



\newpage
\section{Caracteristicas Tecnicas}
% !TeX encoding = UTF-8
% !TeX spellcheck = es_ES
% !TeX root = DccPowerDistribution.tex


\subsection{Diagrama de Bloques}
\begin{figure}[H]
    \centering
    \begin{tikzpicture}
    %\draw [very thin, green] (-1,-3) grid (12,4);
    \begin{scope}[shift={(0,2)}]
        \node [left] (txtCcB) at (0, 1){CC Bloque};
        \pic(picCcB)[rotate=-90] at (1,1)  {screwTerminal} ;
        \node [left](txtCcJ) at (0, 0){CC Jack};
        \pic (picCcJ) at (1,0)  {jack} ;

        \draw[] (txtCcB.east) -- (picCcB-cable-down.center);
        \draw[] (txtCcJ.east) -- (picCcJ-cable.center);
        \node[draw](proteccion) at (4,0.5) {Protección};

        \draw[] (picCcB-cable-up.center) -- (2.25,1) 
            -- (2.25, 0.5) -- (proteccion.west);
        \draw[] (picCcJ-point.center) -- (2.25,0)
            -- (2.25, 0.5);
        \node[](ptVDrive)  at (proteccion.east-| 6,0) {};
        \node[above](lblVdrive) at (ptVDrive.center)  {VDrive};
        \draw (proteccion.east) -- (ptVDrive.center);
        \draw (2.75, 0.5) -- (2.75,1.25) -- (5.25,1.25) 
            -- (5.25,0.5);
        \pic[] at (proteccion.center |- 0,1.25) {SmallDiode};
        \draw[blue, dashed] (3.5,1.25) -- (3.5,1.5) -- (4.5,1.5)
            -- (4.5,1.25);
        \draw[blue, dashed] (proteccion.center |- 0,0)
            -- (5.25,0) -- (5.25,.5);
    \end{scope}

    \begin{scope}
        \node [left] (txtDccIn) at (0, 0){DCC};
        \pic[rotate=-90] (picDccIn) at(1,0){JstXh2};
        \pic[](DccDiode) at (3,0){DiodeBrigde};

        \draw[] (txtDccIn.east) -- (picDccIn-cable.center);
        \draw[] (picDccIn-pcb.center) -- (DccDiode-ac.center);

        \draw[] (DccDiode-cc.center) -- (DccDiode-cc.center -| proteccion.south)
            -- (proteccion.south);
    \end{scope}

    \begin{scope}[shift={(ptVDrive.center |- 0,0)}]
        \pic[](picOutVdrive)  at (0,0){screwTerminal};
        \node[below](lblOutVdrive)  at (picOutVdrive-cable-down.center){Vdrive};
        \draw[] (ptVDrive.center) -- (picOutVdrive-cable-up.center);
        \pic[rotate=-90,yscale=-1](picLedVdrive)  at (-0.75,0){SmallLed=green};
        \draw (0,0.5) -- (picLedVdrive-p.center |- 0,0.5) -- (picLedVdrive-p.center);
        \node[red, scale=.75]at(-0.4,0.5){X};
    \end{scope}

    \begin{scope}[shift={(ptVDrive.center -| 7.5,0)}]
        \node[draw](buck) at (0,0) {Buck}; 
        
        \node[](pt5V)  at (1.5,0) {};
        \node[above](lbl5V) at (pt5V.center)  {+5V};
        \draw (buck.east) -- (pt5V.center);
        \draw (ptVDrive.center) -- (buck.west);
    \end{scope}

    \begin{scope}[shift={(pt5V.center |- 0,0)}]
        \pic[](picOut5V)  at (0,0){screwTerminal};
        \node[below](lblOut5)  at (picOut5V-cable-down.center){+5V};
        \draw[] (pt5V.center) -- (picOut5V-cable-up.center);
        \pic[rotate=-90,yscale=-1](picLed5V)  at (-0.75,0){SmallLed=green};
        \draw (0,0.5) -- (picLed5V-p.center |- 0,0.5) -- (picLed5V-p.center);
        \node[red, scale=.75]at(-0.4,0.5){X};
    \end{scope}


    \begin{scope}[shift={(pt5V.center -| 10.5,0)}]
        \node[draw](ldo) at (0,0) {LDO};  
        \node[](pt33V)  at (1.5,0) {};
        \node[above](lbl33V) at (pt33V.center)  {+3.3V};
        \draw (ldo.east) -- (pt33V.center);
        \draw (pt5V.center) -- (ldo.west);
    \end{scope}

    \begin{scope}[shift={(pt33V.center |- 0,0)}]
        \pic[](picOut33V)  at (0,0){screwTerminal};
        \node[below](lblOut33)  at (picOut33V-cable-down.center){+3.3V};
        \draw[] (pt33V.center) -- (picOut33V-cable-up.center);
        \pic[rotate=-90,yscale=-1](picLed33V)  at (-0.75,0){SmallLed=green};
        \draw (0,0.5) -- (picLed33V-p.center |- 0,0.5) -- (picLed33V-p.center);
        \node[red, scale=.75]at(-0.4,0.5){X};
    \end{scope}

    \begin{scope}[shift={(lblVdrive.center |- 3,-2)}]
        \pic[](dccIface)at (0,0){DccIface};
        
    \end{scope}
    \draw[] (picDccIn-pcb.center -| 1.9,0) -- (1.9,0 |- dccIface-west-out.center) 
        -- (dccIface-west-out.center);
        \draw[] (4,0) -- (4,0 |- dccIface-west-in.center) 
        -- (dccIface-west-in.center);


    \begin{scope}[shift={(lbl5V.center |- dccIface-west.center)}]
        \pic[rotate=90](dccOut)at (0,0){JstXh4};

        \node[right](lblDccOutGnd) at (1,0.6) {GND};
        \node[right](lblDccOut) at (1,0.2) {DCC-TTL};
        \node[right](lblDccOutACK) at (1,-0.2) {DCC-ACK};
        \node[right](lblDccOut5V) at (1,-0.6) {+5V};
        
        \draw[red] (dccOut-c0.center) -- (lblDccOut5V.west);
        \draw[orange] (dccOut-c1.center) -- (lblDccOutACK.west);
        \draw[blue] (dccOut-c2.center) -- (lblDccOut.west);
        \draw[black] (dccOut-c3.center) -- (lblDccOutGnd.west);


        \draw (dccIface-east-out.center) -- (dccOut-p2.center);
        \draw (dccIface-east-in.center) -- (dccOut-p1.center);
        \draw (dccIface-east.center |- dccOut-p0.center) -- (dccOut-p0.center);
    \end{scope}

    \draw[] (pt5V.center |- 0,1.5) -- +(-1.2,0) -- +(-1.2,-3.3) arc(90:-90:.05)
    -- +(0,-.2)  arc(90:-90:.05) -- +(0,-.25);

    \node[red, scale=.75]at(8.4,1.5){X};
    \node[red, scale=.75]at(2.1,0){X};
    \node[red, scale=.75, rotate=90]at(4,1.5){X};
    
\end{tikzpicture}
    \caption{Conectores Dcc}
    \label{fig:BlockDiagrama}
\end{figure}



\subsection{Limites de entrada}
\begin{center}
    \begin{tabular}{ c c c }
        \multicolumn{3}{c}{Entrada DCC} \\ \hline
      & Seguro & Maximo \\ 
     Voltaje & 14-20V & 12-24V \\  
     Corriente & 1A & 1.5A \\
     [1ex] 
     \multicolumn{3}{c}{Entrada Jack y Terminal} \\ \hline
     & Seguro & Maximo \\ 
     Voltaje & 12-20V & 10-24V \\  
     Corriente & 1.5A & 2A \\ \hline
    
    \end{tabular}
\end{center}

\newpage
\section{Configuracion}
% !TeX encoding = UTF-8
% !TeX spellcheck = es_ES
% !TeX root = ./DccPowerDistribution.tex
%!TEX root = ./DccPowerDistribution.tex


La placa DccPowerDistribution tiene una serie de jumpers y puntos de soldaura/corte
que permiten cambiar ciertas partes y ajustar su comportamiento.

\subsection{Jack o Terminal Atornillable}

El conector Jack se puede conectar directamente a un adaptador AC/DC que tenga salida 
jack 2mm, un conector muy generico usado por diferentes dispositivos electronicos.
Por norma general, solo tienen una salida, por lo que solo se puede conectar a una placa

Por otra parte el terminal Atornillable facilita la conexion cuando ya existe un bus de
corriente continua CC. Con lo que es mas facil conectar varias placas.

Ambos estan conectados en paralelo sin ninguna proteccion, por lo que es necesario
asegurarse que solo uno recibe corriente. La forma más sencilla es solo soldar uno 
de ellos o conectar solo uno.

\begin{figure}[H]
    \centering
    \begin{minipage}{0.25\textwidth}
        \centering
        \begin{tikzpicture}
    
    
    \pic[rotate=-90,transform shape](ac) at(-1,0) {WallAc};
    \pic[rotate=-90,transform shape](board)at (1,0) {SmallBoard};

    \draw[red, line width=4pt,rounded corners=6pt] (ac-jack.center) 
    -- (ac-jack.center |- 0,1.8) 
    -- (board-jack.center|- 0,1.8) 
    -- (board-jack.center);
    %\draw[green] (ac-mains.center) -- (board-terminal.center);

\end{tikzpicture} 
        \caption{Solo Jack}
        \label{fig:VccConnectionJack}
    \end{minipage}
    \hfill
    \begin{minipage}{0.7\textwidth}
        \centering
        \input{tikz/power_terminal.tex}
        \caption{Solo Terminal}
        \label{fig:VccConnectionTerminal}
    \end{minipage}
\end{figure}

Si se sueldan los dos conectores se puede usar una placa como iniciador de bus CC

\begin{figure}[H]
    \centering
    \begin{tikzpicture}
    
    
    \pic[rotate=-90,transform shape](ac) at(-3,0.5) {WallAc};
    \pic[rotate=-90,transform shape](board1)at (-1,0.5) {SmallBoard};

    \draw[red, line width=4pt,rounded corners=6pt] (ac-jack.center) 
    -- (ac-jack.center |- 0,2.8) 
    -- (board1-jack.center|- 0,2.8) 
    -- (board1-jack.center);

    \pic[rotate=-90,transform shape](board2)at (1,-0.5) {SmallBoard};
    \pic[rotate=-90,transform shape](board3)at (3,-0.5) {SmallBoard};
    
    \begin{scope}[shift={(5,-0.5)}]
        \draw[black] (-0.7,-1.3) rectangle +(1.4,2.6);
        \node[] at (0,0.3) {Otro};
        \node[] at (0,-0.3) {Modulo};
    \end{scope}
    
    \draw[red!75, line width=2pt,rounded corners=6pt] (board1-terminal.center -| 2,1.8)
    --(board2-terminal.center -| 2,0)
    --(board2-terminal.center);
    
    \draw[red!75, line width=2pt,rounded corners=6pt] (board1-terminal.center -| 4,1.8)
    --(board3-terminal.center -| 4,0)
    --(board3-terminal.center);

    \draw[red!75, line width=2pt,rounded corners=6pt] (board1-terminal.center -| 6,1.8)
    --(board2-terminal.center -| 6,0)
    --(board2-terminal.center -| 5.7,0);


    % Bus
    \draw[red, line width=4pt,rounded corners=6pt] (board1-terminal.center)
        -- (board1-terminal.center -| 6.5,1.8);
    %\draw[green] (ac-mains.center) -- (board-terminal.center);

\end{tikzpicture} 
    \caption{Jack Iniciando Bus}
    \label{fig:VccConnectionJackBus}
\end{figure}

En el caso de que se use un bus de Coriente Continua, es posible añadir modulos que no sean
de la gama DccDiyTools, pero es necesario asegurarse de que sean compatibles a nivel de voltaje
y que el adaptador de corriente sea capaz de suministrar la corriente necesaria para todos los
modulos conectados.

\subsection{Origen Automatico, siempre DCC o siempre CC}
La corriente puede ser tomada desde el bus DCC o un bus de Corriente Continua (CC).
Esto se puede hacer mediante la activacion de los ByPass, o lineas azules en el diagrama
de bloques. El selector principal que determina el origen es J3 y tiene tres opciones

\begin{figure}[H]
    \centering
    \begin{tikzpicture}
    \begin{scope}[shift={(-4,0)}]

        \begin{scope}
            \clip (-1,-2) rectangle  +(2,4);

            \node[inner sep=0pt] at (5.6,-0.)
                {\includegraphics[scale=1.3]{images/front.png}};
        \end{scope}
        \draw[yellow,line width=2pt,rounded corners=4pt] 
            (-0.3,-0.7) rectangle +(0.6,1.4);
        \node[below] at (0,-2) {(a) Automatico};
    \end{scope}

    \begin{scope}
        % \draw [step=0.1,very thin, yellow] (-2,-2) grid (2,2);
        % \draw [step=0.5,very thin, red] (-2,-2) grid (2,2);
        % \draw [very thin, green] (-2,-2) grid (2,2);

        \begin{scope}
            \clip (-1,-2) rectangle  +(2,4);

            \node[inner sep=0pt] at (5.6,-0.)
                {\includegraphics[scale=1.3]{images/front.png}};
        \end{scope}
        %\draw [step=0.1,very thin, yellow] (-0.5,-1) grid +(1,2);
        \draw[yellow,line width=2pt,rounded corners=4pt] 
            (-0.3,-0.7) rectangle +(0.6,1.4);
        \draw[blue, fill=blue!75,line width=1pt]
            (-0.2,0.4) rectangle +(0.35,-.65); 
        \node[below] at (0,-2) {(b) Corriente Continua};
    \end{scope}

    \begin{scope}[shift={(4,0)}]

        \begin{scope}
            \clip (-1,-2) rectangle  +(2,4);

            \node[inner sep=0pt] at (5.6,-0.)
                {\includegraphics[scale=1.3]{images/front.png}};
        \end{scope}
        %\draw [step=0.1,very thin, yellow] (-0.5,-1) grid +(1,2);
        
        \draw[yellow,line width=2pt,rounded corners=4pt] 
            (-0.3,-0.7) rectangle +(0.6,1.4);

        \draw[blue, fill=blue!75,line width=1pt]
        (-0.2,0.05) rectangle +(0.35,-.65); 
        \node[below] at (0,-2) {(c) DCC};
    \end{scope}
\end{tikzpicture}
    \caption{Opciones J3 - Origen de corriente}
    \label{fig:VccSelection}
\end{figure}

J3 por defecto vendra sin componente, para que se pueda soldar un cable en la configuracion desada
y ademas pueda servir como puntos  de prueba. Puede ponerse una cabezera dupont 2.54mm y asi usar 
jumpers.

\begin{itemize}
    \item \textbf{Automatico}: Dejar J3 sin conectar nada.
    \item \textbf{Siempre CC}: Pone un jumer entre los pines más cercanos al Jack CC.
    \begin{itemize}
        \item En esta configuracion se recomienda desactivar la señal DCC, no conectandola
 o mediante \textbf{JP1, JP2, JP3 y JP4}.
        \item Con esta configuracion se puede lograr un maximo de \textbf{3A}. Pero simpre y
cuando se suelde un cable entre los puntos \sidenote[][-1em]{Hay jumpers de \textbf{3A}, en ese caso se pueden usar}.             
    \end{itemize}
     \item \textbf{Siempre DCC}:Pone un jumer entre los pines más cercanos al conector DCC.
     \begin{itemize}
         \item En esta configuracion se recomienda no conectar corriente CC
         \item Con esta configuracion se puede lograr un maximo de \textbf{2A}. Pero simpre y
 cuando se suelde un cable entre los puntos.             
     \end{itemize}
\end{itemize}

\subsection{Desactivacion Entrada de corriente DCC}
Es posible configurar DccPowerDistribution para que sea imposible utilizar la señal DCC como
entrada de corriente, y no perder la señal DCC-TTL. Para ello se pueden utilizar los puntos
de soldadura \textbf{JP1, JP2, JP3} y \textbf{JP4}.


\begin{figure}[H]
    \centering
    \begin{tikzpicture}
    \begin{scope}[shift={(-2,0)}]
        % \draw [step=0.1,very thin, yellow] (-2,-2) grid (2,2);
        % \draw [step=0.5,very thin, red] (-2,-2) grid (2,2);
        % \draw [very thin, green] (-2,-2) grid (2,2);
        \begin{scope}
            \clip (-1.8,-1.5) rectangle  +(3.6,3);
            \node[inner sep=0pt] at (4.7,1.5)
                {\includegraphics[scale=1.3]{images/front.png}};
        \end{scope}
        \draw[yellow,line width=2pt,rounded corners=4pt] 
            (-0.2,-0.1) rectangle +(1,0.95);
        
            \node[below] at (0,-2) {(a) JP1 y JP2};

        %\draw[white] (-2,0)--(2,0);
    \end{scope}
    \begin{scope}[shift={(3,0)}]
        % \draw [step=0.1,very thin, yellow] (-2,-2) grid (2,2);
        % \draw [step=0.5,very thin, red] (-2,-2) grid (2,2);
        % \draw [very thin, green] (-2,-2) grid (2,2);
        \begin{scope}
            \clip (-1.8,-1.5) rectangle  +(3.6,3);
            \node[inner sep=0pt] at (4.7,1.5)
                {\includegraphics[scale=1.3]{images/front.png}};
        \end{scope}
        \draw[yellow,line width=2pt,rounded corners=4pt] 
            (0.25,0.7) rectangle +(1.5,0.55);
        
            \node[below] at (0,-2) {(b) JP3 y JP4};

        %\draw[white] (-2,0)--(2,0);
    \end{scope}
\end{tikzpicture}
    \caption{Ubicacion DCC Disable}
    \label{fig:DccDisable}
\end{figure}

Como se puede ver hay dos juegos de puntos de soldadura, JP1 con JP2 y JP3 con JP4. Cada juego
realiza una funcion diferente.
\begin{itemize}
    \item \textbf{JP1} y \textbf{JP2}: conectan la señal DCC al bloque rectificador. Si se cortan
se pierde la funcionalidad ACK y la entrada de corriente por DCC. Se ha detectado que las centrales
DCC pueden confundirse con el bloque rectificador\sidenote[][-3em]{Tiene que ver con el tiempo de recuparacion
de los diodos} y suponer que hay un corto\sidenote[][1em]{Se han utilizado rectificadores 
rapidos, por lo que no se espera que suceda esto}. En estas centrales
es necesario cortar estos puntos de soldadura.
    \item \textbf{JP3} y \textbf{JP4}: conectan la señal rectificada DCC al bloque de proteccion.
    Se deben cortar ambos para desactivar la corriente DCC y aun mantener la funcionalidad ACK
        
\end{itemize}

\subsection{DCC +5V}
Para que la salida DCC-TTL funcione se necesita una señal de referencia TTL. La interface DCC hace un
Pull-Up a este valor cuando no hay señal DCC y lo situara a 0v conforme la señal vaya recibiendose.
Este valor de referencia puede ser el mismo carrill 5V generado por DccPowerDistribution, o venir de
fuera mediante el pin DCC \textbf{+5V}.

\begin{figure}[H]
    \centering
    \begin{tikzpicture}
    \begin{scope}
        \clip (-2,-2) rectangle  + (4,4);
        %\draw [very thin, green, fill=yellow]  (-6,-4) rectangle (12,8);
        \node[inner sep=0pt] (russell) at (-8.1,0)
            {\includegraphics[scale=2]{images/front.png}};
        %\draw [very thin, green]  (-7,-3) grid (6,4);
    \end{scope}
    %\draw [very thin, green] (-3,-2) grid (3,2);
    \draw [color=yellow,line width=2pt](-1.6,-0.9) rectangle + (1.6,1.8);
    
    \node[right] at (4,1) {GND: Ref 0V};
    \node[right] at (4,0.33) {DCC-TTL: Salida Señal DCC};
    \node[right] at (4,-0.33) {ACK: Entrada Señal ACK};
    \node[right] at (4,-1) {+5V: Entrada/Salida Ref TTL};
    
    \draw [color=black, line width=3pt,cap=round] (2.0,0.75) -- (4,1);
    \draw [color=blue, line width=3pt,cap=round] (2,0.25) -- (4,0.33);
    \draw [color=orange, line width=3pt,cap=round] (2,-0.25) -- (4,-0.33);
    \draw [color=red, line width=3pt,cap=round] (2.0,-0.75) -- (4,-1);
\end{tikzpicture}
    \caption{Entrada/Salida Señal DCC}
    \label{fig:DccOut2}
\end{figure}

Se puede cambiar este comportamiento mediante el punto de soldadura \textbf{JP8}.

\begin{figure}[H]
    \centering
    \begin{tikzpicture}

    %\draw [very thin, green]  (-6,-3) grid (6,3);

    \begin{scope}[shift={(3.4,0)}]
        \pic[scale=2,transform shape] at (-5,0) {SmallBoard}; 
        \begin{scope}
            \clip (-1.5,-1.5) rectangle  +(3,3);
            \node[inner sep=0pt] at (-1,0.45)
                {\includegraphics[scale=2]{images/front.png}};
        \end{scope}
        \draw[green, line width=2pt,rounded corners=4pt](-1.6,-1.6) rectangle  +(3.2,3.2);

        \draw[yellow,line width=2pt,rounded corners=4pt] 
            (0.2,-0.25) rectangle +(1.4,0.55);
        %\draw[green] (-8.7,-1.6) rectangle+(10.6,3.2);
        \node(lupaDown) at (-1.5,-1.6){};
        \node(lupaUp) at (-1.5,1.6){};
    \end{scope}
    \draw[green,rounded corners=1pt] (-1.72,-0.48) rectangle+(0.75,0.75);
    \draw[green, line width=1pt, cap=round] (-1.7,-0.48) -- (lupaDown.center);
    \draw[green, line width=1pt, cap=round] (-1.7,0.27) -- (lupaUp.center);
    \draw[yellow,rounded corners=1pt] (-1.3,-0.18) rectangle+(0.35,0.15);
    
\end{tikzpicture}
    \caption{Ubicacion JP8}
    \label{fig:Jp8Loc}
\end{figure}

Si las conexiones estan cortadas el pin DCC +5V se comportara como entrada, requirendo que se suministre
el voltaje de referencia TTL para la salida DCC-TTL. En este caso se puede usar logica 3.3V

Por otra parte, si se unen mediante soldadura. La linea DCC +5V estara conectada al rail +5V generado
por el conversor Buck. Por lo que se puede usar para suministrar corriente a los modulos DCC.

\subsection{Leds de los Railes}
Los Leds junto a los terminales de salida se pueden desconectar para ahorrar unos 
miliamperios de comsumo. Estos leds son utiles para ver de un vistazo rapido el estado
de cada carril de potencia. Pero una vez instalado bajo una maqueta no se van ver siempre
por lo que puede ser interesante desactivarlos.

Cada cada carril tiene asociado un led, un color\sidenote[][]{Puede variar segun ediciones de
DccPowerDistribution}, un Punto de soldadura y una resistencia para limitar su consumo.
El diseño original se recoge en la siguiente tabla:

\begin{table}[H]
    \centering
    \renewcommand\theadfont{\bfseries}
    \setlength{\tabcolsep}{10pt}
    \renewcommand{\arraystretch}{1.5}
    \begin{tabular}{c |c |c |c |c |}
        Carril & \thead[b]{Punto} & \thead[b]{Color($V_{drop}$)} & \thead[b]{Resistencia} & \thead[b]{Consumo} \\ 
        \Xhline{5\arrayrulewidth}
%VDrive
        \rowcolor{Melon!15} Vdrive(12V)
        & JP6 &Blanco(3V) & 900$\Omega$ & 10mA \\
        \hline
        \rowcolor{blue!15} 5V
        & JP7 & Azul(3V) & 200$\Omega$ & 10mA \\
        \rowcolor{cyan!15} 3.3V
        & JP5 & Cyan(2.7V) & 100$\Omega$ & 5.5mA \\
        \hline
    \end{tabular}
    \caption{Diseño de los leds}
    \label{tab:leds}
\end{table}

Junto a cada terminal de salida esta el led, la resitencia y el punto de soldadura.

\begin{figure}[H]
    \centering
    \begin{tikzpicture}[scale=1]
    \begin{scope}
        \clip (-7.,0) rectangle  +(14,2.5);
        \node[inner sep=0pt] (russell) at (0.6,4.75)
            {\includegraphics[scale=2]{images/front.png}};
    \end{scope}
    \node [text width = 1.3cm] at (-5,-0.5){VDrive};
    \node [text width = 1.3cm] at (1,-0.5){+5V};
    \node [text width = 1.3cm] at (5.4,-0.5){+3.3V};


    \draw [color= yellow, line width=2pt,rounded corners=2pt] (-6.5,1.5) rectangle +(1.,0.9);
    \draw [color= yellow, line width=2pt,rounded corners=2pt] (-0.6,1.5) rectangle +(1.,0.9);
    \draw [color= yellow, line width=2pt,rounded corners=2pt] (3.8,1.5) rectangle +(1.,0.9);
   
\end{tikzpicture}
    \caption{Ubicacion JP5, JP6 and JP7}
    \label{fig:LedDisable}
\end{figure}

Para deshabilitar un led solo es necesario cortar el punto de soldadura  que corresponda.
Y volverlo a soldar para habilitarlo

\subsection{Cortar y recuperar un punto de soldadura}
Los puntos de soldadura son dos Pads cercanos que pueden estar unidos con una pista entre
ambos, sindo NC o Normally Closed, pero tambien puede estar sin conexion, llamandose NO, o
Normally Open. 

\begin{figure}[H]
    \centering
    \begin{tikzpicture}
    \begin{scope}
        \pic[] at (-1,0){JumperPadNC};
        \node[below] at (-1,-1){a) NC}; 
        \pic[] at (1,0){JumperPadNO};
        \node[below] at (1,-1){b) NO}; 
    \end{scope}
\end{tikzpicture}
    \caption{Tipos de Puntos de Soldadura}
    \label{fig:JpTipos}
\end{figure}

Un Punto de Soldadura NC puede deshabilitase cortando la pista que esta uniendo los Pads
con X-Acto, bisturi o similar\sidenote[][]{Como un cutter pequeño}. En estos momentos se
habra convertido en un NO. Para comprobarlo, se recomienda usar un tester en modo
continudad o diodo

\begin{figure}[H]
    \centering
    \begin{tikzpicture}
    \begin{scope}
        \pic[] (nc) at (-1.4,0){JumperPadNC};

        \node[left] at (-3,0.5){Cortar aqui};
        \draw[-stealth] (-3,0.5) -- (nc-cut1.west);

        \draw[red,line width=2pt, dashdotted](nc-cut1.center)--(nc-cut2.center);
        \pic[] at (1.4,0){JumperPadNO};
        \draw[-stealth] (-0.5,0) -- (0.5,0);
    \end{scope}
\end{tikzpicture}
    \caption{Deshabilitar NC}
    \label{fig:JpNc2No}
\end{figure}

Un punto de soldarua NO, o un NC cortado, puede habilitarse poniendo suficiente estaño como
para que se forme un puente entre ambos pads. La distancia es tan pequeña que no se necesita
mucho.

\begin{figure}[H]
    \centering
    \begin{tikzpicture}
    \begin{scope}
        \pic[] (nc) at (-2,0){JumperPadNO};


         \pic[] at (2,0){JumperPadSolder};
        \draw[-stealth] (-1.3,0.5) -- (1.3,0.5);
        \node[above] at (0,0.5) {Soldar};

        \draw[stealth-] (-1.3,-0.5) -- (1.3,-0.5);
        \node[above] at (0,-0.5) {Desoldar};

    \end{scope}
\end{tikzpicture}
    \caption{Habilitar y Deshabilitar NO}
    \label{fig:JpNo}
\end{figure}

Asi mismo, una vez soldado, se puede deshabilitar retirando estaño con mecha
de desoldar. Se recomienda comprobar con un multimetro en modo continudad.

\newpage
\section{Open Software HardWare}
% !TeX encoding = UTF-8
% !TeX spellcheck = es_ES
% !TeX root = DccPowerDistribution.tex

La gama de dispositivos DccDiyTools estan bajo una licencia OSHW, que se define como:

"<Hardware de Fuentes Abiertas (OSHW en inglés) es aquel hardware cuyo diseño se hace disponible
públicamente para que cualquier persona lo pueda estudiar, modificar, distribuir, materializar
y vender, tanto el original como otros objetos basados en ese diseño. Las fuentes del hardware
(entendidas como los ficheros fuente) habrán de estar disponibles en un formato apropiado para
poder realizar modificaciones sobre ellas. Idealmente, el hardware de fuentes abiertas utiliza
componentes y materiales de alta disponibilidad, procesos estandarizados, infraestructuras abiertas,
contenidos sin restricciones, y herramientas de fuentes abiertas de cara a maximizar la habilidad
de los individuos para materializar y usar el hardware. El hardware de fuentes abiertas da libertad
de controlar la tecnología y al mismo tiempo compartir conocimientos y estimular la comercialización
por medio del intercambio abierto de diseños.">
\href{https://www.oshwa.org/definition/spanish/}{Extracto de la licencia por OSHWA}

\subsection{Licencia Resumida}
La licencia completa se puede obtener de \href{https://www.oshwa.org/definition/spanish/}{Open Software
Hardware Association - (OSHWA)} pero en resumidas cuentas significa que:
\begin{itemize}
    \item Cualquier persona pueda fabricar, modificar, distribuir y usar esos objetos.
    \begin{itemize}
        \item Quien lo haga tiene la oblicacion de indicar que su producto no ha sido
manufacturado, vendido, garantizado o autorizado en cualquier forma por el diseñador original.
        \item Tampoco puede usar ninguna marca registrada del diseñador original.
    \end{itemize}
    \item Se tiene acceso a la documentacion libremente de por no más que un razonable coste de reproducción.
    \begin{itemize}
        \item La documentacion, se refiere principalmente a los documentos CAD necesarios para manufacturar
        el producto.
        \item Debe estar en un formato facilmente modificable
    \end{itemize}
    \item El software necesario\sidenote[][]{Firmware, software de control,...} debe ser libre o bien
    documentado para que cualquiera pueda crearlo.
    \item Cualquiera persona puede crear obras derivadas y redistirbuir (incluyendo venta) con la misma licencia
de la obra original
    \begin{itemize}
        \item No se deberia requerir el pago de licencias, regalias o similares por la venta.
        \item El diseñador original puede exigir la atribucion de autoria y restringir el nombre en las obras
derivadas.
    \end{itemize}
    \item Cualquiera significa Cualquiera. Sin discriminacion de raza, sexo,... ni discriminacion de ambito
de aplicacion. Nadie debe requerir de una licencia especial para disfrutar de los derechos aqui defenidos.
    \item No se puede restringir otros productos que se distribuyan junto al producto licenciado. 

\end{itemize}

En resumen "<Cualquier persona pueda fabricar, modificar, distribuir y usar esos objetos."> y el diseñador 
puede añadir alguna pequeña restriccion.

\subsection{Licencia DccPowerDistribution}
DccPowerDistribution se licencia mediante la definicion publicada por la \href{https://www.oshwa.org/definition/spanish/}{OSHWA}
con las siguientes consideraciones:
\begin{itemize}
    \item Toda la documentacion y ficheros se encuentra en repositorios publicos de github.
    \begin{itemize}
        \item \url{https://github.com/danielvilas/DigitalTrains/tree/master/DccBlocks/DccPowerDistribution/board}
        \item \url{https://github.com/danielvilas/DigitalTrainsDocs/tree/main/DccDiyTools/DccPowerDistribution} 
    \end{itemize}
    \item Esta licencia OSHW se aplica para la placa electronica DccPowerDistribution 
    \item Los documentos anexos, como manuales, read me y similares, estan bajo la licencia Creative Commons By Share-Alike 
    \ccbysa
    \item Para editar la placa es necesario usar KiCAD 6.0.6 o superior. La documentacion se encentra en \LaTeX
    \item Cualquier placa derivada debe cambiar el nombre lo suficiente como para que sepa que es una placa derivada
    \begin{itemize}
        \item Cambiar el numero de version \textbf{no} es suficiente
        \item Seria valido añadir un nombre corto ([Nombre]DccPowerDistribution o DccPowerDistribution[Nombre]) 
    \end{itemize}
    \item Cualquiera puede manufacturar esta placa y venderla, sin pago de licencias ni regalias al diseñador.
Pero considera hacer una donacion al diseñador, sobre todo en el caso de venta y teniendo un beneficio economico
considerable. 
\end{itemize}

\newpage
\section{Garantia y Consideraciones de Seguridad}
% !TeX encoding = UTF-8
% !TeX spellcheck = es_ES
% !TeX root = DccPowerDistribution.tex
%!TEX root= DccPowerDistribution.tex
\textit{DCC DiY Tools} no manufactura productos por lo que no proporciona ningun
tipo de garantia legal sobre ningun Dcc Power Distribution manufacturado tal como dice la Introduccion
de la licencia OSHW. Es responsabilidad de quien manufacture el modulo soportar
la garantia a los usuarios finales.

Tambien es responsabilidad de quien manufacture asegurarse que no hay fallos de diseño,
y en el caso de encontrarlo notificar a \textit{DCC DiY Tools} del mismo.

Por otra parte \textit{DCC DiY Tools} se asegurara de que:
\begin{itemize}
    \item Usando los materiales del BOM y los procedimiendos descritos, el modulo
    Dcc Power Distribution cumple las especificaciones.
    \item Dcc Power Distribution no es peligroso siguendo las indicaciones de seguridad
    \item No hay fallos de diseño conscientes en Dcc Power Distribution
    \item En la medida de lo posible\sidenote{Y que tenga sentido}, se siguen buenas practicas de diseño:
    \begin{itemize}
        \item Disipacion termica, tamaño de trazas,\dots
        \item Teniendo en cuenta EMI/EMC
        \item Con idea a poder Certificarlo FCC/CE
    \end{itemize}
    \item Dcc Power Distribution se ha probado siguiendo las especificaciones y un pequeño margen extra
\end{itemize} 

Los modulos Dcc Power Distribution se esperan ser utilizado con sentido comun y se
debe evitar realizar acciones que puedan deriviar en situaciones peligrosas:
\begin{itemize}
    \item No exponer Dcc Power Distribution a fuentes de calor que puedan estropear los dispositivo
    \item Las placas electronicas pueden disipar energia en forma de calor, no
    teneras cerca de materiales inflamables y no tocarlas cuando esten en funcionamiento
    \item Situarla en un lugar que permita el flujo de aire para que dispipe
    correctamente dicho calor
    \item No provocar cortocircuitos a las placas. No mojarlas ni ponerle material metalico.
    \item No sobrepasar los limites indicados en las especificaciones.
	\item Seguir las instrucciones de montaje aqui expuestas
\end{itemize}


% \begin{tip}{Test}
% Ejemplo de tip
% \end{tip}

\newpage
\section{Indice}
\tableofcontents
\listoffigures
\listoftables


\end{document}