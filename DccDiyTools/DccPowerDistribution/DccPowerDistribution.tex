% !TeX encoding = UTF-8
% !TeX spellcheck = es_ES
% !TeX root = DccPowerDistribution.tex

\documentclass{DccDiyTools}
\usepackage[spanish]{babel}


\title{Dcc Power Distribution}
\subtitle{Manual de Usuario}
\author{Daniel Vilas}
\date{Julio 2022}

\dbType{M}
\dbDate{22}
\dbCode{000}
\dbStatus{Draft}
\dbVersion{0.1}
\dbImage{images/front.png}

\begin{document}

\maketitle


\section{Introduccion}
Dcc Power Distribution es un modulo "DCC DiY Tools" que recoje la señal DCC y la utiliza como fuente de alimentacion para otros modulos de una maqueta. 
El objetivo de este modulo es proporcinar una interface DCC para que el usuario de este modulo tenga la señal DCC y energia para sus propios modulos.

Los modulos "DCC DiY Tools" son una serie de "Herramientas DCC Hazlo tu Mismo", pensadas para la gente con conocimiento de las placas Arduino y similares
puedan desarrollar sus porpios modulos sin tener que preocuparse de las complejidades y de los problemas comunes. Cualquiera que tenga un skecth corriendo
sobre una placa blanca de prototipo y quiera moverla a su maqueta se puede beneficiar de este modulo y asi no depender de un ordenador. 

Este modulo de distribucion de energia viene de la necesidad de controlar una serie de desvios con servomotores para hacer un moviento lento. Pero no
esta limitado a esto. Cualquier modulo que neceiste recibir alimentacion y señal DCC, en TTL, puede servirse de este dispositivo. Las caracteristicas de
este modulo son:

\begin{itemize}
    \item \textbf{Multi Alimentacion}: Se puede alimentar por DCC o por un adaptador de Corriente Continua
    \item \textbf{Auto seleccion de Alimentacion}: En caso de conectar DCC y un adaptador, sin cambiar la configuracion se usara el adaptador.
    \item \textbf{Opciones de configuración}: Mediante Jumpers y pads soldables se puede lograr cierto grado de configuracion.
    \item \textbf{Multiples salidas} de voltaje:
    \begin{itemize}
        \item \textbf{VDrive}: lo mismo de de entrada, menos algunas perdidas por proteccion y auto-seleccion
        \item \textbf{5V}: obtenidos mediante un Buck Converter
        \item \textbf{3.3V}: Ajustados linealmente de los 5V
    \end{itemize}
    \item \textbf{Varias entradas} de voltaje:
    \begin{itemize}
        \item \textbf{DCC}: 12-20V y esta protegido ante los otros dos
        \item \textbf{Barrel Jack}: Centro positivo 12-20V, protegido con DCC y desprotegido con el terminal 
        \item \textbf{Terminal de Tornillos}: Conectado en paralelo con el jack.
    \end{itemize}
    \item Hasta \textbf{1A Corriente maxima}: por salida. Ver apartado de alimentacion.
    \item \textbf{Interface DCC opto-aislada}: incorporda y configurable.
    \item \textbf{Conectores standard}: y cambiables
    \begin{itemize}
        \item \textbf{JST XH} paso 2.54mm: DCC. 
        \item \textbf{806-KLDX-0202-A}: Conector Jack 2mm 
        \item \textbf{Terminal} paso 3mm: Terminales atornillables.
    \end{itemize} 
    \item \textbf{Open Software Hardware}: Este modulo se basa en diferentes diseños OSH y asi mismo se publica como OSH.
\end{itemize}


\section{Guia Rapida}
\input{quickStart.tex}
\section{Caracteristicas Tecnicas}
% !TeX encoding = UTF-8
% !TeX spellcheck = es_ES
% !TeX root = DccPowerDistribution.tex


\subsection{Diagrama de Bloques}
\begin{figure}[H]
    \centering
    \begin{tikzpicture}
    %\draw [very thin, green] (-1,-3) grid (12,4);
    \begin{scope}[shift={(0,2)}]
        \node [left] (txtCcB) at (0, 1){CC Bloque};
        \pic(picCcB)[rotate=-90] at (1,1)  {screwTerminal} ;
        \node [left](txtCcJ) at (0, 0){CC Jack};
        \pic (picCcJ) at (1,0)  {jack} ;

        \draw[] (txtCcB.east) -- (picCcB-cable-down.center);
        \draw[] (txtCcJ.east) -- (picCcJ-cable.center);
        \node[draw](proteccion) at (4,0.5) {Protección};

        \draw[] (picCcB-cable-up.center) -- (2.25,1) 
            -- (2.25, 0.5) -- (proteccion.west);
        \draw[] (picCcJ-point.center) -- (2.25,0)
            -- (2.25, 0.5);
        \node[](ptVDrive)  at (proteccion.east-| 6,0) {};
        \node[above](lblVdrive) at (ptVDrive.center)  {VDrive};
        \draw (proteccion.east) -- (ptVDrive.center);
        \draw (2.75, 0.5) -- (2.75,1.25) -- (5.25,1.25) 
            -- (5.25,0.5);
        \pic[] at (proteccion.center |- 0,1.25) {SmallDiode};
        \draw[blue, dashed] (3.5,1.25) -- (3.5,1.5) -- (4.5,1.5)
            -- (4.5,1.25);
        \draw[blue, dashed] (proteccion.center |- 0,0)
            -- (5.25,0) -- (5.25,.5);
    \end{scope}

    \begin{scope}
        \node [left] (txtDccIn) at (0, 0){DCC};
        \pic[rotate=-90] (picDccIn) at(1,0){JstXh2};
        \pic[](DccDiode) at (3,0){DiodeBrigde};

        \draw[] (txtDccIn.east) -- (picDccIn-cable.center);
        \draw[] (picDccIn-pcb.center) -- (DccDiode-ac.center);

        \draw[] (DccDiode-cc.center) -- (DccDiode-cc.center -| proteccion.south)
            -- (proteccion.south);
    \end{scope}

    \begin{scope}[shift={(ptVDrive.center |- 0,0)}]
        \pic[](picOutVdrive)  at (0,0){screwTerminal};
        \node[below](lblOutVdrive)  at (picOutVdrive-cable-down.center){Vdrive};
        \draw[] (ptVDrive.center) -- (picOutVdrive-cable-up.center);
        \pic[rotate=-90,yscale=-1](picLedVdrive)  at (-0.75,0){SmallLed=green};
        \draw (0,0.5) -- (picLedVdrive-p.center |- 0,0.5) -- (picLedVdrive-p.center);
        \node[red, scale=.75]at(-0.4,0.5){X};
    \end{scope}

    \begin{scope}[shift={(ptVDrive.center -| 7.5,0)}]
        \node[draw](buck) at (0,0) {Buck}; 
        
        \node[](pt5V)  at (1.5,0) {};
        \node[above](lbl5V) at (pt5V.center)  {+5V};
        \draw (buck.east) -- (pt5V.center);
        \draw (ptVDrive.center) -- (buck.west);
    \end{scope}

    \begin{scope}[shift={(pt5V.center |- 0,0)}]
        \pic[](picOut5V)  at (0,0){screwTerminal};
        \node[below](lblOut5)  at (picOut5V-cable-down.center){+5V};
        \draw[] (pt5V.center) -- (picOut5V-cable-up.center);
        \pic[rotate=-90,yscale=-1](picLed5V)  at (-0.75,0){SmallLed=green};
        \draw (0,0.5) -- (picLed5V-p.center |- 0,0.5) -- (picLed5V-p.center);
        \node[red, scale=.75]at(-0.4,0.5){X};
    \end{scope}


    \begin{scope}[shift={(pt5V.center -| 10.5,0)}]
        \node[draw](ldo) at (0,0) {LDO};  
        \node[](pt33V)  at (1.5,0) {};
        \node[above](lbl33V) at (pt33V.center)  {+3.3V};
        \draw (ldo.east) -- (pt33V.center);
        \draw (pt5V.center) -- (ldo.west);
    \end{scope}

    \begin{scope}[shift={(pt33V.center |- 0,0)}]
        \pic[](picOut33V)  at (0,0){screwTerminal};
        \node[below](lblOut33)  at (picOut33V-cable-down.center){+3.3V};
        \draw[] (pt33V.center) -- (picOut33V-cable-up.center);
        \pic[rotate=-90,yscale=-1](picLed33V)  at (-0.75,0){SmallLed=green};
        \draw (0,0.5) -- (picLed33V-p.center |- 0,0.5) -- (picLed33V-p.center);
        \node[red, scale=.75]at(-0.4,0.5){X};
    \end{scope}

    \begin{scope}[shift={(lblVdrive.center |- 3,-2)}]
        \pic[](dccIface)at (0,0){DccIface};
        
    \end{scope}
    \draw[] (picDccIn-pcb.center -| 1.9,0) -- (1.9,0 |- dccIface-west-out.center) 
        -- (dccIface-west-out.center);
        \draw[] (4,0) -- (4,0 |- dccIface-west-in.center) 
        -- (dccIface-west-in.center);


    \begin{scope}[shift={(lbl5V.center |- dccIface-west.center)}]
        \pic[rotate=90](dccOut)at (0,0){JstXh4};

        \node[right](lblDccOutGnd) at (1,0.6) {GND};
        \node[right](lblDccOut) at (1,0.2) {DCC-TTL};
        \node[right](lblDccOutACK) at (1,-0.2) {DCC-ACK};
        \node[right](lblDccOut5V) at (1,-0.6) {+5V};
        
        \draw[red] (dccOut-c0.center) -- (lblDccOut5V.west);
        \draw[orange] (dccOut-c1.center) -- (lblDccOutACK.west);
        \draw[blue] (dccOut-c2.center) -- (lblDccOut.west);
        \draw[black] (dccOut-c3.center) -- (lblDccOutGnd.west);


        \draw (dccIface-east-out.center) -- (dccOut-p2.center);
        \draw (dccIface-east-in.center) -- (dccOut-p1.center);
        \draw (dccIface-east.center |- dccOut-p0.center) -- (dccOut-p0.center);
    \end{scope}

    \draw[] (pt5V.center |- 0,1.5) -- +(-1.2,0) -- +(-1.2,-3.3) arc(90:-90:.05)
    -- +(0,-.2)  arc(90:-90:.05) -- +(0,-.25);

    \node[red, scale=.75]at(8.4,1.5){X};
    \node[red, scale=.75]at(2.1,0){X};
    \node[red, scale=.75, rotate=90]at(4,1.5){X};
    
\end{tikzpicture}
    \caption{Conectores Dcc}
    \label{fig:BlockDiagrama}
\end{figure}



\subsection{Limites de entrada}
\begin{center}
    \begin{tabular}{ c c c }
        \multicolumn{3}{c}{Entrada DCC} \\ \hline
      & Seguro & Maximo \\ 
     Voltaje & 14-20V & 12-24V \\  
     Corriente & 1A & 1.5A \\
     [1ex] 
     \multicolumn{3}{c}{Entrada Jack y Terminal} \\ \hline
     & Seguro & Maximo \\ 
     Voltaje & 12-20V & 10-24V \\  
     Corriente & 1.5A & 2A \\ \hline
    
    \end{tabular}
\end{center}
\section{Objetivos}
\subsection{Objetivos de este documento}
\subsubitem{Otra}

% \begin{tip}{Test}
% Ejemplo de tip
% \end{tip}
\end{document}