% !TeX encoding = UTF-8
% !TeX spellcheck = es_ES
% !TeX root = DccPowerDistribution.tex

La placa DccPowerDistribution tiene una serie de jumpers y puntos de soldaura/corte
que permiten cambiar ciertas partes y ajustar su comportamiento.

\subsection{Jack o Terminal Atornillable}

El conector Jack se puede conectar directamente a un adaptador AC/DC que tenga salida 
jack 2mm, un conector muy generico usado por diferentes dispositivos electronicos.
Por norma general, solo tienen una salida, por lo que solo se puede conectar a una placa

Por otra parte el terminal Atornillable facilita la conexion cuando ya existe un bus de
corriente continua CC. Con lo que es mas facil conectar varias placas.

Ambos estan conectados en paralelo sin ninguna proteccion, por lo que es necesario
asegurarse que solo uno recibe corriente. La forma más sencilla es solo soldar uno 
de ellos o conectar solo uno.

\begin{figure}[H]
    \centering
    \begin{minipage}{0.25\textwidth}
        \centering
        \begin{tikzpicture}
    
    
    \pic[rotate=-90,transform shape](ac) at(-1,0) {WallAc};
    \pic[rotate=-90,transform shape](board)at (1,0) {SmallBoard};

    \draw[red, line width=4pt,rounded corners=6pt] (ac-jack.center) 
    -- (ac-jack.center |- 0,1.8) 
    -- (board-jack.center|- 0,1.8) 
    -- (board-jack.center);
    %\draw[green] (ac-mains.center) -- (board-terminal.center);

\end{tikzpicture} 
        \caption{Solo Jack}
        \label{fig:VccConnectionJack}
    \end{minipage}
    \hfill
    \begin{minipage}{0.7\textwidth}
        \centering
        \input{tikz/power_terminal.tex}
        \caption{Solo Terminal}
        \label{fig:VccConnectionTerminal}
    \end{minipage}
\end{figure}

Si se sueldan los dos conectores se puede usar una placa como iniciador de bus VccConnectionJack

\begin{figure}[H]
    \centering
    \begin{tikzpicture}
    
    
    \pic[rotate=-90,transform shape](ac) at(-3,0.5) {WallAc};
    \pic[rotate=-90,transform shape](board1)at (-1,0.5) {SmallBoard};

    \draw[red, line width=4pt,rounded corners=6pt] (ac-jack.center) 
    -- (ac-jack.center |- 0,2.8) 
    -- (board1-jack.center|- 0,2.8) 
    -- (board1-jack.center);

    \pic[rotate=-90,transform shape](board2)at (1,-0.5) {SmallBoard};
    \pic[rotate=-90,transform shape](board3)at (3,-0.5) {SmallBoard};
    
    \begin{scope}[shift={(5,-0.5)}]
        \draw[black] (-0.7,-1.3) rectangle +(1.4,2.6);
        \node[] at (0,0.3) {Otro};
        \node[] at (0,-0.3) {Modulo};
    \end{scope}
    
    \draw[red!75, line width=2pt,rounded corners=6pt] (board1-terminal.center -| 2,1.8)
    --(board2-terminal.center -| 2,0)
    --(board2-terminal.center);
    
    \draw[red!75, line width=2pt,rounded corners=6pt] (board1-terminal.center -| 4,1.8)
    --(board3-terminal.center -| 4,0)
    --(board3-terminal.center);

    \draw[red!75, line width=2pt,rounded corners=6pt] (board1-terminal.center -| 6,1.8)
    --(board2-terminal.center -| 6,0)
    --(board2-terminal.center -| 5.7,0);


    % Bus
    \draw[red, line width=4pt,rounded corners=6pt] (board1-terminal.center)
        -- (board1-terminal.center -| 6.5,1.8);
    %\draw[green] (ac-mains.center) -- (board-terminal.center);

\end{tikzpicture} 
    \caption{Jack Iniciando Bus}
    \label{fig:VccConnectionJackBus}
\end{figure}

En el caso de que se use un bus de Coriente Continua, es posible añadir modulos que no sean
de la gama DccDiyTools, pero es necesario asegurarse de que sean compatibles a nivel de voltaje
y que el adaptador de corriente sea capaz de suministrar la corriente necesaria para todos los
modulos conectados.

\subsection{Entrada de corriente}
La corriente puede ser tomada desde el bus DCC o un bus de Corriente Continua (CC)