% !TeX encoding = UTF-8
% !TeX spellcheck = es_ES
% !TeX root = DccPowerDistribution.tex
%!TEX root= DccPowerDistribution.tex

Una vez instalado y al menos una vez al trimestre conviene realizar una serie de comprobaciones.

\subsection{Antes de montar en la maqueta}
Una vez soldados los componentes, nos aseguraremos que no hay cortos y funciona como se espera.

Se recomienda hacer estas comprobaciones antes de realizar los cortes/cortos en los Jumpers que correspondan a la configuracion deseada. Y repetir las pruebas que correspondan una vez configurada la placa.

\begin{itemize}
	\item Apagado y con un tester de continuidad asegurase que de no hay corto circuitos entre los siguientes puntos:

\begin{itemize}
\item GND con VIN(+), VDrive(+), +5V(+), +3.3V(+), DCC, Ack y +5V(arduino). Probar secuencialmente, GND puede obtenerse de los agujeros de montaje o de un Vx(-).
\item  VIN(+), VDrive(+), +5V(+), +3.3V(+), DCC, Ack y +5V(arduino). Entre ellos.
\item Entre los pines del conector DCC
\end{itemize}

\item Connectar y alimentar la placa mediante el Jack DC y medir los voltajes de cada salida:
	\begin{itemize}
	\item \textbf{No realizar} si la placa esta configurada como DCC solo. 
	\item \textbf{VDrive}: Debe ser el voltaje del alimentador menos 0.7V aproximadamente, o la misma si esta configurada como CC solo.
	\item \textbf{+5V}: Debe estar cercano a 5V
	\item \textbf{+3.3V}: Igualmente debe ser cercano a 3.3V.
	\item Revisar que se enciendan los LEDs indicadores. Salvo que sus correspondientes Jumpers (JP5, JP6 y JP7) no esten cortados.
	\item Revisar temperatura de D3, U1, U2 y L1. Tocar un dedo, deben estar templados, si queman revisar pistas por cortos.
	\end{itemize}
\item Desconectar el Jack DC y conectar la señal DCC.
	\begin{itemize}

	\item \textbf{No realizar} si la placa esta configurada como CC solo. O si alguno de los  jumpers JP1, JP2, JP3 o JP4 estan cortados.

	\item Repetir las mediciones, \textbf{+5V} y \textbf{+3.3V} deben ser las mismas (o muy similares) a como estaban. La señal \textbf{VDrive} sera aproximadamente 1.5V menos que el adapador para DCC\sidenote{O si la central lo soporta, el voltaje configurado en esta}
	\item Revisar que se enciendan los LEDs indicadores. Salvo que sus correspondientes Jumpers (JP5, JP6 y JP7) no esten cortados.
\item Medir el voltaje entre Gate (TPx) y Source (TPx) de Q4, este debe aproximadamente VDrive. Tambien se puede medir entre Gate y GND, siendo en este caso practicamente 0.

\item Revisar temperatura de D2, Q4, U1, U2 y L1. Tocar un dedo, deben estar templados, si queman revisar pistas por cortos.

	\end{itemize}
\item Conectar ahora el Jack DC y la señal DCC
	\begin{itemize}
		\item \textbf{Solo} si la placa esta configurada como "<auto"> y si los jumpers JP1, JP2, JP3 y JP4 estan conectados

\item Repetir las mediciones, \textbf{+5V} y \textbf{+3.3V} deben ser las mismas (o muy similares) a como estaban. La señal \textbf{VDrive} sera aproximadamente 0.7V menos que el adaptador CC

\item Revisar que se enciendan los LEDs indicadores. Salvo que sus correspondientes Jumpers (JP5, JP6 y JP7) no esten cortados.
\item Medir el voltaje entre Gate (TPx) y Source (TPx) de Q4, este debe aproximadamente 0V. Tambien se puede medir entre Gate y GND, siendo en este caso practicamente VDcc.

\item Revisar temperatura de D3,D2, Q4, U1, U2 y L1. Tocar un dedo, deben estar templados, si queman revisar pistas por cortos.

	\end{itemize}

\item Conectar un modulo DCC al conector y la entrada DCC a la señal de via.
\begin{itemize}
\item  Para que funcione es necesario que JP1 y JP2 esten conectados. Si se quiere probar ACK, es necesario tener JP3 y JP4 conectados igualmente.
\item Se puede usar un arduino con la libreria DCC y el ejemplo de Turnout. Se recomienda "<modificarlo">\sidenote{Habilitar, que ya esta en el ejemplo} para que muestre log DCC e informacion de programacion en la salida UART del mismo.
\item \textbf{Si JP8} esta conectado, el decoder recibira alimentacion mediante conector Salida Dcc.
\item Enviar comandos DCC y ver que el decoder actua en consecuencia. Si se usa el ejemplo que muestra la informacion en la terminal.
\item Cambiar la entrada DCC a la via de programacion y enviar un comando de lectura de CV9 y ver que en la consola del arduino se muestra informacion coherente 

\end{itemize}

\end{itemize}
\subsection{Despues de Montar en la maqueta}
Cuando tengamos la placa DccPowerDistribution instalada en la maqueta se nos habra complicado
el poder realizar unas pruebas exhaustivas, pero en general se intentaran realizar todas las posibles.

Periodicamente se deberan realizar los siguientes pasos:
\begin{itemize}
\item \textbf{Limpieza}: Quitar el polvo con un trapo, pasandolo suavenmente sobre la placa o con un aspirador adecuado para la maqueta.
Si hay suciedad resistente se puede humedecer ligeramente el trapo con algun limpiador apto para
componentes electronicos. La limpieza se debe hacer \textbf{sin estar alimentada}

\item \textbf{Observar}: Por si hay alguna descoloracion aparante. Por el efecto de calor y/o del tiempo puede llegar apreciarse cambios
en el color de los componentes. Si se aprecia un cambio significativo desde la ultimo mantenimiento,
hay que rehacer todo el proceso de pruebas, para asegurar que sigue funcionado correctamente siendo
lo más exahaustivo posible\sidenote{Desmontala si es preciso}. 

Si \textit{solo es un ligero oscurecimiento hacia el amarillo}, puede darse en los primeros usos y
luego quedarse fijo, siendo no más una cuestion estetica. Pero hay que controlar que sea estable y
no afecte a las pistas a la placa misma. En caso de duda, cambiarla por una placa nueva\sidenote{Si
se repite una vez más, revisa que no se la sobrecarga}

Cualquier otro defecto visual (pistas quemadas, componentes rotos,\dots) quitar la placa y cambiarla por una nueva.
\item \textbf{Test rapido}: Una vez que la placa esta limpia y sin defectos aparentes, conviene
realizar toda la bateria de pruebas. Pero debido a que ya esta instalada, puede no ser posible asi
que como minimo hay que combrobar:
\begin{itemize}
\item Que no haya corto en la entrada CC ni DCC. Con un multimetro comprobar que no hay continuidad entre los dos pines de DCC (entre si) ni en los de CC (entre si)
\item Que no haya corto entre GND y las salidas Vdrive, +5V y 3.3V
\end{itemize}

\end{itemize}

