% !TeX encoding = UTF-8
% !TeX spellcheck = es_ES
% !TeX root = CBus.tex
%!TEX root=CBus.tex

\subsection{Placa - Cable}
Para conectar las placas al bus CBUS tenemos varias opciones validas, y las usaremos segun nos sea util

\subsubsection{Poca Potencia/MERG Screw Terminal Block}
Para las placas de poca potencia (consumo del bus <\SI{100}{\milli\ampere}, o con su propia fuente) podemos usar la solucion de MERG con un bloque terminal tipo MKDS - PHOENIX CONTACT.

El cable a utilizar es el de latiguillo (\SI{0,25}{\milli\metre\squared}/22AWG) usando una ferrula adecuada (Codigo-color). 

\begin{mdframed}
Nota: Sale mas barato comprar 2 de 1x02 que 1 de 1x04 
\end{mdframed}

Tabla de refencias 

\subsection{Media Potencia/MERG Plug Terminal Block}
Cuando la placa requiera mas potencia (Consumo del bus <\SI{500}{\milli\ampere}, ej: esp32 usando Wifi activamente)  podremos usar una version plug 3,5mm o 5mm

Si es posible utilizar el cable de latiguillo (\SI{0,25}{\milli\metre\squared}/22AWG) usando una ferrula adecuada (Codigo), pero si no se puede usar el del bus general
(\SI{0,5}{\milli\metre\squared}/20AWG) con su ferrula adecuada (codigo-color). 

Tabla de refencias 
\subsubsection{Alta Potencia}
Finalmente si la placa requiere mucha potencia, como puede ser un distribuidor de CBUS, o un motor alimentado del bus. Se podra usar la version PCB del conector Cable-Cable o soldar directamente el cable a la PCB
