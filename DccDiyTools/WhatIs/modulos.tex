% !TeX encoding = UTF-8
% !TeX spellcheck = es_ES
% !TeX root = WhatIs.tex
%!TEX root=WhatIs.tex
\textit{DCC DiY Tools} como se ha dicho es un repositorio de informacion sobre
modulos para usar en maquetas de tren y similares, y los tipos de modulos que tiene
bajo su paraguas son de varios tipos, segun sus "<artefactos generados">. Principalmente
son tres tipos:
\begin{itemize}
    \item \textbf{Documentacion}: El resultado solo es uno o varios documentos
    \item \textbf{Modulo Electronico}: Se tendra la documentacion y los ficheros necesarios 
    para poder adquirir y fabricar una placa de circuito impreso. 
    \item \textbf{Objeto Imprimible 3D}: Lo mismos que el anterior, pero para poder 
    imprimir un objeto 3d.
\end{itemize}

Todos los modulos tendran un identificador "<\textbf{AA-SSS}"> dode "<AA"> son las dos
ultimas cifras del año actual y "<SSS"> un número secuencial. A partir de este identificador
se pueden generar idenficadores para cada artefacto.

Los artefactos se identificaran con el esquema "<\textbf{T AA-SSS[-N]}">, donde "<T"> es una
letra que identifica el tipo y "<N"> un número secuencial, empenzando en 1 para el primer
artefacto. Este numero se añadira en el caso de que el modulo se componga de varios
artefactos.

Para que un modulo sea Dcc DiY Tools, es necesario que cumplan las normativas que se
definan en los documentos/modulos correspondientes. Estos documentos contendran 
normas obligatorias, como formatos y otros más laxos como recomendaciones.

Todos los modulos deberan estar publicados bajo una licencia libre, como Creative-Commons,
OSHW, o cualquier otra similar. Y ademas con todos los artefactos necesarios para poder
fabricar o manufacturar el objeto fisico. 