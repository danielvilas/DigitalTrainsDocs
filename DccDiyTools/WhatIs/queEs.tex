% !TeX encoding = UTF-8
% !TeX spellcheck = es_ES
% !TeX root = WhatIs.tex
%!TEX root=WhatIs.tex

En un principio, como la mayoria de los aficionados al tren en miniatura, la aventura empieza con un niño y su ovalo de vias, un transformador o mando analogico, una maquina y un par de vagones\sidenote{Y en los años 90, en españa con Ibertren}. El tiempo pasa y el niño se hace adulto y deja temporalmente la aficion, la vida pasa. 

Pero un dia descubre la plataforma arduino y ese joven ve que puede controlar su tren analogico con un movil y una placa de 10€, Y vuelve la su viejo ovalo.
Durante ese impas, la tecnologia\sidenote{Aplicada al modelismo ferroviario} ha avanzado mucho: Maquinas digitales, control de desvios lentos, Automatizacion, JMRI, DCC, LocoNet, XPress-Net\dots

Ese Joven recuerda sus conocimientos de electronica y se hace unas placas electronicas para alguna cosa s de su maqueta. Y el tiempo vuelve a pasar. Madura\dots Lo que quiere decir que ahora tiene algo de dinero, espacio y un poquito mas de sentido comun, para usarlo en la aficion.

Un dia ve sus diseños, en su maqueta y se da cuenta de que otras personas podrian usarlas en sus maquetas. En ese momento nace Dcc DiY Tools. Un emprendedor diria "<Eureka, ya tengo algo que vender">, pero esa persona no es un emprendedor, es alguien que cree que la ciencia y la informacion debe ser libre.

La transmision libre y publica de informacion hace posible que el mundo avance más rapido.
Esos diseños existen gracias al OSHW\sidenote{Open Source HardWare} por que gente desinteresada ha documentado y hecho publica informacion para que sea accesible para todo el mundo.