% !TeX encoding = UTF-8
% !TeX spellcheck = es_ES
% !TeX root = WhatIs.tex
%!TEX root=WhatIs.tex
Los modulos \textit{DCC DiY Tools} no tienen una tienda oficial, por lo que no
se pueden comprar directamente de la iniciativa actual. Pero, como es un requisito que
todos los modulos esten bajo una licencia libre y conentengan los artefactos necesarios
para su manufactura, cualquiera puede solicitar su fabricacion a una empresa
especializada\sidenote{O hacerlo en su casa si sabe como\dots}.

\subsection{Compra Conjunta}
Para un ahorro de costes se recomienda hacer una compra conjunta por parte de algun
colectivos, como una asociacion de modelismo, foro o similar. Los fabricantes de 
PCBs suelen tener un minimo de placas a pedir\sidenote{Que puede ser pequeño, 5, o 
grande, 100}, por lo que para un individal puede subir mucho el precio para 
una sola placa.

Recordemos que por muy seria que sea la fabrica de PCBs y el origen de los componentes
siempre debemos esperar fallos de calidad\sidenote{O fallos por nuestra parte} y
necesitaremos unos cuantos extras para aseguranos que hay suficientes modulos fabricados
para todos los interesados.

\subsection{Artefactos Documentos}
Estos artefactos por normativa seran PDF con licencia \doclicenseName por lo que 
cualquiera tiene permitido imprimir las copias que necesite, incluso maquetar un libro
y venderlo. 
Las fuentes estaran disponibles en formato \LaTeX, o en su defecto ODT. para que 
cualquiera tambien pueda ajustarlo a sus necesidades.

En la seccion de licencia podras encontrar más informacion al respecto.

En el caso de no querer imprimir muchas paginas, recomendamos contactar con alguna
imprenta, o copisteria, para conseguir un precio más economico. O incluir una pagina
con los QR para acceder a la version en GitHub del documento correspondiente.

\subsection{Artefactos PCB}
El caso de los artefactos que sean placas de circuito impreso en el repositorio Git 
se incluira un fichero ZIP con los gerber necesarios para enviar a un fabricante low
cost en china\sidenote{En estos momentos JLCPCB, pudiendo variar en el tiempo}, asi
como los ficheros de proyecto en un EDA OpenSource\sidenote{KiCad}.

Tambien se acompañara con un xls con el listado de materiales necesarios y donde sea
posible los ficheros extra para PCBA del mismo fabricante low cost.

En general realizar un pedido solo requerira enviar dicho zip utilizando un formulario
y validar las dimensiones junto la posicion de los taladros. En el caso de que se 
necesitara algo especifico se documentaria para dicho artefacto.

\subsection{Artefactos 3D}
Para los artefactos que sean objetos 3D se incluiran los ficheros
correspondientes de un modelador 3D open source como FreeCad, Blender, o Fusion360
\footnote{Lo consideramos como OpenSource por que para proyectos OSS la licencia es 
gratis y es un standard de facto} y los ficheros STL\sidenote{U otros similares}, 
que permitan ser impresos en una impresora 3D "<casera"> o enviados a una fabrica
low cost en china.

Al igual que con los artefactos PCB se incluira documentacion en el caso de ser
necesario.