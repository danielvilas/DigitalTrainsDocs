% !TeX encoding = UTF-8
% !TeX spellcheck = es_ES
% !TeX root = WireAndConnectors.tex
%!TEX root=WireAndConnectors.tex

\subsection{Un poco de teoria}
Un cable\sidenote{En castellano} es un hilo conductor recubierto de un aislante que permite enviar energia y/o señales electricas a larga distancia. Por norma general, un solo conductor no es practico y solemos agrupar varios para que puedan realizar una funcion\sidenote{Ejemplo: el cable de audio tiene tres conductores, izquierda, derecha y comun/masa} y le seguimos llamando cable\footnote{Incidimos: en castellano}.

Dentro de un cable los hilos conductores estan <<relacionados>> de alguna forma, normalente por su responsabilidad dentro de la funcion que realiza el cable.
No se pueden separar, por que de hacerlo, no podra realizar su funcion. Para facilitar su uso en cada extremo del cable se pone un conector con, al menos, tantos pines como cables. Estos cables tienen una capa aislante que los mantienen juntos, a veces de fabrica.

Por ultimo podemos agrupar varios cables en una manguera, Dentro de esta cada cable mantiene su individualidad. Las mangueras pueden tener su capa aislante o pueden sujetarse solo con bridas o similar.

\subsubsection{En el mundo anglosajon}
Hemos insistido que esa terminologia es en el castellano. 

\subsection{Caracteristicas}


\subsection{PortaPilas}
\begin{table}[H]
    \centering
    \renewcommand\theadfont{\bfseries}
    \setlength{\tabcolsep}{10pt}
    \renewcommand{\arraystretch}{1.5}

    \begin{tabular}{|c|c|c|c|c|}
        \beginConnectorTable{Portapilas 2xAA}
        \multirow{5}{*}{\makecell{Cableado }}
  \end{tabular}
    \caption{Porta Pilas 2463}
    \label{tab:pp2463}
\end{table}
